% lang.tex

%%%%%%%%%%%%%%%%%%%%
\begin{frame}{}
  \begin{center}
    词法单元的规约

    \vspace{0.40cm}
    \begin{columns}
      \column{0.30\textwidth}
        \fig{width = 0.70\textwidth}{figs/do-not-understand}
      \column{0.70\textwidth}
        \fig{width = 1.00\textwidth}{figs/token-table}
    \end{columns}

    \vspace{0.60cm}
    我们需要词法单元的\red{\bf 形式化}规约
  \end{center}
\end{frame}
%%%%%%%%%%%%%%%%%%%%

%%%%%%%%%%%%%%%%%%%%
\begin{frame}{}
  \begin{center}
    \id: 字母开头的字母/数字串

    \vspace{0.80cm}
    \id{} 定义了一个集合, 我们称之为\red{\bf 语言 (Language)} \\[15pt]
    它使用了字母与数字等符号集合, 我们称之为\blue{\bf 字母表 (Alphabet)} \\[15pt]
    该语言中的每个元素(即, 标识符)称为\purple{\bf 串 (String)}
  \end{center}
\end{frame}
%%%%%%%%%%%%%%%%%%%%

%%%%%%%%%%%%%%%%%%%%
\begin{frame}{}
  \begin{definition}[字母表]
    \red{\bf 字母表} $\Sigma$ 是一个有限的符号集合。
  \end{definition}

  \fig{width = 0.35\textwidth}{figs/symbols}
\end{frame}
%%%%%%%%%%%%%%%%%%%%

%%%%%%%%%%%%%%%%%%%%
\begin{frame}{}
  \begin{definition}[串]
    字母表 $\Sigma$ 上的\red{\bf 串} ($s$) 是由 $\Sigma$ 中符号构成的一个\blue{\bf 有穷}序列。
  \end{definition}

  \fig{width = 0.30\textwidth}{figs/epsilon}
  \vspace{-0.30cm}
  \[
    \text{空串}: |\epsilon| = 0
  \]
\end{frame}
%%%%%%%%%%%%%%%%%%%%

%%%%%%%%%%%%%%%%%%%%
\begin{frame}{}
  \begin{definition}[串上的``连接''运算]
    \[
      x = dog, y = house \qquad xy = doghouse
    \]

    \[
      s \epsilon = \epsilon s = s
    \]
  \end{definition}

  \pause
  \vspace{0.60cm}
  \begin{definition}[串上的``指数''运算]
    \[
      s^{0} \triangleq \epsilon
    \]

    \[
      s^{i} \triangleq s s^{i-1}, i > 0
    \]
  \end{definition}
\end{frame}
%%%%%%%%%%%%%%%%%%%%

%%%%%%%%%%%%%%%%%%%%
\begin{frame}{}
  \begin{definition}[语言]
    \red{\bf 语言}是给定字母表 $\Sigma$ 上一个任意的\blue{\bf 可数的}串集合。
  \end{definition}

  \[
    \emptyset
  \]

  \[
    \set{\epsilon}
  \]

  \pause
  \[
    \id: \set{a, b, c, a1, a2, \dots}
  \]

  \[
    \ws: \set{\blank, \tab, \line}
  \]

  \[
    \ifkw: \set{\texttt{if}}
  \]
\end{frame}
%%%%%%%%%%%%%%%%%%%%

%%%%%%%%%%%%%%%%%%%%
\begin{frame}{}
  \begin{center}
    \red{\Large \bf 语言是串的集合}

    \vspace{0.80cm}
    因此, 我们可以通过集合操作{\bf 构造}新的语言。
  \end{center}
\end{frame}
%%%%%%%%%%%%%%%%%%%%

%%%%%%%%%%%%%%%%%%%%
\begin{frame}{}
  \begin{columns}
    \column{0.75\textwidth}
      \fig{width = 1.00\textwidth}{figs/language-operator}

      \begin{center}
        $L^{\ast}$ ($L^{+}$) 允许我们构造{\bf 无穷}集合
      \end{center}
    \column{0.25\textwidth}
      \fig{width = 1.00\textwidth}{figs/kleene}
      \begin{center}
        Stephen Kleene (1909 $\sim$ 1994)
      \end{center}
  \end{columns}
\end{frame}
%%%%%%%%%%%%%%%%%%%%

%%%%%%%%%%%%%%%%%%%%
\begin{frame}{}
  \[
    L = \set{A, B, \dots, Z, a, b, \dots, z}
  \]
  \[
    D = \set{0, 1, \dots, 9}
  \]
  \fig{width = 0.70\textwidth}{figs/language-operator}

  \pause
  \[
    L \cup D \qquad LD \qquad L^4 \qquad L^{\ast} \qquad D^{+}
    \qquad \red{L(L \cup D)^{\ast}}
  \]
\end{frame}
%%%%%%%%%%%%%%%%%%%%
