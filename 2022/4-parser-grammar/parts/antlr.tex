% antlr.tex

%%%%%%%%%%%%%%%%%%%%
\begin{frame}{}
  \begin{center}
    {\Large \href{https://github.com/courses-at-nju-by-hfwei/compilers-antlr/blob/main/src/main/antlr/cymbol/Cymbol.g4}{Cymbol.g4}}

    \vspace{0.30cm}
    \begin{columns}
      \column{0.30\textwidth}
        \fig{width = 0.80\textwidth}{figs/call-graph}
        \begin{center}
          函数调用图
        \end{center}
      \column{0.33\textwidth}
        \fig{width = 1.00\textwidth}{figs/talk-cheap}
      \column{0.30\textwidth}
        \fig{width = 0.80\textwidth}{figs/calculator}
        \begin{center}
          迷你计算器
        \end{center}
    \end{columns}
  \end{center}
\end{frame}
%%%%%%%%%%%%%%%%%%%%

%%%%%%%%%%%%%%%%%%%%
\begin{frame}{}
  \begin{center}
    {\Large \href{https://github.com/courses-at-nju-by-hfwei/compilers-antlr/blob/main/src/main/antlr/if-stat/IfStat.g4}{IfStat.g4}}
  \end{center}

  \begin{columns}
	\column{0.40\textwidth}
	  \fig{width = 0.80\textwidth}{figs/girl-woman}
    \begin{center}
      二义性文法
    \end{center}
	\column{0.60\textwidth}
	  \fig{width = 0.90\textwidth}{figs/ifstat-g4}

    \pause
    \vspace{0.30cm}
	  \fig{width = 1.00\textwidth}{figs/ifstat-open-matched-g4}
  \end{columns}
\end{frame}
%%%%%%%%%%%%%%%%%%%%

%%%%%%%%%%%%%%%%%%%%
\begin{frame}{}
  \begin{center}
    Left-Factoring

    \vspace{0.30cm}
    \begin{columns}
      \column{0.50\textwidth}
        \fig{width = 0.90\textwidth}{figs/ifstat-g4}
      \column{0.50\textwidth}
        \fig{width = 0.90\textwidth}{figs/ifstat-factor-g4}
    \end{columns}

    \vspace{0.80cm}
    很明显, 提取左公因子无助于消除文法二义性 \\[12pt]
    而且, ANTLR 4 可以处理有左公因子的文法
  \end{center}
\end{frame}
%%%%%%%%%%%%%%%%%%%%

%%%%%%%%%%%%%%%%%%%%
\begin{frame}{}
  \begin{center}
    {\Large \href{https://github.com/courses-at-nju-by-hfwei/compilers-antlr/blob/main/src/main/antlr/expr/Expr.g4}{Expr.g4}}
  \end{center}

  \vspace{-0.30cm}
  \begin{columns}
	\column{0.40\textwidth}
	  \fig{width = 0.80\textwidth}{figs/girl-woman}
    \begin{center}
      二义性文法
    \end{center}
	\column{0.60\textwidth}
	  \fig{width = 0.40\textwidth}{figs/expr-g4}
    \begin{center}
      结合性、优先级 \pause ({$-, \quad \hat{}\;$})
    \end{center}
  \end{columns}
\end{frame}
%%%%%%%%%%%%%%%%%%%%

%%%%%%%%%%%%%%%%%%%%
\begin{frame}{}
  \begin{columns}
    \column{0.30\textwidth}
    \uncover<2->{
      \fig{width = 0.80\textwidth}{figs/expr-lr-g4}
      \begin{center}
        左递归 (左结合)
      \end{center}
    }
    \column{0.30\textwidth}
      \fig{width = 0.80\textwidth}{figs/expr-g4}
      \vspace{0.20cm}
      \begin{center}
        ANTLR 4 可以处理 (直接) 左递归
      \end{center}
    \column{0.30\textwidth}
    \uncover<3->{
      \fig{width = 1.00\textwidth}{figs/expr-rr-g4}
      \begin{center}
        右递归 (右结合)
      \end{center}
    }
  \end{columns}
\end{frame}
%%%%%%%%%%%%%%%%%%%%

%%%%%%%%%%%%%%%%%%%%
\begin{frame}{}
  \begin{center}
    {\Large \href{https://github.com/courses-at-nju-by-hfwei/compilers-antlr/tree/main/src/main/java/cymbol/callgraph}{Call Graphs}}
  \end{center}

  \vspace{-0.80cm}
	\fig{width = 0.40\textwidth}{figs/call-graph}
\end{frame}
%%%%%%%%%%%%%%%%%%%%

%%%%%%%%%%%%%%%%%%%%
\begin{frame}{}
  \begin{center}
    {\Large \href{https://github.com/courses-at-nju-by-hfwei/compilers-antlr/tree/main/src/main/java/cymbol/calc}{Calculator}}
  \end{center}

	\fig{width = 0.40\textwidth}{figs/calculator}
\end{frame}
%%%%%%%%%%%%%%%%%%%%