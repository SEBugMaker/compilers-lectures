% dfa-to-re.tex

%%%%%%%%%%%%%%%%%%%%
\begin{frame}{}
  \fig{width = 0.70\textwidth}{figs/re-dfa-lexer}

  \vspace{0.30cm}
  \begin{center}
    \blue{DFA $\implies$ RE}
    \[
      \boxed{D \implies r}
    \]
    \[
      \text{要求}: \red{L(r) = L(D)}
    \]
  \end{center}
\end{frame}
%%%%%%%%%%%%%%%%%%%%

%%%%%%%%%%%%%%%%%%%%
\begin{frame}{}
  \begin{center}
    \fig{width = 0.30\textwidth}{figs/dfa-re-wiki}

    \pause
    \vspace{-0.30cm}
    \[
      L(D) = \set{x \mid \exists f \in F_{D}.\; s_{0} \rel{x} f}
    \]
    \pause
    \vspace{-0.30cm}
    \[
      r = \red{\mid}_{x \in L(D)}\;\; x
    \]

    \pause
    \vspace{0.30cm}
    字符串 $x$ 对应于有向图中的路径

    \pause
    \vspace{0.30cm}
    求有向图中所有 (从初始状态到接受状态的) 路径

    \pause
    \vspace{0.30cm}
    \red{但是, 如果有向图中含有{\bf 环}, 则存在无穷多条路径}

    \pause
    \vspace{0.30cm}
    \blue{不要怕, 我们有 Kleene 闭包}
  \end{center}
\end{frame}
%%%%%%%%%%%%%%%%%%%%

%%%%%%%%%%%%%%%%%%%%
\begin{frame}{}
  \begin{center}
    假设有向图中节点编号为 $0$ \purple{(初始状态)} 到 $n - 1$

    \vspace{0.50cm}
    $R^{k}_{ij}:$ 从节点 $i$ 到节点 $j$、且\red{\bf 中间节点}编号 \red{\bf $\le k$} 的所有路径

    \pause
    \fig{width = 0.50\textwidth}{figs/dfa-re-kleene}
    \[
      R^{k}_{ij} = R_{ik}^{k-1} \blue{(R^{k-1}_{kk})^{\ast}} R^{k-1}_{kj}
        \mid R_{ij}^{k-1}
    \]
  \end{center}
\end{frame}
%%%%%%%%%%%%%%%%%%%%

%%%%%%%%%%%%%%%%%%%%
\begin{frame}{}
  \begin{center}
    \red{$Q:$ 如何初始化?}

    \vspace{0.30cm}
    $R^{-1}_{ij}:$ \pause 从节点 $i$ 到节点 $j$、且\red{\bf 不经过中间节点}的所有路径
  \end{center}

  \pause
  \vspace{0.30cm}
  \begin{columns}
    \column{0.50\textwidth}
      \fig{width = 0.70\textwidth}{figs/i-j}
      \[
        R^{-1}_{ij} = a_{1} | a_{2} | \dots | a_{m}
      \]

      \fig{width = 0.60\textwidth}{figs/i-j-no-edge}
      \[
        R^{-1}_{ij} = \red{\emptyset}\; \blue{\text{(``无路可走'')}}
      \]
    \column{0.50\textwidth}
      \pause
      \fig{width = 0.60\textwidth}{figs/i-loop}
      \[
        R^{-1}_{ii} = a_{1} | a_{2} | \dots | a_{m} | \red{\epsilon}
      \]

      \fig{width = 0.15\textwidth}{figs/i}
      \[
        R^{-1}_{ii} = \red{\epsilon}\; \blue{\text{(``无需走路'')}}
      \]
  \end{columns}
\end{frame}
%%%%%%%%%%%%%%%%%%%%

%%%%%%%%%%%%%%%%%%%%
\begin{frame}{}
  \begin{center}
    关于 \red{$\emptyset$} (注意: 它不是正则表达式) 的规定
    \[
      \emptyset r = r \emptyset = \emptyset
    \]

    \[
      \emptyset | r = r
    \]
  \end{center}
\end{frame}
%%%%%%%%%%%%%%%%%%%%

%%%%%%%%%%%%%%%%%%%%
\begin{frame}{}
  \begin{center}
    \red{$Q:$ $r$ 的最终结果是什么?}

    \vspace{0.50cm}
    求有向图中所有 (从初始状态到接受状态的) 路径

    \pause
    \vspace{0.50cm}
    \[
      r = |_{\red{s_{j} \in F_{D}}} R^{\blue{|S_{D}| - 1}}_{0j}
    \]
  \end{center}
\end{frame}
%%%%%%%%%%%%%%%%%%%%

%%%%%%%%%%%%%%%%%%%%
\begin{frame}{}
  \fig{width = 0.60\textwidth}{figs/dfa-re-alg}

  \begin{center}
    $|D|$: 状态数 ($|S_{D}|$); \qquad $D_{A}$: 接受状态集合 ($F_{D}$)
  \end{center}
\end{frame}
%%%%%%%%%%%%%%%%%%%%

%%%%%%%%%%%%%%%%%%%%
\begin{frame}{}
  \fig{width = 0.30\textwidth}{figs/dfa-re-wiki}

  \vspace{-0.50cm}
  \fig{width = 0.12\textwidth}{figs/dfa-re-init}
\end{frame}
%%%%%%%%%%%%%%%%%%%%

%%%%%%%%%%%%%%%%%%%%
\begin{frame}{}
  \fig{width = 0.30\textwidth}{figs/dfa-re-wiki}

  \vspace{-0.50cm}
  \fig{width = 0.65\textwidth}{figs/dfa-re-step-0}
\end{frame}
%%%%%%%%%%%%%%%%%%%%

%%%%%%%%%%%%%%%%%%%%
\begin{frame}{}
  \fig{width = 0.30\textwidth}{figs/dfa-re-wiki}

  \vspace{-0.50cm}
  \fig{width = 0.70\textwidth}{figs/dfa-re-step-1}
\end{frame}
%%%%%%%%%%%%%%%%%%%%

%%%%%%%%%%%%%%%%%%%%
\begin{frame}{}
  \[
	a^{\ast}b(a(a|b)|b)^{\ast}
  \]
  \fig{width = 0.30\textwidth}{figs/dfa-re-wiki}

  \vspace{-0.50cm}
  \fig{width = 0.90\textwidth}{figs/dfa-re-step-2}
\end{frame}
%%%%%%%%%%%%%%%%%%%%