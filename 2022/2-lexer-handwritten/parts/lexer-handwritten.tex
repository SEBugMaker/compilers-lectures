% lexer-handwritten.tex

%%%%%%%%%%%%%%%%%%%%
\begin{frame}{}
  \fig{width = 0.60\textwidth}{figs/by-hand}

  \begin{center}
    手写词法分析器
  \end{center}
\end{frame}
%%%%%%%%%%%%%%%%%%%%

%%%%%%%%%%%%%%%%%%%%
\begin{frame}{}
  \begin{center}
    \fig{width = 0.80\textwidth}{figs/tokens}

    \pause
    \vspace{0.60cm}
    \texttt{\teal{DragonLexerGrammar.g4}}
  \end{center}
\end{frame}
%%%%%%%%%%%%%%%%%%%%

%%%%%%%%%%%%%%%%%%%%
\begin{frame}{}
  \begin{center}
    识别字符串$s$中符合\red{\bf 特定}词法单元模式的\blue{\bf 前缀}词素

    \vspace{0.80cm}
    \uncover<2->{
      \blue{\fbox{\red{\bf 分支:} 先判断属于哪一类, 然后进入特定词法单元的前缀词素匹配流程}}

      \vspace{0.80cm}
      识别字符串$s$中符合\red{\bf 某种}词法单元模式的\blue{\bf 前缀}词素
    }

    \vspace{0.80cm}
    \uncover<3->{
      \blue{\fbox{\red{\bf 循环:} 返回当前识别出来的词法单元与词素, 继续识别下一个前缀词素}}
    }

    \vspace{0.60cm}
    \uncover<1->{识别字符串$s$中符合\red{\bf 某种}词法单元模式的\blue{\bf 所有}词素}
  \end{center}
\end{frame}
%%%%%%%%%%%%%%%%%%%%

%%%%%%%%%%%%%%%%%%%%
\begin{frame}{}
  \begin{center}
    先: \ws \quad \ifkw \quad \elsekw \quad \id \quad \intnum

    \vspace{1.00cm}
    然后: \relop

    \vspace{1.00cm}
    最后: \floatnum \quad \scinum
  \end{center}
\end{frame}
%%%%%%%%%%%%%%%%%%%%

%%%%%%%%%%%%%%%%%%%%
\begin{frame}{}
  \begin{center}
    用于识别 \blue{\ws} 的状态转移图
    \fig{width = 0.50\textwidth}{figs/ws}

    \pause
    \vspace{0.60cm}
    \red{$\ast$:} 识别出的空白符\blue{\bf 不包含}当前\texttt{peek}指向的字符

    \pause
    \vspace{0.40cm}
    \red{$22:$} 碰到\texttt{other}怎么办?
  \end{center}
\end{frame}
%%%%%%%%%%%%%%%%%%%%

%%%%%%%%%%%%%%%%%%%%
\begin{frame}{}
  \begin{center}
    用于识别 \blue{\id} 的状态转移图

    \vspace{0.60cm}
    \fig{width = 0.60\textwidth}{figs/id}

    \pause
    \vspace{0.40cm}
    \red{$9:$} 碰到\texttt{other}怎么办?
  \end{center}
\end{frame}
%%%%%%%%%%%%%%%%%%%%

%%%%%%%%%%%%%%%%%%%%
\begin{frame}{}
  \begin{center}
    用于识别 \blue{\intnum} 的状态转移图
    \fig{width = 0.40\textwidth}{figs/intnum}

    \pause
    \vspace{0.40cm}
    \red{$12:$} 碰到\texttt{other}怎么办?
  \end{center}
\end{frame}
%%%%%%%%%%%%%%%%%%%%

%%%%%%%%%%%%%%%%%%%%
\begin{frame}{}
  \begin{center}
    识别字符串$s$中符合\red{\bf 某种}词法单元模式的\blue{\bf 前缀}词素
    \blue{($\Call{nextToken}{\null}$)}
  \end{center}

  \begin{columns}
    \column{0.45\textwidth}
      \begin{center}
        \fig{width = 0.80\textwidth}{figs/ws}
        \ws: 空白符

        \vspace{0.20cm}
        \fig{width = 0.60\textwidth}{figs/intnum}
        \intnum: 整数
      \end{center}
    \column{0.55\textwidth}
      \begin{center}
        \fig{width = 1.00\textwidth}{figs/id}
        \id: 标识符

        \vspace{0.40cm}
        \fig{width = 0.60\textwidth}{figs/error-code}
        错误处理模块
      \end{center}
  \end{columns}

  \pause
  \vspace{0.30cm}
  \begin{center}
    \red{\bf 关键点:} 合并$22, 12, 9$, 根据\purple{\bf 下一个字符}即可判定词法单元的类型 \\[8pt]
    否则, 调用错误处理模块(对应\texttt{other}), 报告\purple{\bf 该字符有误}, 并忽略该字符
  \end{center}
\end{frame}
%%%%%%%%%%%%%%%%%%%%

%%%%%%%%%%%%%%%%%%%%
\begin{frame}{}
  \begin{center}
    用于识别 \blue{\relop} 的状态转移图
    \fig{width = 0.50\textwidth}{figs/relop}

    \pause
    \vspace{0.30cm}
    \blue{\bf ``最长优先原则'':} 例如, 识别出\texttt{<=}, 而不是\texttt{<}与\texttt{=}

    \pause
    \vspace{0.30cm}
    \red{$0:$} 碰到\texttt{other}怎么办?
  \end{center}
\end{frame}
%%%%%%%%%%%%%%%%%%%%

%%%%%%%%%%%%%%%%%%%%
\begin{frame}{}
  \begin{center}
    识别字符串$s$中符合\red{\bf 某种}词法单元模式的\blue{\bf 前缀}词素
    \blue{($\Call{nextToken}{\null}$)}
  \end{center}

  \begin{columns}
    \column{0.45\textwidth}
      \begin{center}
        \fig{width = 0.80\textwidth}{figs/ws}
        \fig{width = 0.60\textwidth}{figs/intnum}
        \fig{width = 0.80\textwidth}{figs/id}
      \end{center}
    \column{0.55\textwidth}
      \begin{center}
        \fig{width = 0.70\textwidth}{figs/relop}
        \fig{width = 0.60\textwidth}{figs/error-code}
      \end{center}
  \end{columns}

  \pause
  \vspace{0.30cm}
  \begin{center}
    \red{\bf 关键点:} 合并$22, 12, 9, 0$, 根据\purple{\bf 下一个字符}即可判定词法单元的类型 \\[8pt]
    否则, 调用错误处理模块(对应\texttt{other}), 报告\purple{\bf 该字符有误}, 并忽略该字符
  \end{center}
\end{frame}
%%%%%%%%%%%%%%%%%%%%

%%%%%%%%%%%%%%%%%%%%
\begin{frame}{}
  \begin{center}
    但是, 词法分析器的设计并没有这么容易

    \vspace{0.50cm}
    \fig{width = 0.50\textwidth}{figs/life-always-hard}
  \end{center}
\end{frame}
%%%%%%%%%%%%%%%%%%%%

%%%%%%%%%%%%%%%%%%%%
\begin{frame}{}
  \begin{center}
    \ws \quad \ifkw \quad \elsekw \quad \id \quad \intnum \quad \relop

    \vspace{0.80cm}
    根据\purple{\bf 下一个字符}即可判定词法单元的类型

    \vspace{0.50cm}
    每个状态转移图的每个状态要么是\purple{\bf 接受状态}, 要么带有 \purple{\bf \texttt{other}边}

    \pause
    \vspace{1.20cm}
    \red{如何同时识别\; \intnum{}、\floatnum{} 与 \scinum?}
  \end{center}
\end{frame}
%%%%%%%%%%%%%%%%%%%%

%%%%%%%%%%%%%%%%%%%%
\begin{frame}{}
  \begin{columns}
    \column{0.50\textwidth}
      \fig{width = 0.70\textwidth}{figs/intnum}
      \begin{center}
        \intnum: 整数
      \end{center}
    \column{0.50\textwidth}
      \pause
      \fig{width = 1.00\textwidth}{figs/realnum}
      \begin{center}
        \floatnum: 浮点数 (无科学计数法)\\
        (不识别 \texttt{2.})
      \end{center}
  \end{columns}

  \pause
  \vspace{1.00cm}
  \fig{width = 0.80\textwidth}{figs/scinum}
  \begin{center}
    \scinum: 带科学计数法的浮点数 \\
    (\texttt{2.99792458E8 \quad 3E8})
  \end{center}
\end{frame}
%%%%%%%%%%%%%%%%%%%%

%%%%%%%%%%%%%%%%%%%%
\begin{frame}{}
  \begin{center}
    \num:
    整数部分\blue{[\texttt{.}可选的小数部分]}\purple{[\texttt{E}[可选的\texttt{+-}]可选的指数部分]}

    \pause
    \vspace{0.60cm}
    \fig{width = 0.80\textwidth}{figs/number}
    \red{$19, 20, 21:$} 代表了不同的数字类型

    \pause
    \vspace{0.80cm}
    \red{$12:$} 碰到\texttt{other}怎么办?
    \pause {\hfill \small (尝试其它词法单元或进入错误处理模块)}

    \pause
    \vspace{0.30cm}
    \red{$14, 16, 17:$} 碰到\texttt{other}怎么办?
    \pause
    {\hfill (回退, 寻找\red{\bf 最长匹配})}
  \end{center}
\end{frame}
%%%%%%%%%%%%%%%%%%%%

%%%%%%%%%%%%%%%%%%%%
\begin{frame}{}
  \begin{center}
    识别字符串$s$中符合\red{\bf 某种词法单元模式}的{\bf 前缀词素}
    \blue{($\Call{scan}{\null}$)}
  \end{center}

  \begin{columns}
    \column{0.50\textwidth}
      \begin{center}
        \fig{width = 0.80\textwidth}{figs/ws}
        \fig{width = 1.00\textwidth}{figs/number}
        \fig{width = 0.80\textwidth}{figs/id}
      \end{center}
    \column{0.50\textwidth}
      \begin{center}
        \fig{width = 0.70\textwidth}{figs/relop}
        \fig{width = 0.60\textwidth}{figs/error-code}
      \end{center}
  \end{columns}

  \pause
  \vspace{0.30cm}
  \begin{center}
    \red{\bf 关键点:} 合并$22, 12, 9, 0$, 根据\purple{\bf 下一个字符}即可判定词法单元的类型 \\[4pt]
    否则, 调用错误处理模块(对应\texttt{other}), 报告\purple{\bf 该字符有误}, 忽略该字符。 \\[4pt]
    注意, 在 \floatnum{} 与 \scinum{}中, 有时需要\purple{\bf 回退}, 寻找最长匹配。
  \end{center}
\end{frame}
%%%%%%%%%%%%%%%%%%%%