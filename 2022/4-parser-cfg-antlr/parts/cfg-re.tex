% cfg-re.tex

%%%%%%%%%%%%%%%%%%%%
\begin{frame}{}
  \begin{center}
    为什么不使用优雅、强大的\blue{\bf 正则表达式}描述程序设计语言的语法?

    \vspace{0.50cm}
    \fig{width = 0.40\textwidth}{figs/grammar-hierarchy}
    正则表达式的表达能力\red{\bf 严格弱于}上下文无关文法
  \end{center}
\end{frame}
%%%%%%%%%%%%%%%%%%%%

%%%%%%%%%%%%%%%%%%%%
\begin{frame}{}
  \begin{center}
    每个\blue{正则表达式} $r$ 对应的语言 $L(r)$ 都可以使用\blue{上下文无关文法}来描述

    \[
      r = (a | b)^{\ast} abb
    \]

    \pause
    \fig{width = 0.50\textwidth}{figs/nfa-abb}

    \pause
    \fig{width = 0.40\textwidth}{figs/nfa-abb-cfg}
    此外, 若 $\delta(A_i, \epsilon) = A_{j}$, 则添加 $A_{i} \to A_{j}$
  \end{center}
\end{frame}
%%%%%%%%%%%%%%%%%%%%

%%%%%%%%%%%%%%%%%%%%
\begin{frame}{}
  \begin{center}
    % cfg-anbn.tex

\begin{empheq}[box=\widefbox]{align*}
  S &\to aSb \\[8pt]
  S &\to \epsilon
\end{empheq}

    \[
      L = \set{a^{n} b^{n} \mid n \ge 0}
    \]
    该语言\red{\bf 无法}使用正则表达式来描述
  \end{center}
\end{frame}
%%%%%%%%%%%%%%%%%%%%

%%%%%%%%%%%%%%%%%%%%
\begin{frame}{}
  \begin{theorem}
    $L = \set{a^{n} b^{n} \mid n \ge 0}$ 无法使用正则表达式描述。
  \end{theorem}

  \pause
  \begin{center}
    \red{\bf 反证法}

    \pause
    \vspace{0.30cm}
    假设存在正则表达式 $r$: $L(r) = L$

    \pause
    \vspace{0.30cm}
    则存在\blue{\bf 有限}状态自动机 $D(r)$: $L(D(r)) = L$; 设其状态数为 $k$

    \pause
    \vspace{0.30cm}
    \fbox{\red{考虑输入 $a^{m} (m > k)$}}
    \fig{width = 0.90\textwidth}{figs/pumping-lemma-anbn}

    \pause
    \vspace{0.30cm}
    $D(r)$ 也能接受 $a^{i+j} b^{i}$; \red{\bf 矛盾!}
  \end{center}
\end{frame}
%%%%%%%%%%%%%%%%%%%%

%%%%%%%%%%%%%%%%%%%%
\begin{frame}{}
  \begin{center}
    \[
      L = \set{a^{n}b^{n} \mid n \ge 0}
    \]
    \href{https://en.wikipedia.org/wiki/Pumping\_lemma\_for\_regular\_languages}{
      \teal{Pumping Lemma for \red{Regular Languages}}}

    \pause
    \vspace{0.60cm}
    \[
      L = \set{a^{n}b^{n}c^{n} \mid n \ge 0}
    \]
    \href{https://en.wikipedia.org/wiki/Pumping\_lemma\_for\_context-free\_languages}{
      \teal{Pumping Lemma for \red{Context-free Languages}}}
  \end{center}
\end{frame}
%%%%%%%%%%%%%%%%%%%%
