% course.tex

%%%%%%%%%%%%%%%%%%%%
\begin{frame}{}
  \begin{columns}
    \column{0.50\textwidth}
      \fig{width = 0.70\textwidth}{figs/discrete-mathematics}
    \column{0.50\textwidth}
      \fig{width = 0.85\textwidth}{figs/compiler-construction}
  \end{columns}

  \vspace{0.50cm}
  \begin{quote}
    \centering
    {\large 吾与离散数学, 孰难?}

    \pause
    \vspace{0.50cm}
    对于本学期课程,编译器实践比编译器理论更重要
  \end{quote}
\end{frame}
%%%%%%%%%%%%%%%%%%%%

%%%%%%%%%%%%%%%%%%%%
\begin{frame}{}
  \begin{center}
    {\large $7$ 周 = $14$ 次课 \red{$<$} \gray{$8$ 周 = $16$ 次课}}

    \fig{width = 0.50\textwidth}{figs/12-21}

    \vspace{0.30cm}
    {\large 本学期课程的考核方式做了较大调整}
  \end{center}
\end{frame}
%%%%%%%%%%%%%%%%%%%%

%%%%%%%%%%%%%%%%%%%%
\begin{frame}{}
  \begin{columns}
    \column{0.10\textwidth}
    \column{0.80\textwidth}
      \begin{description}
        \setlength{\itemsep}{25pt}
        \item[{\bf 平时作业 ($15$ 分):}] $7 + 1$ 次作业, 每周 $1 \sim 2$ 题
        \item[{\bf 期末测试 ($40$ 分):}] 考试周统一安排; $2$ 小时; 开卷
        \item[{\bf 课程实验 ($45$ 分):}] $4$ 次必做实验 + $1$ 次选做实验 \blue{($5$ 分)}
      \end{description}
    \column{0.10\textwidth}
  \end{columns}

  \pause
  \vspace{1.00cm}
  \begin{center}
    \red{\large 这是去年的啦 $\dots$}
  \end{center}
\end{frame}
%%%%%%%%%%%%%%%%%%%%

%%%%%%%%%%%%%%%%%%%%
\begin{frame}{}
  \begin{columns}
    \column{0.10\textwidth}
    \column{0.80\textwidth}
      \begin{description}[<+->]
        \setlength{\itemsep}{25pt}
        \item[\cyan{\bf 平时作业 ($00$ 分):}] $7$ 次作业, 每周 $1 \sim 2$ 题
        \item[\red{\bf 课程实验 ($60$ 分):}] $7$ 次必做实验 + $1$ 次选做实验 \blue{($5$ 分)}
        \item[\red{\bf 期末测试 ($40$ 分):}] 考试周统一安排; $2$ 小时; \blue{开卷}
        \item[\teal{\bf 附加作业 ($05$ 分):}] 报告 + 录屏方式
      \end{description}
    \column{0.10\textwidth}
  \end{columns}
\end{frame}
%%%%%%%%%%%%%%%%%%%%

%%%%%%%%%%%%%%%%%%%%
\begin{frame}{}
  \begin{center}
    每周三晚上发布作业 \qquad 下周三 \blue{$23:55$} 前\teal{自愿提交}作业
  \end{center}

  \begin{columns}
    \column{0.50\textwidth}
    \fig{width = 0.80\textwidth}{figs/square-logo}
    \column{0.50\textwidth}
    \fig{width = 0.60\textwidth}{figs/2022-Compilers-square-qrcode}
    \vspace{-0.80cm}
    \begin{center}
      \purple{邀请码: 5JP57Q5H}
    \end{center}
  \end{columns}

  \vspace{0.30cm}
  \begin{center}
    发布阅读材料 \qquad 发布调查问卷 \qquad 发布作业答案
  \end{center}
\end{frame}
%%%%%%%%%%%%%%%%%%%%

%%%%%%%%%%%%%%%%%%%%
\begin{frame}{}
  \begin{center}
    课程实验:开发 \href{https://compiler.educg.net/}{\textsf{SysY}} 语言编译器

    \fig{width = 0.80\textwidth}{figs/huawei-bisheng}

    \vspace{0.30cm}
    \red{小步迭代、多步迭代、贴合课堂讲授进度}
  \end{center}
\end{frame}
%%%%%%%%%%%%%%%%%%%%

%%%%%%%%%%%%%%%%%%%%
\begin{frame}{}
  \begin{center}
    \red{\bf L0:} \red{\bf 环境配置} 今晚 18:00 发布 \\[5pt]
    ($20221107 \sim 20221113$)

    \vspace{0.10cm}
    \fig{width = 0.40\textwidth}{figs/compilers-labs}

    \pause
    \vspace{0.10cm}
    \red{\bf L1:} \red{\bf 词法分析} 本周三 18:00 发布 \\[5pt]
    ($20221109 \sim 20221116$)
  \end{center}
\end{frame}
%%%%%%%%%%%%%%%%%%%%

%%%%%%%%%%%%%%%%%%%%
\begin{frame}{}
  \begin{center}
    附加作业: 报告 + 录屏方式,了解更现代的编译器原理与技术 \\[6pt]

    \fig{width = 0.45\textwidth}{figs/compilers-papers-we-love}
  \end{center}
\end{frame}
%%%%%%%%%%%%%%%%%%%%

%%%%%%%%%%%%%%%%%%%%
\begin{frame}
  \begin{center}
    鼓励讨论,但需独立编码完成课程实验
    \fig{width = 0.80\textwidth}{figs/no-plagiarism}
  \end{center}

  \pause
  \begin{columns}
    \column{0.20\textwidth}
    \column{0.60\textwidth}
    \begin{description}
      \setlength{\itemsep}{6pt}
      \item[课程实验:] 抄袭者当次实验计 $0$ 分
      \item[附加作业:] 抄袭者当次作业计 $0$ 分
    \end{description}
    \column{0.20\textwidth}
  \end{columns}
\end{frame}
%%%%%%%%%%%%%%%%%%%%

%%%%%%%%%%%%%%%%%%%%
\begin{frame}
  \begin{center}
    请\red{\bf 经常}使用 Git/GitHub 保留代码编写记录 (OJ 系统原生支持)
    \fig{width = 0.40\textwidth}{figs/i-am-serious}

    \vspace{0.50cm}
    如果你不确定自己的行为是否构成抄袭, 请\red{\bf 事先}咨询助教。
  \end{center}
\end{frame}
%%%%%%%%%%%%%%%%%%%%

%%%%%%%%%%%%%%%%%%%%
\begin{frame}{}
  \begin{columns}
    \column{0.50\textwidth}
    \begin{center}
      QQ 群号: \blue{\bf 755783220}

      \fig{width = 0.60\textwidth}{figs/2022-Compilers-qq}
    \end{center}
    \column{0.50\textwidth}
    \begin{center}
      {\bf \teal{助教:}} 夏宇、潘昱光、顾龙、$\dots$
    \end{center}
  \end{columns}
\end{frame}
%%%%%%%%%%%%%%%%%%%%

%%%%%%%%%%%%%%%%%%%%
\begin{frame}{}
  \begin{center}
    \url{http://47.96.123.231:8081} \\[5pt]

    \fig{width = 0.30\textwidth}{figs/docs-compilers}

    编译原理课程网站,请\red{\bf 收藏}并及时关注网站更新
  \end{center}
\end{frame}
%%%%%%%%%%%%%%%%%%%%

%%%%%%%%%%%%%%%%%%%%
\begin{frame}{}
  \begin{center}
    \fig{width = 0.50\textwidth}{figs/compilers-lectures-2022}

    \vspace{0.50cm}
    \url{https://github.com/courses-at-nju-by-hfwei/compilers-lectures/tree/master/2022}
  \end{center}
\end{frame}
%%%%%%%%%%%%%%%%%%%%

%%%%%%%%%%%%%%%%%%%%
\begin{frame}{}
  \begin{columns}
    \column{0.50\textwidth}
    \fig{width = 0.60\textwidth}{figs/dragon-book}
    \begin{center}
      也可使用\blue{\bf ``本科教学版''}
    \end{center}
    \column{0.50\textwidth}
    \begin{center}
      (本书仅供参考)
      \fig{width = 0.80\textwidth}{figs/lab-book}
      \url{https://cs.nju.edu.cn/changxu/2_compiler/index.html} \\[3pt]
      (本学期对课程实验做了大幅改动)
    \end{center}
  \end{columns}
\end{frame}
%%%%%%%%%%%%%%%%%%%%

%%%%%%%%%%%%%%%%%%%%
\begin{frame}{}
  \begin{columns}
    \column{0.33\textwidth}
    \fig{width = 0.80\textwidth}{figs/flex}
    \begin{center}
      \href{https://en.wikipedia.org/wiki/Flex_(lexical_analyser_generator)}{\footnotesize Flex: 词法分析器生成器}
    \end{center}
    \column{0.33\textwidth}
    \fig{width = 0.80\textwidth}{figs/lex-yacc-book}
    \column{0.33\textwidth}
    \fig{width = 0.80\textwidth}{figs/bison}
    \begin{center}
      \href{https://en.wikipedia.org/wiki/GNU_Bison}{\footnotesize Bison: 语法分析器生成器}
    \end{center}
  \end{columns}

  \vspace{0.50cm}
  \begin{center}
    不够现代, 本学期课程实验\red{\bf 不再支持}这些工具
  \end{center}
\end{frame}
%%%%%%%%%%%%%%%%%%%%

%%%%%%%%%%%%%%%%%%%%
\begin{frame}{}
  \begin{columns}
    \column{0.40\textwidth}
    \fig{width = 1.00\textwidth}{figs/antlr-logo}
    \begin{center}
      (Since 1988)
    \end{center}
    \column{0.60\textwidth}
    \fig{width = 0.50\textwidth}{figs/parr.jpeg}
    \begin{center}
      \href{https://parrt.cs.usfca.edu/}{\small Terence Parr (University of San Francisco)}
    \end{center}
  \end{columns}

  \vspace{0.80cm}
  \begin{center}
    \url{https://www.antlr.org/index.html} \\[5pt]
    \url{https://www.antlr.org/tools.html} (\red{IntelliJ Plugin})
  \end{center}
\end{frame}
%%%%%%%%%%%%%%%%%%%%

%%%%%%%%%%%%%%%%%%%%
\begin{frame}{}
  \begin{center}
    在线玩: \url{http://lab.antlr.org/}
  \end{center}

  \vspace{0.80cm}
  \begin{columns}
    \column{0.15\textwidth}
    \column{0.70\textwidth}
    \begin{description}[arithmetic:]
      \setlength{\itemsep}{10pt}
      \item[arithmetic:] \texttt{number1.txt, simple2.txt}
      \item[fol:] \texttt{example1.txt}
      \item[guitartab:] \texttt{example1.txt}
      \item[xyz:] \texttt{example1.txt}
      \item[C:] \texttt{add.c}
    \end{description}
    \column{0.15\textwidth}
  \end{columns}
\end{frame}
%%%%%%%%%%%%%%%%%%%%

%%%%%%%%%%%%%%%%%%%%
\begin{frame}{}
  \begin{columns}
    \column{0.50\textwidth}
    \fig{width = 0.70\textwidth}{figs/antlr4-book-en}
    \column{0.50\textwidth}
    \fig{width = 0.60\textwidth}{figs/antlr4-book-ch}
  \end{columns}

  \vspace{0.50cm}
  \begin{center}
    基于 ANTLR 4,是课程实验指导的\red{\bf 重要}参考资料
  \end{center}
\end{frame}
%%%%%%%%%%%%%%%%%%%%

%%%%%%%%%%%%%%%%%%%%
\begin{frame}{}
  \begin{columns}
    \column{0.50\textwidth}
    \fig{width = 0.65\textwidth}{figs/patterns-book-en}
    \column{0.50\textwidth}
    \fig{width = 0.70\textwidth}{figs/patterns-book-ch}
  \end{columns}

  \vspace{0.50cm}
  \begin{center}
    基于 ANTLR 3,与 ANTLR 4 相比,有些过时,\\[3pt]
    但可以看作理解 ANTLR 4 的基础
  \end{center}
\end{frame}
%%%%%%%%%%%%%%%%%%%%

%%%%%%%%%%%%%%%%%%%%
\begin{frame}{}
  \fig{width = 0.40\textwidth}{figs/how-to-develop-a-compiler-book}
  \begin{center}
    ``从零开始制作真正的编译器'',对课程实验很有帮助,\red{\bf 强烈推荐}
  \end{center}
\end{frame}
%%%%%%%%%%%%%%%%%%%%

%%%%%%%%%%%%%%%%%%%%
\begin{frame}{}
  \fig{width = 0.50\textwidth}{figs/llvm-cookbook-book}

  \begin{center}
    本学期课程实验引入了 LLVM (\url{https://llvm.org/})
  \end{center}
\end{frame}
%%%%%%%%%%%%%%%%%%%%

%%%%%%%%%%%%%%%%%%%%
\begin{frame}{}
  \fig{width = 0.50\textwidth}{figs/llvm-logo}

  \vspace{0.20cm}
  \begin{center}
    \href{https://www.bilibili.com/video/BV1RF411K7F5/?vd_source=e3cbbf5ca80db268fa006d63626e267e}{LLVM 简介 @ Bilibili}
  \end{center}
\end{frame}
%%%%%%%%%%%%%%%%%%%%