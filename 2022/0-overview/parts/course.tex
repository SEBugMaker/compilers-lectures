% course.tex

%%%%%%%%%%%%%%%%%%%%
\begin{frame}{}
  \begin{center}
    {\large $7$ 周 = $14$ 次课 \red{$<$} \gray{$8$ 周 = $16$ 次课}}
  \end{center}

  \fig{width = 0.50\textwidth}{figs/12-21}
\end{frame}
%%%%%%%%%%%%%%%%%%%%

%%%%%%%%%%%%%%%%%%%%
\begin{frame}{}
  \begin{description}[<+->]
    \setlength{\itemsep}{25pt}
    \item[\red{\bf 作业 ($0$ 分):}] $\approx 7$ 次作业, 每周 $1 \sim 2$ 题
    \item[\red{\bf 实验 ($60$ 分):}] $\approx 8$ 次必做实验 + $1$ 次选做实验 \blue{($5$ 分)}
    \item[\red{\bf 期末测试 ($40$ 分):}] 考试周统一安排; $2$ 小时; \blue{开卷}
    \item[\red{\bf 附加作业 ($5$ 分):}] 报告 + 录屏方式, 学习更现代的编译原理与技术
  \end{description}
\end{frame}
%%%%%%%%%%%%%%%%%%%%

%%%%%%%%%%%%%%%%%%%%
\begin{frame}
  \begin{center}
    \fig{width = 0.80\textwidth}{figs/no-plagiarism}
  \end{center}

  \begin{columns}
    \column{0.20\textwidth}
    \column{0.60\textwidth}
    \begin{description}
      \setlength{\itemsep}{6pt}
      \item[实验:] 当次实验计零分
      \item[附加项:] 不计分
      \item[期末测试:] 交与教务处
    \end{description}
    \column{0.20\textwidth}
  \end{columns}
\end{frame}
%%%%%%%%%%%%%%%%%%%%

%%%%%%%%%%%%%%%%%%%%
\begin{frame}{}
  \begin{center}
    每周三晚上发布作业 \qquad 下周三 \blue{$23:55$} 前\teal{提交}作业
  \end{center}

  \begin{columns}
    \column{0.50\textwidth}
    \fig{width = 0.80\textwidth}{figs/square-logo}
    \column{0.50\textwidth}
    \fig{width = 0.60\textwidth}{figs/2022-Compilers-square-qrcode}
    \begin{center}
      邀请码: 5JP57Q5H
    \end{center}
  \end{columns}

  \vspace{0.30cm}
  \begin{center}
    发布调查问卷  \qquad\qquad 发布阅读材料
  \end{center}
\end{frame}
%%%%%%%%%%%%%%%%%%%%

%%%%%%%%%%%%%%%%%%%%
\begin{frame}{}
  \begin{columns}
    \column{0.50\textwidth}
    \begin{center}
      QQ 群号: \blue{\bf 755783220}

      \fig{width = 0.60\textwidth}{figs/2022-Compilers-qq}

      % QQ 验证: \green{\bf 2021-编译原理}
    \end{center}
    \column{0.50\textwidth}
    \begin{center}
      {\bf \teal{助教:}} 夏宇、潘煜光、顾龙、$\dots$
    \end{center}
  \end{columns}
\end{frame}
%%%%%%%%%%%%%%%%%%%%

%%%%%%%%%%%%%%%%%%%%
\begin{frame}{}
  \begin{center}
    \url{http://47.96.123.231:8081} \\[5pt]

    \fig{width = 0.30\textwidth}{figs/docs-compilers}

    编译原理课程网站,请\red{\bf 收藏}并及时关注网站更新
  \end{center}
\end{frame}
%%%%%%%%%%%%%%%%%%%%

%%%%%%%%%%%%%%%%%%%%
\begin{frame}{}
  \[
    45 = \red{0} + 5 + 15 + 15 + 10 + \red{5}
  \]

  \vspace{0.30cm}
  \fig{width = 0.60\textwidth}{figs/labs}

  \begin{center}
    \red{\bf L0:} \red{\bf 环境配置} 今晚 18:00 发布
  \end{center}
\end{frame}
%%%%%%%%%%%%%%%%%%%%

%%%%%%%%%%%%%%%%%%%%
\begin{frame}{}
  \begin{center}
    % \fig{width = 0.50\textwidth}{figs/2022-compilers-lectures}

    \vspace{0.50cm}
    \url{https://github.com/courses-at-nju-by-hfwei/compilers-lectures/tree/master/2022}
  \end{center}
\end{frame}
%%%%%%%%%%%%%%%%%%%%

%%%%%%%%%%%%%%%%%%%%
\begin{frame}{}
  \begin{columns}
    \column{0.50\textwidth}
    \fig{width = 0.60\textwidth}{figs/dragon-book}
    \begin{center}
      也可使用\blue{\bf ``本科教学版''}
    \end{center}
    \column{0.50\textwidth}
    \begin{center}
      (本书仅供参考)
      \fig{width = 0.80\textwidth}{figs/lab-book}
      \url{https://cs.nju.edu.cn/changxu/2_compiler/index.html} \\[3pt]
      (本学期对课程实验做了大幅改动)
    \end{center}
  \end{columns}
\end{frame}
%%%%%%%%%%%%%%%%%%%%

%%%%%%%%%%%%%%%%%%%%
\begin{frame}{}
  \begin{columns}
    \column{0.50\textwidth}
    \fig{width = 0.60\textwidth}{figs/flex}
    \begin{center}
      \href{https://en.wikipedia.org/wiki/Flex_(lexical_analyser_generator)}{Flex: 词法分析器生成器}
    \end{center}
    \column{0.50\textwidth}
    \fig{width = 0.60\textwidth}{figs/bison}
    \begin{center}
      \href{https://en.wikipedia.org/wiki/GNU_Bison}{Bison: 语法分析器生成器}
    \end{center}
  \end{columns}

  \vspace{0.50cm}
  \begin{center}
    不够现代, 本学期课程实验不再支持这两种工具
  \end{center}
\end{frame}
%%%%%%%%%%%%%%%%%%%%

%%%%%%%%%%%%%%%%%%%%
\begin{frame}{}
  \begin{columns}
    \column{0.50\textwidth}
    \fig{width = 0.80\textwidth}{figs/antlr-logo}
    \column{0.50\textwidth}
    \fig{width = 0.60\textwidth}{figs/parr.jpeg}
    \begin{center}
      Terence Parr
    \end{center}
  \end{columns}

  \vspace{0.80cm}
  \begin{center}
    \url{https://www.antlr.org/index.html} \\[5pt]
    \url{https://www.antlr.org/tools.html} (\red{IntelliJ Plugin})
  \end{center}
\end{frame}
%%%%%%%%%%%%%%%%%%%%

%%%%%%%%%%%%%%%%%%%%
\begin{frame}{}
  \begin{center}
    \url{http://lab.antlr.org/}
  \end{center}

  \pause
  \vspace{0.80cm}

  \begin{columns}
    \column{0.15\textwidth}
    \column{0.70\textwidth}
    \begin{description}[<+->]
      \setlength{\itemsep}{10pt}
      \item[arithmetic:] \texttt{number1.txt, simple2.txt}
      \item[fol:] \texttt{example1.txt}
      \item[guitartab:] \texttt{example1.txt}
      \item[xyz:] \texttt{example1.txt}
      \item[C:] \texttt{add.c}
    \end{description}
    \column{0.15\textwidth}
  \end{columns}
\end{frame}
%%%%%%%%%%%%%%%%%%%%

%%%%%%%%%%%%%%%%%%%%
\begin{frame}{}
  \begin{columns}
    \column{0.50\textwidth}
    \fig{width = 0.70\textwidth}{figs/antlr4-book-en}
    \column{0.50\textwidth}
    \fig{width = 0.60\textwidth}{figs/antlr4-book-ch}
  \end{columns}

  \vspace{0.50cm}
  \begin{center}
    基于 ANTLR 4,是课程实验指导的重要参考资料
  \end{center}
\end{frame}
%%%%%%%%%%%%%%%%%%%%

%%%%%%%%%%%%%%%%%%%%
\begin{frame}{}
  \begin{columns}
    \column{0.50\textwidth}
    \fig{width = 0.70\textwidth}{figs/patterns-book-en}
    \column{0.50\textwidth}
    \fig{width = 0.80\textwidth}{figs/patterns-book-ch}
  \end{columns}

  \vspace{0.50cm}
  \begin{center}
    基于 ANTLR 3,与 ANTLR 4 相比,有些过时,\\[3pt]
    但可以看作理解 ANTLR 4 的基础
  \end{center}
\end{frame}
%%%%%%%%%%%%%%%%%%%%