% overview.tex

%%%%%%%%%%%%%%%%%%%%
\begin{frame}{}
  \begin{center}
    只考虑\red{\bf 无二义性}的文法 \\[4pt]
    这意味着, 每个句子对应唯一的一棵语法分析树

    \fig{width = 0.60\textwidth}{figs/cfg-hierarchy}

    今日份主题: \red{\bf $LR$ 语法分析器}
  \end{center}
\end{frame}
%%%%%%%%%%%%%%%%%%%%

%%%%%%%%%%%%%%%%%%%%
\begin{frame}{}
  \fig{width = 0.60\textwidth}{figs/danazhina}
  \begin{center}
    \blue{\bf ``打哪指哪''}的思维方式
  \end{center}
\end{frame}
%%%%%%%%%%%%%%%%%%%%

%%%%%%%%%%%%%%%%%%%%
\begin{frame}{}
  \begin{center}
    自顶向下的、\\[15pt]
    递归下降的、\\[15pt]
    预测分析的、\\[15pt]
    适用于\red{\bf $LL(1)$文法}的、\\[15pt]
    $LL(1)$语法分析器
  \end{center}
\end{frame}
%%%%%%%%%%%%%%%%%%%%

%%%%%%%%%%%%%%%%%%%%
\begin{frame}{}
  \begin{center}
    $LL(k)$ 的\red{\bf 弱点}: \\[10pt]
    在仅看到\violet{右部的前 $k$ 个词法单元}时就必须预测要使用哪条产生式

    \pause
    \vspace{1.50cm}
    $LR(k)$ 的\blue{\bf 优点}: \\[10pt]
    看到\purple{与正在考虑的这个产生式}的\violet{整个右部}对应的词法单元之后再决定
  \end{center}
\end{frame}
%%%%%%%%%%%%%%%%%%%%