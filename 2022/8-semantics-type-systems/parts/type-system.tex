% type-system.tex

%%%%%%%%%%%%%%%%%%%%
\begin{frame}{}
  \begin{center}
    \fig{width = 0.80\textwidth}{figs/type-systems}
  \end{center}
\end{frame}
%%%%%%%%%%%%%%%%%%%%

%%%%%%%%%%%%%%%%%%%%
\begin{frame}{}
  \fig{width = 0.48\textwidth}{figs/js-eq}
\end{frame}
%%%%%%%%%%%%%%%%%%%%

%%%%%%%%%%%%%%%%%%%%
\begin{frame}{}
  \begin{center}
    \fig{width = 0.75\textwidth}{figs/js-table}

    \vspace{-0.20cm}
    \href{https://thomas-yang.me/projects/oh-my-dear-js/}{\textsc{Oh My Dear JavaScript}}
  \end{center}
\end{frame}
%%%%%%%%%%%%%%%%%%%%

%%%%%%%%%%%%%%%%%%%%
% \begin{frame}{}
%   \begin{center}
%     \fig{width = 0.80\textwidth}{figs/dynamic-static-weak-strong}
%
%     \vspace{0.30cm}
%     \teal{\url{https://youtu.be/C5fr0LZLMAs}}
%   \end{center}
% \end{frame}
%%%%%%%%%%%%%%%%%%%%

%%%%%%%%%%%%%%%%%%%%
\begin{frame}{}
  \begin{center}
    \blue{\bf 类型检查}

    \[
      x = a + b
    \]

    \fig{width = 0.80\textwidth}{figs/type-checking-rule}
  \end{center}
\end{frame}
%%%%%%%%%%%%%%%%%%%%

%%%%%%%%%%%%%%%%%%%%
\begin{frame}{}
  \begin{center}
    \blue{\bf 类型转换}

    \[
      x = 2 \times 3.14
    \]

    \vspace{0.50cm}
    \fig{width = 1.00\textwidth}{figs/type-int-float}

    \pause
    \red{\bf 不要写这样的代码!!!}
  \end{center}
\end{frame}
%%%%%%%%%%%%%%%%%%%%

%%%%%%%%%%%%%%%%%%%%
\begin{frame}{}
  \begin{center}
    \fig{width = 0.80\textwidth}{figs/widen-narrow}

    \vspace{0.60cm}
    \href{https://en.cppreference.com/w/c/language/conversion}{Conversion@cppreference.com}
  \end{center}
\end{frame}
%%%%%%%%%%%%%%%%%%%%

%%%%%%%%%%%%%%%%%%%%
\begin{frame}{}
  \begin{center}
    \blue{\bf 类型综合:} 根据子表达式的类型确定表达式的类型

    \vspace{0.60cm}
    \fig{width = 0.80\textwidth}{figs/type-synthesis}

    \[
        \teal{E_{1} + E_{2}}
    \]
  \end{center}
\end{frame}
%%%%%%%%%%%%%%%%%%%%

%%%%%%%%%%%%%%%%%%%%
\begin{frame}{}
  \begin{center}
    \blue{\bf 类型推导:} 根据某语言结构的使用方式确定表达式的类型

    \vspace{0.50cm}
    \fig{width = 1.00\textwidth}{figs/type-inference}

    \[
        \teal{null(x): \text{$x$ 是一个列表, 它的元素类型未知}}
    \]
  \end{center}
\end{frame}
%%%%%%%%%%%%%%%%%%%%

%%%%%%%%%%%%%%%%%%%%
% \begin{frame}{}
%   \begin{center}
%     \fig{width = 0.80\textwidth}{figs/expr-types}
%     \fig{width = 0.70\textwidth}{figs/widen}
%   \end{center}
% \end{frame}
%%%%%%%%%%%%%%%%%%%%

%%%%%%%%%%%%%%%%%%%%
\begin{frame}{}
  \begin{center}
    \red{\bf 数组类型}文法举例

    \fig{width = 0.30\textwidth}{figs/SDD-type-array}
    \[
      \teal{\intkw[2][3]}
    \]

    \red{\bf 类型表达式} \blue{$array(2, array(3, integer))$}
  \end{center}
\end{frame}
%%%%%%%%%%%%%%%%%%%%

%%%%%%%%%%%%%%%%%%%%
\begin{frame}{}
  \fig{width = 1.00\textwidth}{figs/parsetree-a[2][3]}
\end{frame}
%%%%%%%%%%%%%%%%%%%%

%%%%%%%%%%%%%%%%%%%%
\begin{frame}{}
  \fig{width = 0.70\textwidth}{figs/TypeCheckingListener-Fields}
\end{frame}
%%%%%%%%%%%%%%%%%%%%

%%%%%%%%%%%%%%%%%%%%
\begin{frame}{}
  \begin{center}
    继承属性 \red{$C.b$} 将一个基本类型沿着树向下传播

    \fig{width = 0.80\textwidth}{figs/anno-type-array}
    \[
        \teal{\intkw[2][3]}
    \]

    % \begin{columns}
    %   \column{0.40\textwidth}
    %     \fig{width = 1.00\textwidth}{figs/SDD-type-array}
    %   \column{0.60\textwidth}
    % \end{columns}

    \vspace{0.30cm}
    综合属性 \blue{$C.t$} 收集最终得到的类型表达式
  \end{center}
\end{frame}
%%%%%%%%%%%%%%%%%%%%

%%%%%%%%%%%%%%%%%%%%
\begin{frame}{}
  \begin{center}
    \red{\bf 类型声明}文法举例

    \fig{width = 0.40\textwidth}{figs/SDD-type-decl}
    \[
      \teal{\floatkw\; \id_{1}, \id_{2}, \id_{3}}
    \]
  \end{center}
\end{frame}
%%%%%%%%%%%%%%%%%%%%

%%%%%%%%%%%%%%%%%%%%
\begin{frame}{}
  \begin{center}
    \red{$L.inh$} 将声明的类型沿着标识符列表向下传递, 并被加入到符号表中

    \fig{width = 0.70\textwidth}{figs/anno-type-decl}
    \[
        \teal{\floatkw\; \id_{1}, \id_{2}, \id_{3}}
    \]
  \end{center}
\end{frame}
%%%%%%%%%%%%%%%%%%%%

%%%%%%%%%%%%%%%%%%%%
\begin{frame}{}
  \begin{center}
    \fig{width = 0.90\textwidth}{figs/type-systems-good-bad-ugly}

    \vspace{0.30cm}
    \teal{\url{https://youtu.be/SWTWkYbcWU0}}
  \end{center}
\end{frame}
%%%%%%%%%%%%%%%%%%%%