% backpatch.tex

%%%%%%%%%%%%%%%%%%%%
\begin{frame}{}
  \begin{center}
    \fig{width = 1.00\textwidth}{figs/if}

    \vspace{0.60cm}
    \red{\bf $B$ 还不知道 $S.next$ 的\blue{\bf 指令地址}, 如何跳转?}

    \pause
    \vspace{0.80cm}
    再扫描一遍中间代码, 将标号替换成指令(相对)地址

    \pause
    \vspace{1.20cm}
    \red{\bf 可否在生成中间代码的时候就填入指令地址?}
  \end{center}
\end{frame}
%%%%%%%%%%%%%%%%%%%%

%%%%%%%%%%%%%%%%%%%%
\begin{frame}{}
  \begin{center}
    \red{\bf 回填 (Backpatching) 技术}

    \vspace{0.30cm}
    \fig{width = 0.60\textwidth}{figs/wakeng}
    \blue{\bf 子节点挖坑、祖先节点填坑}

    \pause
    \vspace{0.50cm}
    子节点暂时不指定跳转指令的目标 \\[5pt]
    待祖先节点能够确定正确的目标地址时回头填充

    \pause
    \vspace{0.50cm}
    父节点通过\red{\bf 综合属性}收集子节点中具有相同目标的跳转指令
  \end{center}
\end{frame}
%%%%%%%%%%%%%%%%%%%%

%%%%%%%%%%%%%%%%%%%%
\begin{frame}{}
  \begin{center}
    在自底向上的分析过程中 \\[15pt]
    为左部非终结符 $B$ 计算 $B.\text{truelist}$ 与 $B.\text{falselist}$ \\[10pt]
    为左部非终结符 $S$ 计算 $S.\text{nextlist}$ \\[10pt]
    并为已能确定目标地址的跳转指令进行回填
  \end{center}
\end{frame}
%%%%%%%%%%%%%%%%%%%%

%%%%%%%%%%%%%%%%%%%%
% \begin{frame}{}
%   \begin{center}
%     \begin{columns}
%       \column{0.50\textwidth}
%         \fig{width = 1.00\textwidth}{figs/xuzhou-dong}
%       \column{0.50\textwidth}
%         \fig{width = 0.80\textwidth}{figs/xuzhou-guanyin}
%     \end{columns}
%     \fig{width = 0.40\textwidth}{figs/daba}
%   \end{center}
% \end{frame}
%%%%%%%%%%%%%%%%%%%%

%%%%%%%%%%%%%%%%%%%%
\begin{frame}{}
  \begin{center}
    \red{\bf 针对布尔表达式的回填技术}

    \fig{width = 0.80\textwidth}{figs/cfg-boolexpr-M}
  \end{center}
\end{frame}
%%%%%%%%%%%%%%%%%%%%

%%%%%%%%%%%%%%%%%%%%
\begin{frame}{}
  \begin{center}
    \red{\bf 综合属性 $B.truelist$ 保存 \teal{\bf 需要跳转到 $B.true$} 的\blue{\bf 指令地址}}
    \fig{width = 0.90\textwidth}{figs/B-true-false-backpatch}
    \red{\bf 综合属性 $B.falselist$ 保存 \teal{\bf 需要跳转到 $B.false$} 的\blue{\bf 指令地址}}

    \pause
    \vspace{0.60cm}
    \fig{width = 0.70\textwidth}{figs/B-true-false}
  \end{center}
\end{frame}
%%%%%%%%%%%%%%%%%%%%

%%%%%%%%%%%%%%%%%%%%
\begin{frame}{}
  \begin{center}
    \fig{width = 1.00\textwidth}{figs/B-rel-backpatch}

    \vspace{0.80cm}
    \fig{width = 0.90\textwidth}{figs/B-rel}
  \end{center}
\end{frame}
%%%%%%%%%%%%%%%%%%%%

%%%%%%%%%%%%%%%%%%%%
\begin{frame}{}
  \begin{center}
    \fig{width = 1.00\textwidth}{figs/B-lnot-backpatch}

    \vspace{0.80cm}
    \fig{width = 0.90\textwidth}{figs/B-lnot}
  \end{center}
\end{frame}
%%%%%%%%%%%%%%%%%%%%

%%%%%%%%%%%%%%%%%%%%
\begin{frame}{}
  \begin{center}
    \fig{width = 1.00\textwidth}{figs/B-land-backpatch}

    \vspace{0.80cm}
    \fig{width = 0.80\textwidth}{figs/M-backpatch}

    \vspace{0.80cm}
    \fig{width = 0.90\textwidth}{figs/B-land}
  \end{center}
\end{frame}
%%%%%%%%%%%%%%%%%%%%

%%%%%%%%%%%%%%%%%%%%
\begin{frame}{}
  \begin{center}
    \fig{width = 1.00\textwidth}{figs/B-lor-backpatch}

    \vspace{0.80cm}
    \fig{width = 0.80\textwidth}{figs/M-backpatch}

    \vspace{0.80cm}
    \fig{width = 0.90\textwidth}{figs/B-lor}
  \end{center}
\end{frame}
%%%%%%%%%%%%%%%%%%%%

%%%%%%%%%%%%%%%%%%%%
\begin{frame}{}
  \begin{center}
    \teal{\Large \texttt{x < 100 || x > 200 \&\& x != y}}

    \fig{width = 1.00\textwidth}{figs/anno-boolexpr-backpatch}
  \end{center}
\end{frame}
%%%%%%%%%%%%%%%%%%%%

%%%%%%%%%%%%%%%%%%%%
\begin{frame}{}
  \begin{center}
    \begin{columns}
      \column{0.50\textwidth}
        \fig{width = 1.00\textwidth}{figs/backpatch-104}
      \column{0.50\textwidth}
        \fig{width = 1.00\textwidth}{figs/backpatch-102}
    \end{columns}
  \end{center}
\end{frame}
%%%%%%%%%%%%%%%%%%%%

%%%%%%%%%%%%%%%%%%%%
\begin{frame}{}
  \fig{width = 1.00\textwidth}{figs/cfg-S-L}
\end{frame}
%%%%%%%%%%%%%%%%%%%%

%%%%%%%%%%%%%%%%%%%%
\begin{frame}{}
  \fig{width = 0.65\textwidth}{figs/cf-backpatch}
\end{frame}
%%%%%%%%%%%%%%%%%%%%

%%%%%%%%%%%%%%%%%%%%
\begin{frame}{}
  \begin{center}
    \fig{width = 1.00\textwidth}{figs/if-backpatch}

    \vspace{0.80cm}
    \fig{width = 0.80\textwidth}{figs/M-backpatch-2}

    \pause
    \vspace{0.80cm}
    \fig{width = 0.90\textwidth}{figs/SDD-if}
  \end{center}
\end{frame}
%%%%%%%%%%%%%%%%%%%%

%%%%%%%%%%%%%%%%%%%%
\begin{frame}{}
  \begin{center}
    \fig{width = 1.00\textwidth}{figs/if-else-backpatch}

    \vspace{0.50cm}
    \fig{width = 0.80\textwidth}{figs/M-N-backpatch}

    \pause
    \vspace{0.50cm}
    \fig{width = 0.70\textwidth}{figs/SDD-if-else}
  \end{center}
\end{frame}
%%%%%%%%%%%%%%%%%%%%

%%%%%%%%%%%%%%%%%%%%
\begin{frame}{}
  \begin{center}
    \fig{width = 1.00\textwidth}{figs/while-backpatch}

    \vspace{0.50cm}
    \fig{width = 0.90\textwidth}{figs/M-backpatch-2}

    \pause
    \vspace{0.50cm}
    \fig{width = 0.80\textwidth}{figs/SDD-while}
  \end{center}
\end{frame}
%%%%%%%%%%%%%%%%%%%%

%%%%%%%%%%%%%%%%%%%%
\begin{frame}{}
  \begin{center}
    \fig{width = 1.00\textwidth}{figs/S-L-backpatch}
  \end{center}
\end{frame}
%%%%%%%%%%%%%%%%%%%%

%%%%%%%%%%%%%%%%%%%%
\begin{frame}{}
  \fig{width = 0.60\textwidth}{figs/cf-backpatch-gen}

  \begin{center}
    只有 (3) 与 (7) 生成了新的代码, 控制流语句的主要目的是\red{``控制''流}。
  \end{center}
\end{frame}
%%%%%%%%%%%%%%%%%%%%