% antlr3.tex

%%%%%%%%%%%%%%%%%%%%
\begin{frame}{}
  \begin{center}
    \Large{\blue{\bf ANTLR3} 语法分析器是如何工作的?}
  \end{center}
\end{frame}
%%%%%%%%%%%%%%%%%%%%

%%%%%%%%%%%%%%%%%%%%
\begin{frame}{}
  \begin{center}
    \blue{带记忆功能的}\purple{可回溯的}\red{递归下降的}语法分析器

    \vspace{0.50cm}
    \fig{width = 0.50\textwidth}{figs/antlr-logo}
    \gray{(适用于 ANTLR3 与部分 ANTLR4)}
    \vspace{1.00cm}

    甚至可以使用\violet{谓词解析器}处理\green{上下文相关文法}
  \end{center}
\end{frame}
%%%%%%%%%%%%%%%%%%%%

%%%%%%%%%%%%%%%%%%%%
\begin{frame}{}
  \begin{center}
	\red{$LL(1)$} \\[20pt]
	\texttt{tpdsl: rd/NameList.g4}  \\[40pt]
	\pause
	\texttt{tpdsl: rd/ListParser.java}
	\\ (\texttt{\blue{elements()}}) \\[30pt]
	\pause
	\texttt{tpdsl: rd/NameListParser.java}
  \end{center}
\end{frame}
%%%%%%%%%%%%%%%%%%%%

%%%%%%%%%%%%%%%%%%%%
\begin{frame}{}
  \begin{center}
	\red{$LL(k = 2)$} \\[20pt]
	\texttt{tpdsl: multi/NameListWithAssign.g4}  \\[40pt]
	\pause
	\texttt{tpdsl: multi/LAParser.java}
	\\(\texttt{\blue{element()}}) \\[30pt]
	\pause
	\texttt{tpdsl: multi/NameListWithAssignParser.java}
	\\(\texttt{\purple{.adaptivePredict()}})
  \end{center}
\end{frame}
%%%%%%%%%%%%%%%%%%%%

%%%%%%%%%%%%%%%%%%%%
\begin{frame}{}
  \begin{center}
	\red{Backtrack (回溯)} \\[20pt]
	\texttt{tpdsl: backtrack/NameListWithParallelAssign.g4}  \\[40pt]
	\pause
	\texttt{tpdsl: backtrack/BacktrackParser.java}
	\\(\texttt{\blue{stat()}}) \\[30pt]
	\pause
	\texttt{tpdsl: backtrack/NameListWithParallelAssignParser.java}
	\\(\texttt{\purple{.adaptivePredict()}})

	\pause
	\vspace{20pt}
	\violet{\bf ANTLR4 不需要回溯, 这是 ANTLR4 的一大创新之处}
  \end{center}
\end{frame}
%%%%%%%%%%%%%%%%%%%%

%%%%%%%%%%%%%%%%%%%%
\begin{frame}{}
  \fig{width = 0.50\textwidth}{figs/antlr95}
  \fig{width = 0.60\textwidth}{figs/antlr11}
  \fig{width = 0.60\textwidth}{figs/antlr14}

  \pause
  \begin{center}
	\href{https://github.com/courses-at-nju-by-hfwei/compilers-papers-we-love/tree/master/parsing}
	{courses-at-nju-by-hfwei/compilers-papers-we-love}
  \end{center}
\end{frame}
%%%%%%%%%%%%%%%%%%%%