% grammar-design.tex

%%%%%%%%%%%%%%%%%%%%
\begin{frame}{}
  \fig{width = 0.50\textwidth}{figs/antlr-logo}

  \begin{center}
    \texttt{CymbolCFG.g4}
  \end{center}
\end{frame}
%%%%%%%%%%%%%%%%%%%%

%%%%%%%%%%%%%%%%%%%%
\begin{frame}{}
  \begin{center}
    \red{[Extended] Backus–Naur form ([E]BNF)}
  \end{center}

  \begin{columns}
    \column{0.33\textwidth}
      \fig{width = 0.70\textwidth}{figs/Backus}
      \begin{center}
        John Backus ($1924 \sim 2007$)
      \end{center}
    \column{0.34\textwidth}
      \fig{width = 0.75\textwidth}{figs/Naur}
      \begin{center}
        Peter Naur ($1928 \sim 2016$)
      \end{center}
    \column{0.34\textwidth}
      \pause
      \fig{width = 0.85\textwidth}{figs/Wirth}
      \begin{center}
        Niklaus Wirth ($1934 \sim$)
      \end{center}
  \end{columns}

  \pause
  \begin{center}
    \teal{1977 (FORTRAN) \qquad\qquad 2005 (ALGOL60) \qquad\qquad 1984 (PLs; PASCAL)}
  \end{center}
\end{frame}
%%%%%%%%%%%%%%%%%%%%

%%%%%%%%%%%%%%%%%%%%
\begin{frame}{}
  \begin{center}
    Extended Backus–Naur form (EBNF)
  \end{center}

  \vspace{0.50cm}
  \begin{columns}
    \column{0.50\textwidth}
      \fig{width = 0.80\textwidth}{figs/choice}
    \column{0.50\textwidth}
      \fig{width = 0.90\textwidth}{figs/choice-text}
  \end{columns}
  \begin{center}
    \blue{Choice}
  \end{center}
\end{frame}
%%%%%%%%%%%%%%%%%%%%

%%%%%%%%%%%%%%%%%%%%
\begin{frame}{}
  \begin{center}
    Extended Backus–Naur form (EBNF)
  \end{center}

  \vspace{0.50cm}
  \fig{width = 0.90\textwidth}{figs/optional}
  \begin{center}
    \blue{Optional}
  \end{center}
\end{frame}
%%%%%%%%%%%%%%%%%%%%

%%%%%%%%%%%%%%%%%%%%
\begin{frame}{}
  \begin{center}
    Extended Backus–Naur form (EBNF)
  \end{center}

  \vspace{0.50cm}
  \fig{width = 0.90\textwidth}{figs/zero-or-more}
  \begin{center}
    \blue{Zero or More}
  \end{center}
\end{frame}
%%%%%%%%%%%%%%%%%%%%

%%%%%%%%%%%%%%%%%%%%
\begin{frame}{}
  \begin{center}
    Extended Backus–Naur form (EBNF)
  \end{center}

  \vspace{0.50cm}
  \fig{width = 0.90\textwidth}{figs/one-or-more}
  \begin{center}
    \blue{One or More}
  \end{center}
\end{frame}
%%%%%%%%%%%%%%%%%%%%

%%%%%%%%%%%%%%%%%%%%
\begin{frame}{}
  \fig{width = 0.50\textwidth}{figs/antlr-logo}

  \begin{center}
    \texttt{Cymbol.g4}
  \end{center}
\end{frame}
%%%%%%%%%%%%%%%%%%%%

%%%%%%%%%%%%%%%%%%%%
\begin{frame}{}
  \begin{center}
    \red{\bf 语法分析树}

    \vspace{0.30cm}
    语法分析树是静态的, 它不关心动态的推导顺序

    \fig{width = 0.35\textwidth}{figs/tree-(id+id)}

    一棵语法分析树对应多个推导
  \end{center}
\end{frame}
%%%%%%%%%%%%%%%%%%%%

%%%%%%%%%%%%%%%%%%%%
\begin{frame}{}
  % cfg-expr-add-sub-mul-div.tex

\begin{empheq}[box=\widefbox]{align*}
  E \to E + E \mid E - E \mid E \ast E \mid E / E \mid (E) \mid \id \mid \num
\end{empheq}

  \[
    1 - 2 - 3 \text{ 的语法树?}
  \]

  \pause
  \begin{columns}
    \column{0.50\textwidth}
      \fig{width = 0.60\textwidth}{figs/1-2-3-parse-tree-left}
    \column{0.50\textwidth}
      \pause
      \fig{width = 0.60\textwidth}{figs/1-2-3-parse-tree-right}
  \end{columns}

  \pause
  \vspace{0.50cm}
  \begin{center}
    \blue{``运算符结合性''}导致的\red{二义性}
  \end{center}
\end{frame}
%%%%%%%%%%%%%%%%%%%%

%%%%%%%%%%%%%%%%%%%%
\begin{frame}{}
  \begin{definition}[\red{\bf 二义性}(Ambiguous)文法]
    如果$L(G)$中的\purple{\bf 某个}句子有\blue{\bf 一个以上}\teal{语法树}, \\[5pt]
    则文法$G$是二义性的。
  \end{definition}
\end{frame}
%%%%%%%%%%%%%%%%%%%%

%%%%%%%%%%%%%%%%%%%%
\begin{frame}{}
  % cfg-expr-add-sub-mul-div.tex

\begin{empheq}[box=\widefbox]{align*}
  E \to E + E \mid E - E \mid E \ast E \mid E / E \mid (E) \mid \id \mid \num
\end{empheq}

  \[
    1 + 2 \ast 3 \text{ 的语法树?}
  \]

  \pause
  \begin{columns}
    \column{0.50\textwidth}
      \fig{width = 0.60\textwidth}{figs/1+2-mul3-parse-tree-add}
    \column{0.50\textwidth}
      \pause
      \fig{width = 0.60\textwidth}{figs/1+2-mul3-parse-tree-mul}
  \end{columns}

  \pause
  \vspace{0.50cm}
  \begin{center}
    \blue{``运算符优先级''}导致的\red{二义性}
  \end{center}
\end{frame}
%%%%%%%%%%%%%%%%%%%%

%%%%%%%%%%%%%%%%%%%%
\begin{frame}{}
  \fig{width = 0.50\textwidth}{figs/cfg-if-then-else}
  \begin{center}
    ``悬空-else'' 文法
  \end{center}

  \[
    \ifkw\; E_{1} \;\thenkw\; \ifkw\; E_{2} \;\thenkw\; S_{1} \;\elsekw\; S_{2}
  \]

  \pause
  \begin{columns}
    \column{0.50\textwidth}
      \fig{width = 1.00\textwidth}{figs/tree-if-then-else-1}
      \[
        \ifkw\; E_{1} \;\thenkw\; \red{\big(}\ifkw\; E_{2} \;\thenkw\; S_{1} \;\elsekw\; S_{2}\red{\big)}
      \]
    \column{0.50\textwidth}
      \fig{width = 1.00\textwidth}{figs/tree-if-then-else-2}
      \[
        \ifkw\; E_{1} \;\thenkw\; \red{\big(}\ifkw\; E_{2} \;\thenkw\; S_{1}\red{\big)} \;\elsekw\; S_{2}
      \]
  \end{columns}
\end{frame}
%%%%%%%%%%%%%%%%%%%%

%%%%%%%%%%%%%%%%%%%%
\begin{frame}{}
  \begin{center}
    \red{\bf 二义性文法}

    \vspace{0.60cm}
    不同的语法分析树产生不同的语义

    \fig{width = 0.40\textwidth}{figs/girl-woman}
  \end{center}
\end{frame}
%%%%%%%%%%%%%%%%%%%%

%%%%%%%%%%%%%%%%%%%%
\begin{frame}{}
  \fig{width = 0.60\textwidth}{figs/cfg-hierarchy}

  \begin{center}
    所有语法分析器都要求文法是\red{\bf 无二义性}的
  \end{center}
\end{frame}
%%%%%%%%%%%%%%%%%%%%

%%%%%%%%%%%%%%%%%%%%
\begin{frame}{}
  \begin{center}
    \red{\bf 二义性文法}
  \end{center}

  \begin{columns}
    \column{0.50\textwidth}
      \begin{center}
        $Q:$ 如何\blue{\bf 识别}二义性文法?

        \uncover<2->{
          \fig{width = 0.60\textwidth}{figs/impossible}
          这是\red{\bf 不可判定}的问题
        }
      \end{center}
    \column{0.50\textwidth}
      \begin{center}
        $Q:$ 如何\blue{\bf 消除}文法的二义性?

        \uncover<3->{
          \fig{width = 0.90\textwidth}{figs/learn-by-examples}
        }
      \end{center}
  \end{columns}
\end{frame}
%%%%%%%%%%%%%%%%%%%%

%%%%%%%%%%%%%%%%%%%%
\begin{frame}{}
  % cfg-expr-add-sub-mul-div.tex

\begin{empheq}[box=\widefbox]{align*}
  E \to E + E \mid E - E \mid E \ast E \mid E / E \mid (E) \mid \id \mid \num
\end{empheq}

  \vspace{0.30cm}
  \begin{center}
    四则运算均是\red{\bf 左结合}的

    \vspace{0.30cm}
    \red{\bf 优先级:} 括号最先, 先乘除后加减

    \vspace{0.80cm}
    二义性表达式文法以\blue{\bf 相同的方式}处理所有的算术运算符

    \vspace{0.30cm}
    要消除二义性, 需要\blue{\bf 区别对待}不同的运算符

    \pause
    \vspace{0.80cm}
    \purple{将运算的``先后''顺序信息编码到语法树的``层次''结构中}
  \end{center}
\end{frame}
%%%%%%%%%%%%%%%%%%%%

%%%%%%%%%%%%%%%%%%%%
\begin{frame}{}
  % cfg-expr-add.tex

\begin{empheq}[box=\widefbox]{align*}
  E \to E + E \mid \id
\end{empheq}

  \vspace{0.80cm}
  \begin{columns}
    \column{0.50\textwidth}
      \pause
      % cfg-expr-add-left-assoc.tex

\begin{empheq}[box=\widefbox]{align*}
  E &\to E + \red{T} \\[8pt]
  \red{T} &\to \id
\end{empheq}
      \begin{center}
        \blue{\bf 左结合}文法
      \end{center}
    \column{0.50\textwidth}
      \pause
      % cfg-expr-add-right-assoc.tex

\begin{empheq}[box=\widefbox]{align*}
  E &\to \red{T} + E \\[8pt]
  \red{T} &\to \id
\end{empheq}
      \begin{center}
        \blue{\bf 右结合}文法
      \end{center}
  \end{columns}

  \pause
  \vspace{0.80cm}
  \begin{center}
    使用\red{\bf 左(右)递归}实现\blue{\bf 左(右)结合}
  \end{center}
\end{frame}
%%%%%%%%%%%%%%%%%%%%

%%%%%%%%%%%%%%%%%%%%
\begin{frame}{}
  % cfg-expr-add-mul.tex

\begin{empheq}[box=\widefbox]{align*}
  E \to E + E \mid E \ast E \mid (E) \mid \id
\end{empheq}

  \pause
  \vspace{0.80cm}
  % cfg-expr-add-mul-mul-first.tex

\begin{empheq}[box=\widefbox]{align*}
  E &\to E + \red{T} \mid \red{T} \\[8pt]
  T &\to \red{T \ast F} \mid F \\[8pt]
  F &\to \purple{(E)} \mid \id
\end{empheq}

  \begin{center}
    \blue{\bf 括号最先, 先乘后加}文法
  \end{center}
\end{frame}
%%%%%%%%%%%%%%%%%%%%

%%%%%%%%%%%%%%%%%%%%
\begin{frame}{}
  % cfg-expr-add-sub-mul-div.tex

\begin{empheq}[box=\widefbox]{align*}
  E \to E + E \mid E - E \mid E \ast E \mid E / E \mid (E) \mid \id \mid \num
\end{empheq}

  \vspace{0.50cm}
  % cfg-expr-add-sub-mul-div-ETF.tex

\begin{empheq}[box=\widefbox]{align*}
  E &\to E + T \mid E - T \mid T \\[8pt]
  T &\to T \ast F \mid T / F \mid F \\[8pt]
  F &\to (E) \mid \id \mid \num
\end{empheq}

  \begin{center}
    \blue{\bf 无二义性}的表达式文法
    \[
      \gray{
      E: \text{表达式} (expression); \quad
      T: \text{项} (term) \quad
      F: \text{因子} (factor)}
    \]

    \pause
    \vspace{0.20cm}
    \purple{将运算的``先后''顺序信息编码到语法树的``层次''结构中}
  \end{center}
\end{frame}
%%%%%%%%%%%%%%%%%%%%

%%%%%%%%%%%%%%%%%%%%
\begin{frame}{}
  \begin{center}
    \fig{width = 0.50\textwidth}{figs/cfg-if-then-else}

    \[
      \ifkw\; E_{1} \;\thenkw\; \ifkw\; E_{2} \;\thenkw\; S_{1} \;\elsekw\; S_{2}
    \]

    \fig{width = 0.70\textwidth}{figs/tree-if-then-else-1}

    ``每个\elsekw{}与\red{\bf 最近的尚未匹配的}\thenkw{}匹配''
  \end{center}
\end{frame}
%%%%%%%%%%%%%%%%%%%%

%%%%%%%%%%%%%%%%%%%%
\begin{frame}{}
  \begin{center}
    \fig{width = 0.50\textwidth}{figs/cfg-if-then-else}

    \fig{width = 0.95\textwidth}{figs/cfg-if-then-else-unambiguous}

    基本思想: \blue{\thenkw{} 与 \elsekw{} 之间}的语句必须是\red{\bf ``已匹配的''}
  \end{center}
\end{frame}
%%%%%%%%%%%%%%%%%%%%

%%%%%%%%%%%%%%%%%%%%
\begin{frame}{}
  \begin{center}
    \fig{width = 0.50\textwidth}{figs/you-are-a-teacher}
  \end{center}
\end{frame}
%%%%%%%%%%%%%%%%%%%%

%%%%%%%%%%%%%%%%%%%%
\begin{frame}{}
  \begin{center}
    \fig{width = 0.40\textwidth}{figs/keep-calm-prove-it-blue}
    我们要证明\red{\bf 两件}事情

    \pause
    \[
      L(G) = L(G')
    \]

    \pause
    $G'$ 是无二义性的
  \end{center}
\end{frame}
%%%%%%%%%%%%%%%%%%%%

%%%%%%%%%%%%%%%%%%%%
\begin{frame}{}
  \begin{center}
    \fig{width = 0.50\textwidth}{figs/cfg-if-then-else}

    \fig{width = 0.95\textwidth}{figs/cfg-if-then-else-unambiguous}

    \pause
    \begin{columns}
      \column{0.50\textwidth}
        \[
          \blue{L(G') \subseteq L(G)}
        \]
      \column{0.50\textwidth}
        \[
          \blue{L(G) \subseteq L(G')}
        \]
        \pause
        \vspace{-0.60cm}
        \begin{center}
          \red{\bf 对推导步数作数学归纳}
        \end{center}
    \end{columns}
  \end{center}
\end{frame}
%%%%%%%%%%%%%%%%%%%%

%%%%%%%%%%%%%%%%%%%%
\begin{frame}{}
  \begin{center}
    $G'$ 是\red{\bf 无二义性}的
    \fig{width = 0.95\textwidth}{figs/cfg-if-then-else-unambiguous}

    \pause
    \vspace{0.20cm}
    每个句子对应的\blue{\bf 语法分析树}是唯一的

    \pause
    \vspace{0.30cm}
    只需证明: 每个非终结符的\blue{\bf ``展开''方式}是唯一的

    \pause
    \[
      L(matched\_stmt) \cap L(open\_stmt) = \emptyset
    \]

    \pause
    \vspace{-0.50cm}
    \[
      L(matched\_stmt_{1}) \cap L(matched\_stmt_{2}) = \emptyset
    \]

    \pause
    \vspace{-0.50cm}
    \[
      L(open\_stmt_{1}) \cap L(open\_stmt_{2}) = \emptyset
    \]
  \end{center}
\end{frame}
%%%%%%%%%%%%%%%%%%%%

%%%%%%%%%%%%%%%%%%%%
\begin{frame}{}
  \fig{width = 0.50\textwidth}{figs/antlr-logo}

  \begin{center}
    \texttt{Cymbol.g4}
  \end{center}
\end{frame}
%%%%%%%%%%%%%%%%%%%%

%%%%%%%%%%%%%%%%%%%%
\begin{frame}{}
  \begin{center}
    \red{\bf 左递归文法 (Left Recursion)}
    % E-T-left-recursion.tex

\begin{empheq}[box=\widefbox]{align*}
  E &\to E + T \mid T
\end{empheq}

    \pause
    % E-T-right-recursion.tex

\begin{empheq}[box=\widefbox]{align*}
  E &\to T E' \\[8pt]
  E' &\to +\; T E' \mid \epsilon
\end{empheq}
    将左递归转为\blue{\bf 右递归}

    \pause
    \vspace{0.50cm}
    (注: 右递归对应右结合; 需要在后续阶段进行额外处理)
  \end{center}
\end{frame}
%%%%%%%%%%%%%%%%%%%%

%%%%%%%%%%%%%%%%%%%%
\begin{frame}{}
  \begin{center}
    % cfg-A-left-recursion.tex

\begin{empheq}[box=\widefbox]{align*}
  A \to A \alpha_{1} \mid A \alpha_{2} \mid \dots A \alpha_{m}
    \mid \beta_{1} \mid \beta_{2} \mid \dots \beta_{n}
\end{empheq}

    其中, $\beta_{i}$ 都不以 $A$ 开头

    \vspace{0.30cm}
    % cfg-A-right-recursion.tex

\begin{empheq}[box=\widefbox]{align*}
  A &\to \beta_{1}A' \mid \beta_{2}A' \mid \dots \mid \beta_{n}A' \\[8pt]
  A' &\to \alpha_{1}A' \mid \alpha_{2}A' \mid \dots \mid \alpha_{m}A' \mid \epsilon
\end{empheq}
  \end{center}
\end{frame}
%%%%%%%%%%%%%%%%%%%%

%%%%%%%%%%%%%%%%%%%%
\begin{frame}{}
  % cfg-expr-add-mul-mul-first.tex

\begin{empheq}[box=\widefbox]{align*}
  E &\to E + \red{T} \mid \red{T} \\[8pt]
  T &\to \red{T \ast F} \mid F \\[8pt]
  F &\to \purple{(E)} \mid \id
\end{empheq}

  \pause
  % cfg-expr-add-mul-no-left-recursion.tex

\begin{empheq}[box=\widefbox]{align*}
  E &\to T E' \\[8pt]
  E' &\to +\; T E' \mid \epsilon \\[8pt]
  T &\to F T' \\[8pt]
  T' &\to \ast\; F T' \mid \epsilon \\[8pt]
  F &\to (E) \mid \id
\end{empheq}
\end{frame}
%%%%%%%%%%%%%%%%%%%%

%%%%%%%%%%%%%%%%%%%%
\begin{frame}{}
  \begin{center}
    \red{\bf 间接左递归}

    \vspace{-0.30cm}
    % cfg-S-A.tex

\begin{empheq}[box=\widefbox]{align*}
  S &\to A a \mid b \\[8pt]
  A &\to A c \mid S b \mid \epsilon
\end{empheq}

    \vspace{-0.30cm}
    \[
      S \implies Aa \implies Sba
    \]

    \pause
    \fig{width = 0.90\textwidth}{figs/left-recursion-alg}

    \vspace{-1.00cm}
    \[
      \blue{\boxed{A_{k} \to A_{l} \alpha \implies l > k}}
    \]
  \end{center}
\end{frame}
%%%%%%%%%%%%%%%%%%%%

%%%%%%%%%%%%%%%%%%%%
\begin{frame}{}
  % cfg-S-A.tex

\begin{empheq}[box=\widefbox]{align*}
  S &\to A a \mid b \\[8pt]
  A &\to A c \mid S b \mid \epsilon
\end{empheq}

  \[
    A \to Ac \mid Aad \mid bd \mid \epsilon
  \]

  % cfg-S-A-no-left-recursion.tex

\begin{empheq}[box=\widefbox]{align*}
  S &\to A a \mid b \\[8pt]
  A &\to b d A' \mid A' \\[8pt]
  A' &\to c A' \mid a d A' \mid \epsilon
\end{empheq}

  \vspace{0.30cm}
  \[
      \blue{\boxed{A_{k} \to A_{l} \alpha \implies l > k}}
  \]
\end{frame}
%%%%%%%%%%%%%%%%%%%%

%%%%%%%%%%%%%%%%%%%%
\begin{frame}{}
  % cfg-S-T.tex

\begin{empheq}[box=\widefbox]{align*}
  (0)\; S &\to Sa \mid Tbc \mid Td \\[8pt]
  (1)\; T &\to Se \mid gh
\end{empheq}

  \pause
  \vspace{0.30cm}
  % cfg-S-T.tex

\begin{empheq}[box=\widefbox]{align*}
  S &\to T (bc \mid d) S' \\[8pt]
  S' &\to a S' \mid \epsilon \\[8pt]
  T &\to ghT' \\[8pt]
  T' &\to (bc \mid d) S' e T' \mid \epsilon
\end{empheq}
\end{frame}
%%%%%%%%%%%%%%%%%%%%

%%%%%%%%%%%%%%%%%%%%
\begin{frame}{}
  \begin{center}
    ANTLR4 可以处理直接左递归文法, \red{\bf 不要}改写文法

    \vspace{0.50cm}
    \texttt{Expr.g4}
  \end{center}
\end{frame}
%%%%%%%%%%%%%%%%%%%%

%%%%%%%%%%%%%%%%%%%%
\begin{frame}{}
  \begin{center}
    \fig{width = 0.60\textwidth}{figs/cfg-if-then-else-short}

    \vspace{0.50cm}
    \red{\bf 提取左公因子}
    \vspace{0.50cm}

    \fig{width = 0.40\textwidth}{figs/cfg-if-then-else-short-left-factor}
  \end{center}
\end{frame}
%%%%%%%%%%%%%%%%%%%%

%%%%%%%%%%%%%%%%%%%%
\begin{frame}{}
  \begin{center}
    ANTLR4 可以处理有左公因子的文法, \red{\bf 不要}改写文法

    \vspace{0.50cm}
    \texttt{IfStat.g4}
  \end{center}
\end{frame}
%%%%%%%%%%%%%%%%%%%%