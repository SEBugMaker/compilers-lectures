% overview.tex

%%%%%%%%%%%%%%%%%%%%
\begin{frame}{}
  \fig{width = 0.60\textwidth}{figs/knuth}

  \begin{center}
    Donald Knuth
  \end{center}
\end{frame}
%%%%%%%%%%%%%%%%%%%%

%%%%%%%%%%%%%%%%%%%%
\begin{frame}{}
  \fig{width = 0.70\textwidth}{figs/knuth-paper-1967}

  \begin{center}
    \red{\bf 属性文法 (Attribute Grammar):} 为上下文无关文法赋予\blue{\bf 语义}
  \end{center}
\end{frame}
%%%%%%%%%%%%%%%%%%%%

%%%%%%%%%%%%%%%%%%%%
\begin{frame}{}
  \begin{center}
    \red{\bf 关键问题: 如何基于上下文无关文法做上下文相关分析?}

    \vspace{0.50cm}
    \fig{width = 0.70\textwidth}{figs/info-flow}

    \blue{\bf 语法分析树}上的\red{\bf 有序}信息流动
  \end{center}
\end{frame}
%%%%%%%%%%%%%%%%%%%%

%%%%%%%%%%%%%%%%%%%%
\begin{frame}{}
  \begin{center}
    \begin{columns}[b]
      \column{0.25\textwidth}
        \fig{width = 0.80\textwidth}{figs/1}
        \vspace{-0.10cm}
        \begin{center}
          \blue{一对概念}
        \end{center}
      \column{0.25\textwidth}
        \fig{width = 0.85\textwidth}{figs/2}
        \vspace{-0.60cm}
        \begin{center}
          \blue{两类属性定义}
        \end{center}
      \column{0.25\textwidth}
        \fig{width = 0.65\textwidth}{figs/3}
        \begin{center}
          \blue{三种实现方式}
        \end{center}
      \column{0.25\textwidth}
        \fig{width = 0.65\textwidth}{figs/4}
        \begin{center}
          \blue{四大应用}
        \end{center}
    \end{columns}
  \end{center}
\end{frame}
%%%%%%%%%%%%%%%%%%%%

%%%%%%%%%%%%%%%%%%%%
\begin{frame}{}
  \begin{center}
    \teal{\bf 表达式求值}

    \fig{width = 0.30\textwidth}{figs/4}

    \vspace{0.30cm}
    \red{\bf 类型系统} (语义分析)

    \vspace{0.30cm}
    抽象语法树

    \vspace{0.30cm}
    \blue{\bf 后缀表达式} (中间代码生成)
  \end{center}
\end{frame}
%%%%%%%%%%%%%%%%%%%%