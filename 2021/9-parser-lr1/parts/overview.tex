% overview.tex

%%%%%%%%%%%%%%%%%%%%
\begin{frame}{}
  \begin{center}
    只考虑\red{\bf 无二义性}的文法 \\[4pt]
    这意味着, 每个句子对应唯一的一棵语法分析树

    \fig{width = 0.60\textwidth}{figs/cfg-hierarchy}

    今日份主题: \red{\bf $LR$ 语法分析器}
  \end{center}
\end{frame}
%%%%%%%%%%%%%%%%%%%%

%%%%%%%%%%%%%%%%%%%%
\begin{frame}{}
  \begin{center}
    自底向上的、\\[15pt]
    不断归约的、\\[15pt]
    基于句柄识别自动机的、\\[15pt]
    适用于\red{\bf $LR$ 文法}的、\\[15pt]
    $LR$ 语法分析器
  \end{center}
\end{frame}
%%%%%%%%%%%%%%%%%%%%

%%%%%%%%%%%%%%%%%%%%
\begin{frame}{}
  \begin{center}
    \red{\bf $LR(0)$ 语法分析器}

    \vspace{0.80cm}
    \begin{columns}
      \column{0.10\textwidth}
      \column{0.80\textwidth}
        \begin{description}
          \setlength{\itemsep}{15pt}
          \item[$L:$] \purple{\bf 从左向右} (Left-to-right) 扫描输入
          \item[$R:$] 构建\purple{\bf 反向} (Reverse) \purple{\bf 最右推导}
          \item[$0:$] \purple{\bf 归约}时无需向前看
        \end{description}
      \column{0.10\textwidth}
    \end{columns}
  \end{center}
\end{frame}
%%%%%%%%%%%%%%%%%%%%

%%%%%%%%%%%%%%%%%%%%
\begin{frame}
  \begin{columns}
    \column{0.50\textwidth}
      \fig{width = 0.90\textwidth}{figs/lr0-automaton-expr}
    \column{0.50\textwidth}
      % lr0-table-expr.tex

% \usepackage{multirow}
% \usepackage{graphicx}
\begin{table}[]
  \centering
  \resizebox{\textwidth}{!}{%
  \renewcommand{\arraystretch}{1.3}
  \begin{tabular}{|c|cccccc|ccc|}
    \hline
    \multirow{2}{*}{}
    & \multicolumn{6}{c|}{\blue{\action}}
    & \multicolumn{3}{c|}{\blue{\goto}}
    \\ \cline{2-10}
        & $\id$ & $+$  & $\ast$   & $($  & $)$   & $\$$  & $E$     & $T$    & $F$     \\ \hline
    0   & $s5$  &      &          & $s4$ &       &       & $g1$    & $g2$   & $g3$    \\
    1   &       & $s6$ &          &      &       & \teal{$acc$} &         &        &         \\
    \blue{\bf 2}   & $r2$  & $r2$ & \purple{$s7, r2$} & $r2$ & $r2$  & $r2$  &         &        &         \\
    3   & $r4$  & $r4$ & $r4$     & $r4$ & $r4$  & $r4$  &         &        &         \\
    \red{\bf 4}   & $s5$  &      &          & $s4$ &       &       & $g8$    & $g2$   & $g3$    \\
    5   & $r6$  & $r6$ & $r6$     & $r6$ & $r6$  & $r6$  &         &        &         \\
    6   & $s5$  &      &          & $s4$ &       &       &         & $g9$   & $g3$    \\
    7   & $s5$  &      &          & $s4$ &       &       &         &        & $g10$   \\
    8   &       & $s6$ &          &      & $s11$ &       &         &        &         \\
    9   & $r1$  & $r1$ & \purple{$s7, r1$} & $r1$ & $r1$  & $r1$  &         &        &         \\
    \blue{\bf 10}  & $r3$  & $r3$ & $r3$     & $r3$ & $r3$  & $r3$  &         &        &         \\
    11  & $r5$  & $r5$ & $r5$     & $r5$ & $r5$  & $r5$  &         &        &         \\ \hline
  \end{tabular}}
\end{table}
  \end{columns}
\end{frame}
%%%%%%%%%%%%%%%%%%%%

%%%%%%%%%%%%%%%%%%%%
\begin{frame}
  \[
    \id \qquad + \qquad \ast \qquad ( \qquad ) \qquad \$
  \]

  % cfg-expr-add-mul-mul-first-numbering.tex

\begin{empheq}[box=\widefbox]{align*}
  (1)\; E &\to E + T \\[8pt]
  (2)\; E &\to T \\[8pt]
  (3)\; T &\to T \ast F \\[8pt]
  (4)\; T &\to F \\[8pt]
  (5)\; F &\to (E) \\[8pt]
  (6)\; F &\to \id
\end{empheq}

  \pause
  \[
    \red{\follow(E) = \pause \set{+, ), \$}}
  \]
\end{frame}
%%%%%%%%%%%%%%%%%%%%