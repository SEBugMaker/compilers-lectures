% expr.tex

%%%%%%%%%%%%%%%%%%%%
\begin{frame}{}
  \begin{center}
    \red{\bf 表达式的中间代码翻译}

    \fig{width = 0.80\textwidth}{figs/SDD-expr-tac}
    \blue{\bf 综合属性 $E.code$ 与 $E.addr$}
  \end{center}
\end{frame}
%%%%%%%%%%%%%%%%%%%%

%%%%%%%%%%%%%%%%%%%%
\begin{frame}{}
  \begin{center}
    \begin{columns}
      \column{0.60\textwidth}
        \fig{width = 1.00\textwidth}{figs/SDD-expr-tac}
      \column{0.40\textwidth}
        \fig{width = 0.80\textwidth}{figs/expr-tac}
    \end{columns}
    \[
      \teal{a = b + -c}
    \]
  \end{center}
\end{frame}
%%%%%%%%%%%%%%%%%%%%

%%%%%%%%%%%%%%%%%%%%
\begin{frame}{}
  \begin{center}
    \red{\bf 表达式的中间代码翻译} \teal{({\bf 增量式})}

    \fig{width = 0.80\textwidth}{figs/SDT-expr-tac}
    \vspace{0.10cm}
    \blue{\bf 综合属性 $E.addr$} \\[5pt]
    假想一个全局指令缓冲区, 对 \emph{gen} 的连续调用将生成一个指令序列
  \end{center}
\end{frame}
%%%%%%%%%%%%%%%%%%%%