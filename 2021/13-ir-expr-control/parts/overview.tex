% overview.tex

%%%%%%%%%%%%%%%%%%%%
\begin{frame}{}
  \begin{center}
    \red{\bf Intermediate Representation (IR)}
    \fig{width = 0.95\textwidth}{figs/ir-phase}
  \end{center}
\end{frame}
%%%%%%%%%%%%%%%%%%%%

%%%%%%%%%%%%%%%%%%%%
\begin{frame}{}
  \fig{width = 1.00\textwidth}{figs/ir}
  \begin{center}
    \href{https://dl.acm.org/doi/pdf/10.1145/2542661.2544374}{The Increasing Significance of Intermediate Representations in Compilers (Fred Chow; 2013)}
  \end{center}
\end{frame}
%%%%%%%%%%%%%%%%%%%%

%%%%%%%%%%%%%%%%%%%%
\begin{frame}{}
  \fig{width = 0.70\textwidth}{figs/LLVM-logo}
  \begin{center}
    LLVM 的核心就是它的 LLVM IR

    \vspace{0.60cm}
    \red{(希望下一轮授课可以加入 LLVM IR 内容)}
  \end{center}
\end{frame}
%%%%%%%%%%%%%%%%%%%%

%%%%%%%%%%%%%%%%%%%%
% \begin{frame}{}
%   \fig{width = 0.95\textwidth}{figs/llvm-ir}
% \end{frame}
%%%%%%%%%%%%%%%%%%%%

%%%%%%%%%%%%%%%%%%%%
\begin{frame}{}
  \begin{center}
    \red{\bf Intermediate Representation (IR)}
    \fig{width = 0.70\textwidth}{figs/irs}

    \vspace{1.00cm}
    \blue{\bf 精确:} 不能丢失源程序的信息

    \vspace{0.30cm}
    \blue{\bf 独立:} 不依赖特定的源语言与目标语言
    \vspace{0.20cm}

    \teal{(如, 没有复杂的寻址方式)}
  \end{center}
\end{frame}
%%%%%%%%%%%%%%%%%%%%

%%%%%%%%%%%%%%%%%%%%
\begin{frame}{}
  \fig{width = 0.80\textwidth}{figs/fangzhou}
  \begin{center}
    华为方舟编译器的 Maple IR 采用多层设计
  \end{center}
\end{frame}
%%%%%%%%%%%%%%%%%%%%

%%%%%%%%%%%%%%%%%%%%
\begin{frame}{}
  \begin{center}
    \red{\bf Intermediate Representation (IR)}
    \fig{width = 0.70\textwidth}{figs/irs}

    \vspace{0.80cm}
    图 (抽象语法树)、\blue{\bf 三地址代码}、C 语言
  \end{center}
\end{frame}
%%%%%%%%%%%%%%%%%%%%