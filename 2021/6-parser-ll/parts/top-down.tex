% top-down.tex

%%%%%%%%%%%%%%%%%%%%
\begin{frame}{}
  \begin{center}
    \blue{带记忆功能的}\purple{可回溯的}\red{递归下降的}语法分析器

    \vspace{0.50cm}
    \fig{width = 0.50\textwidth}{figs/antlr-logo}
    \vspace{0.50cm}

    甚至可以使用\violet{谓词解析器}处理\green{上下文相关文法}
  \end{center}
\end{frame}
%%%%%%%%%%%%%%%%%%%%

%%%%%%%%%%%%%%%%%%%%
\begin{frame}{}
  \fig{width = 1.00\textwidth}{figs/exprpred}

  \begin{center}
    \texttt{6-parser-ll/ExprPred.g4}
  \end{center}
\end{frame}
%%%%%%%%%%%%%%%%%%%%

%%%%%%%%%%%%%%%%%%%%
\begin{frame}{}
  \begin{center}
    只考虑\red{\bf 无二义性}的文法 \\[4pt]
    这意味着, 每个句子对应唯一的一棵语法分析树

    \fig{width = 0.60\textwidth}{figs/cfg-hierarchy}

    今日份主题: \red{\bf $LL(1)$ 语法分析器}
  \end{center}
\end{frame}
%%%%%%%%%%%%%%%%%%%%

%%%%%%%%%%%%%%%%%%%%
\begin{frame}{}
  \begin{center}
    自顶向下的、\\[15pt]
    递归下降的、\\[15pt]
    预测分析的、\\[15pt]
    适用于\red{\bf $LL(1)$文法}的、\\[15pt]
    $LL(1)$语法分析器
  \end{center}
\end{frame}
%%%%%%%%%%%%%%%%%%%%

%%%%%%%%%%%%%%%%%%%%
\begin{frame}{}
  \begin{center}
    {\large \red{\bf 自顶向下}构建语法分析树}

    \vspace{0.60cm}
    \blue{\bf 根节点}是文法的起始符号 $S$

    \uncover<2->{
      \vspace{0.80cm}
      每个\blue{\bf 中间节点}表示\purple{\bf 对某个非终结符应用某个产生式进行推导}

      \vspace{0.20cm}
      (\red{$Q:$} 选择哪个非终结符, 以及选择哪个产生式)}

    \vspace{0.80cm}
    \blue{\bf 叶节点}是词法单元流 $w\$$ \\[8pt]
    仅包含终结符号与特殊的\teal{\bf 文件结束符$\$$ (\texttt{EOF})}
  \end{center}
\end{frame}
%%%%%%%%%%%%%%%%%%%%

%%%%%%%%%%%%%%%%%%%%
\begin{frame}{}
  \begin{center}
    每个\blue{\bf 中间节点}表示\purple{\bf 对某个非终结符应用某个产生式进行推导}

    \vspace{0.50cm}
    \red{$Q:$} 选择哪个非终结符, 以及选择哪个产生式

    \pause
    \vspace{1.50cm}
    在推导的每一步, $L\red{L}(1)$ 总是选择\red{\bf 最左边的非终结符进行展开}

    \pause
    \vspace{1.60cm}
    $\red{L}L(1)$: 从左向右读入字符串
  \end{center}
\end{frame}
%%%%%%%%%%%%%%%%%%%%

%%%%%%%%%%%%%%%%%%%%
\begin{frame}{}
  \begin{center}
    {\large \red{\bf 递归下降}的实现框架}

    \fig{width = 0.90\textwidth}{figs/recursive-descent}

    为每个\blue{\bf 非终结符}写一个\teal{\bf 递归函数}

    \vspace{0.20cm}
    内部按需调用其它非终结符对应的递归函数
  \end{center}
\end{frame}
%%%%%%%%%%%%%%%%%%%%

%%%%%%%%%%%%%%%%%%%%
\begin{frame}{}
  \begin{center}
    % cfg-s-f-a.tex
% see https://en.wikipedia.org/wiki/LL_parser

\begin{empheq}[box=\widefbox]{align*}
  S &\to F \\[8pt]
  S &\to (S + F) \\[8pt]
  F &\to a
\end{empheq}
    \[
      w = ((a + a) + a)
    \]
  \end{center}
\end{frame}
%%%%%%%%%%%%%%%%%%%%

%%%%%%%%%%%%%%%%%%%%
\begin{frame}{}
  \begin{center}
    演示\red{\bf 递归下降}过程 (\texttt{jflap: SFa.jff})

    \begin{columns}
      \column{0.50\textwidth}
        % cfg-s-f-a.tex
% see https://en.wikipedia.org/wiki/LL_parser

\begin{empheq}[box=\widefbox]{align*}
  S &\to F \\[8pt]
  S &\to (S + F) \\[8pt]
  F &\to a
\end{empheq}
      \column{0.50\textwidth}
        \fig{width = 0.90\textwidth}{figs/tree-s-f-a}

        \vspace{-0.80cm}
        \[
          \red{w = ((a + a) + a)}
        \]
    \end{columns}

    \pause
    \vspace{0.30cm}
    每次都选择语法分析树\red{\bf 最左边}的非终结符进行展开
  \end{center}
\end{frame}
%%%%%%%%%%%%%%%%%%%%

%%%%%%%%%%%%%%%%%%%%
\begin{frame}{}
  \begin{center}
    同样是展开非终结符$S$, \\[4pt]
    为什么前两次选择了 $S \to (S + F)$, 而第三次选择了 $S \to F$?

    \begin{columns}
      \column{0.50\textwidth}
        % cfg-s-f-a.tex
% see https://en.wikipedia.org/wiki/LL_parser

\begin{empheq}[box=\widefbox]{align*}
  S &\to F \\[8pt]
  S &\to (S + F) \\[8pt]
  F &\to a
\end{empheq}
      \column{0.50\textwidth}
        \fig{width = 0.90\textwidth}{figs/tree-s-f-a}

        \vspace{-0.80cm}
        \[
          \red{w = ((a + a) + a)}
        \]
    \end{columns}

    \pause
    \vspace{0.50cm}
    因为它们面对的\red{\bf 当前词法单元}不同
  \end{center}
\end{frame}
%%%%%%%%%%%%%%%%%%%%

%%%%%%%%%%%%%%%%%%%%
\begin{frame}{}
  \begin{center}
    {\large 使用\red{\bf 预测分析表}确定产生式}
    % cfg-s-f-a.tex
% see https://en.wikipedia.org/wiki/LL_parser

\begin{empheq}[box=\widefbox]{align*}
  S &\to F \\[8pt]
  S &\to (S + F) \\[8pt]
  F &\to a
\end{empheq}

    % table-s-f-a.tex

% \usepackage{graphicx}
\begin{table}[]
  \centering
  \resizebox{0.50\textwidth}{!}{
    \begin{tabular}{|c||c|c|c|c|c|}
      \hline
      &  \red{$($} & \red{$)$} & \red{$a$} & \red{$+$} & \red{\$} \\ \hline \hline
  \blue{$S$} & \teal{2} &  & \teal{1} &  &  \\ \hline
  \blue{$F$} &  &  & \teal{3} &  &  \\ \hline
    \end{tabular}}
\end{table}
    指明了每个\blue{\bf 非终结符}在面对不同的\red{\bf 词法单元或文件结束符}时, \\[4pt]
    该选择哪个\teal{\bf 产生式} (按编号进行索引) 或者\teal{\bf 报错}
  \end{center}
\end{frame}
%%%%%%%%%%%%%%%%%%%%

%%%%%%%%%%%%%%%%%%%%
\begin{frame}{}
  \begin{center}
    \begin{definition}[$LL(1)$文法]
      如果文法 $G$ 的\red{\bf 预测分析表}是\blue{\bf 无冲突}的,
      则 $G$ 是 $LL(1)$ 文法。
    \end{definition}

    \blue{\bf 无冲突:} 每个单元格里只有一个生成式(编号) \\[8pt]

    \begin{columns}
      \column{0.50\textwidth}
        % cfg-s-f-a.tex
% see https://en.wikipedia.org/wiki/LL_parser

\begin{empheq}[box=\widefbox]{align*}
  S &\to F \\[8pt]
  S &\to (S + F) \\[8pt]
  F &\to a
\end{empheq}
      \column{0.50\textwidth}
        % table-s-f-a.tex

% \usepackage{graphicx}
\begin{table}[]
  \centering
  \resizebox{0.50\textwidth}{!}{
    \begin{tabular}{|c||c|c|c|c|c|}
      \hline
      &  \red{$($} & \red{$)$} & \red{$a$} & \red{$+$} & \red{\$} \\ \hline \hline
  \blue{$S$} & \teal{2} &  & \teal{1} &  &  \\ \hline
  \blue{$F$} &  &  & \teal{3} &  &  \\ \hline
    \end{tabular}}
\end{table}
    \end{columns}

    \vspace{0.60cm}
    对于当前选择的\blue{非终结符}, \\[4pt]
    仅根据输入中\red{当前的词法单元} ($LL(\red{1})$) 即可确定需要使用哪条\teal{产生式}
  \end{center}
\end{frame}
%%%%%%%%%%%%%%%%%%%%

%%%%%%%%%%%%%%%%%%%%
\begin{frame}{}
  \begin{center}
    {\large \red{\bf 递归下降的、预测分析}实现方法}

    \begin{columns}
      \column{0.50\textwidth}
        % cfg-s-f-a.tex
% see https://en.wikipedia.org/wiki/LL_parser

\begin{empheq}[box=\widefbox]{align*}
  S &\to F \\[8pt]
  S &\to (S + F) \\[8pt]
  F &\to a
\end{empheq}
        % table-s-f-a.tex

% \usepackage{graphicx}
\begin{table}[]
  \centering
  \resizebox{0.50\textwidth}{!}{
    \begin{tabular}{|c||c|c|c|c|c|}
      \hline
      &  \red{$($} & \red{$)$} & \red{$a$} & \red{$+$} & \red{\$} \\ \hline \hline
  \blue{$S$} & \teal{2} &  & \teal{1} &  &  \\ \hline
  \blue{$F$} &  &  & \teal{3} &  &  \\ \hline
    \end{tabular}}
\end{table}
        % match.tex

\begin{algorithm}[H]
% \caption{}
% \label{alg:S}
\begin{algorithmic}[1]
  \Procedure{match}{$t$} % \Comment{$t$ 是预期的词法单元}
    \If{token = $t$}  % \Comment{token 是输入中当前的词法单元}
      \State token $\gets \Call{\purple{next-token}}{\null}$
    \Else
      \State \Call{error}{token, $t$}
    \EndIf
  \EndProcedure
\end{algorithmic}
\end{algorithm}
      \column{0.50\textwidth}
        \only<1>{% ll1-s-f-a-S.tex

\begin{algorithm}[H]
% \caption{}
% \label{alg:S}
\begin{algorithmic}[1]
  \Procedure{\blue{$S$}}{\null}
    \If{\red{token = `('}}
      \State \Call{match}{`('}
      \State \Call{$S$}{\null}
      \State \Call{match}{`+'}
      \State \Call{$F$}{\null}
      \State \Call{match}{`)'}
    \ElsIf{\red{token = `a'}}
      \State \Call{$F$}{\null}
    \Else
      \State \Call{\teal{error}}{token, \set{`(', `a'}}
    \EndIf
  \EndProcedure
\end{algorithmic}
\end{algorithm}}
        \only<2>{% ll1-s-f-a-F.tex

\begin{algorithm}[H]
% \caption{}
% \label{alg:S}
\begin{algorithmic}[1]
  \Procedure{\blue{$F$}}{\null}
    \If{\red{token = `a'}}
      \State \Call{match}{`a'}
    \Else
      \State \Call{\teal{error}}{token, \set{`a'}}
    \EndIf
  \EndProcedure
\end{algorithmic}
\end{algorithm}}
    \end{columns}
  \end{center}
\end{frame}
%%%%%%%%%%%%%%%%%%%%

%%%%%%%%%%%%%%%%%%%%
\begin{frame}{}
  \begin{center}
    再次演示\red{\bf 递归下降}过程 (\texttt{jflap: SFa.jff})

    \vspace{0.30cm}
    \fig{width = 0.80\textwidth}{figs/jflap-aaa}
  \end{center}
\end{frame}
%%%%%%%%%%%%%%%%%%%%

%%%%%%%%%%%%%%%%%%%%
\begin{frame}{}
  \begin{center}
    \texttt{6-parser-ll/OptionalInit.g4}
    \fig{width = 0.60\textwidth}{figs/optional-init}

    \pause
    \texttt{int x = y; \qquad int x;}

    \pause
    \vspace{0.20cm}
    \texttt{f(int x = y) \qquad f(int x)}
  \end{center}
\end{frame}
%%%%%%%%%%%%%%%%%%%%

%%%%%%%%%%%%%%%%%%%%
\begin{frame}{}
  \begin{center}
    \red{\large 如何计算给定文法$G$的预测分析表?}

    \vspace{0.60cm}
    $\first(\alpha)$是可从$\alpha$推导得到的句型的\blue{\bf 首终结符号}的集合
    \begin{definition}[$\first(\alpha)$集合]
      对于任意的(产生式的右部) $\alpha \in (N \cup T)^{\ast}:$
      \[
        \first(\alpha) = \Big\{t \in T \cup \set{\epsilon}
          \mid \alpha \dstar \red{t}\beta \lor \alpha \dstar \epsilon \Big\}.
      \]
    \end{definition}

    \pause
    \vspace{0.60cm}
    考虑非终结符 $A$ 的所有产生式
    $A \to \alpha_{1}, A \to \alpha_{2}, \dots, A \to \alpha_{m}$, \\[4pt]
    如果\red{它们对应的 $\first(\alpha_{i})$ 集合互不相交}, \\[4pt]
    则只需查看当前输入词法单元, 即可确定选择哪个产生式 (\teal{或报错})
  \end{center}
\end{frame}
%%%%%%%%%%%%%%%%%%%%

%%%%%%%%%%%%%%%%%%%%
\begin{frame}{}
  \begin{center}
    \red{\large 如何计算给定文法$G$的预测分析表?}

    \vspace{0.60cm}
    $\follow(A)$是可能在某些句型中\blue{\bf 紧跟在$A$右边的终结符}的集合
    \begin{definition}[$\follow(A)$集合]
        对于任意的(产生式的左部)非终结符 $A \in N:$
        \[
          \follow(A) = \Big\{t \in T \cup \set{\$}
            \mid \exists s.\; S \dstar s \triangleq \beta A\red{t} \gamma\Big\}.
        \]
    \end{definition}

    \pause
    \vspace{0.60cm}
    考虑产生式 $A \to \alpha$, \\[4pt]
    如果从 $\alpha$ 可能推导出空串 (\red{$\alpha \dstar \epsilon$}), \\[4pt]
    则只有当当前词法单元 $t \in \follow(A)$, 才可以选择该产生式
  \end{center}
\end{frame}
%%%%%%%%%%%%%%%%%%%%

%%%%%%%%%%%%%%%%%%%%
\begin{frame}{}
  \begin{center}
    \red{\large 先计算每个符号 $X$ 的 $\first(X)$集合}

    % first-X.tex

\begin{algorithm}[H]
% \caption{}
% \label{alg:S}
\begin{algorithmic}[1]
  \Procedure{\purple{first}}{$X$}
    \pause
    \If{\red{$X \in T$}}  \Comment{\blue{规则1: $X$是终结符}}
      \State $\first(X) = X$
    \EndIf

    \pause
    \hStatex
    \For{\red{$X \to Y_{1} Y_{2} \dots Y_{k}$}} \Comment{\blue{规则2: $X$是非终结符}}
      \State $\first(X) \gets \first(X) \cup \set{\first(Y_{1}) \;\teal{\setminus\; \set{\epsilon}}}$
      \For{$i \gets 2 \;\text{\bf to } k$}
          \If{$\epsilon \in L(Y_{1} \dots Y_{i-1})$}
            \State $\first(X) \gets \first(X) \cup \set{\first(Y_{i}) \;\teal{\setminus\; \set{\epsilon}}}$
          \EndIf
      \EndFor
      \pause
      \If{\red{$\epsilon \in L(Y_{1} \dots Y_{k})$}} \Comment{\blue{规则3: $X$ 可推导出空串}}
        \State $\first(X) \gets \first(X) \cup \set{\epsilon}$
      \EndIf
    \EndFor
  \EndProcedure
\end{algorithmic}
\end{algorithm}

    \pause
    \purple{不断应用上面的规则, 直到每个 $\first(X)$ 都不再变化 ({\bf 闭包!!!})}
  \end{center}
\end{frame}
%%%%%%%%%%%%%%%%%%%%

%%%%%%%%%%%%%%%%%%%%
\begin{frame}{}
  \begin{center}
    \red{\large 再计算每个符号串 $\alpha$ 的 $\first(\alpha)$集合}

    \[
      \alpha = X \beta
    \]
    \[
      \first(\alpha) =
      \begin{cases}
        \first(X) & \epsilon \notin L(X) \\[6pt]
        (\first(X) \;\teal{\setminus\; \set{\epsilon}}) \cup \first(\beta) & \epsilon \in L(X)
      \end{cases}
    \]

    \vspace{0.60cm}
    最后, 如果 $\epsilon \in L(\alpha)$, 则将 $\epsilon$ 加入 $\first(\alpha)$。
  \end{center}
\end{frame}
%%%%%%%%%%%%%%%%%%%%

%%%%%%%%%%%%%%%%%%%%
\begin{frame}{}
  \begin{columns}
    \column{0.50\textwidth}
      % cfg-XYZ.tex

\begin{empheq}[box=\widefbox]{align*}
  (1)\; X &\to Y \\[8pt]
  (2)\; X &\to a \\[8pt]
  (3)\; Y &\to \epsilon \\[8pt]
  (4)\; Y &\to c \\[8pt]
  (5)\; Z &\to d \\[8pt]
  (6)\; Z &\to XYZ
\end{empheq}
      \begin{center}
        \texttt{jflap: XYZ.jff}
      \end{center}
    \column{0.50\textwidth}
      \pause
      \begin{align*}
        \first(X) &= \set{a, c, \epsilon} \\
        \first(Y) &= \set{c, \epsilon} \\
        \first(Z) &= \set{a, c, d} \\[8pt]
        \first(XYZ) &= \set{a, c, d}
      \end{align*}
  \end{columns}
\end{frame}
%%%%%%%%%%%%%%%%%%%%

%%%%%%%%%%%%%%%%%%%%
\begin{frame}{}
  \begin{center}
    {\large 为每个非终结符$X$计算$\follow(X)$集合}

    % follow-X.tex

\begin{algorithm}[H]
% \caption{}
% \label{alg:S}
\begin{algorithmic}[1]
  \Procedure{\purple{follow}}{$X$}
    \pause
    \For{\red{$X$ 是开始符号}}  \Comment{\blue{规则1: $X$ 是开始符号}}
      \State $\follow(X) \gets \follow(X) \cup \set{\$}$
    \EndFor

    \pause
    \hStatex
    \For{\red{$A \to \alpha X$}} \Comment{\blue{规则3: $X$ 是某产生式右部的最后一个符号}}
      \State $\follow(X) \gets \follow(X) \cup \follow(A)$
    \EndFor

    \pause
    \hStatex
    \For{\red{$A \to \alpha X \beta$}} \Comment{\blue{规则2: $X$ 是某产生式右部中间的一个符号}}
      \State $\follow(X) \gets \follow(X) \cup (\first(\beta) \setminus \set{\epsilon})$
      \If{\red{$\epsilon \in \first(\beta)$}}
        \State $\follow(X) \gets \follow(X) \cup \follow(A)$
      \EndIf
    \EndFor
  \EndProcedure
\end{algorithmic}
\end{algorithm}

    \pause
    \purple{不断应用上面的规则, 直到每个 $\follow(X)$ 都不再变化 ({\bf 闭包!!!})}
  \end{center}
\end{frame}
%%%%%%%%%%%%%%%%%%%%

%%%%%%%%%%%%%%%%%%%%
\begin{frame}{}
  \begin{columns}
    \column{0.50\textwidth}
      % cfg-XYZ.tex

\begin{empheq}[box=\widefbox]{align*}
  (1)\; X &\to Y \\[8pt]
  (2)\; X &\to a \\[8pt]
  (3)\; Y &\to \epsilon \\[8pt]
  (4)\; Y &\to c \\[8pt]
  (5)\; Z &\to d \\[8pt]
  (6)\; Z &\to XYZ
\end{empheq}
      \begin{center}
        \texttt{jflap: XYZ.jff}
      \end{center}
    \column{0.50\textwidth}
      \pause
      \begin{align*}
        \follow(X) &= \set{a, c, d, \$} \\
        \follow(Y) &= \set{a, c, d, \$} \\
        \follow(Z) &= \emptyset
      \end{align*}
  \end{columns}
\end{frame}
%%%%%%%%%%%%%%%%%%%%

%%%%%%%%%%%%%%%%%%%%
\begin{frame}{}
  \begin{center}
    \red{\large 如何根据\teal{\first{}与\follow{}集合}计算给定文法$G$的预测分析表?}

    \vspace{0.80cm}
    按照以下规则, 在表格 $[A, t]$ 中填入生成式 $A \to \alpha$ (编号):

    \begin{gather}
      t \in \first(\alpha) \\[6pt]
      \alpha \dstar \epsilon \land t \in \follow(A)
    \end{gather}

    \pause
    \begin{definition}[$LL(1)$文法]
      如果文法 $G$ 的\red{\bf 预测分析表}是\blue{\bf 无冲突}的,
      则 $G$ 是 $LL(1)$ 文法。
    \end{definition}
  \end{center}
\end{frame}
%%%%%%%%%%%%%%%%%%%%

%%%%%%%%%%%%%%%%%%%%
\begin{frame}{}
  \begin{center}
    按照以下规则, 在表格 $[A, t]$ 中填入生成式 $A \to \alpha$ (编号):

    \setcounter{equation}{0}
    \begin{gather}
      t \in \first(\alpha) \\[6pt]
      \alpha \dstar \epsilon \land t \in \follow(A)
    \end{gather}

    \vspace{0.60cm}
    \blue{\bf 因其``唯一'' (\purple{无冲突}), 必要变充分}
  \end{center}
\end{frame}
%%%%%%%%%%%%%%%%%%%%

%%%%%%%%%%%%%%%%%%%%
\begin{frame}{}
  \begin{columns}
    \column{0.50\textwidth}
      % cfg-XYZ.tex

\begin{empheq}[box=\widefbox]{align*}
  (1)\; X &\to Y \\[8pt]
  (2)\; X &\to a \\[8pt]
  (3)\; Y &\to \epsilon \\[8pt]
  (4)\; Y &\to c \\[8pt]
  (5)\; Z &\to d \\[8pt]
  (6)\; Z &\to XYZ
\end{empheq}
    \column{0.50\textwidth}
      \begin{align*}
        \first(X) &= \set{a, c, \epsilon} \\
        \first(Y) &= \set{c, \epsilon} \\
        \first(Z) &= \set{a, c, d} \\[8pt]
        \first(XYZ) &= \set{a, c, d}
      \end{align*}
      \begin{align*}
        \follow(X) &= \set{a, c, d, \$} \\
        \follow(Y) &= \set{a, c, d, \$} \\
        \follow(Z) &= \emptyset
      \end{align*}
  \end{columns}

  % ll-table-XYZ.tex

% \usepackage{graphicx}
\begin{table}[]
  \centering
  \resizebox{0.40\textwidth}{!}{
    \begin{tabular}{|c||c|c|c|c|c|}
      \hline
      &  \red{$a$} & \red{$c$} & \red{$d$} & \red{\$} \\ \hline \hline
  \blue{$X$} & \teal{1, 2} & \teal{1} & \teal{1} & \teal{1}  \\ \hline
  \blue{$Y$} & \purple{3} & \teal{\purple{3}, 4} & \purple{3} & \purple{3} \\ \hline
  \blue{$Z$} & \teal{6} & \teal{6} & \teal{5, 6} &  \\ \hline
    \end{tabular}}
\end{table}
\end{frame}
%%%%%%%%%%%%%%%%%%%%

%%%%%%%%%%%%%%%%%%%%
\begin{frame}{}
  \begin{center}
    {\large \red{\bf $LL(1)$ 语法分析器}}

    \vspace{0.80cm}
    \begin{description}
      \setlength{\itemsep}{12pt}
      \item[$L:$] \purple{从左向右} (left-to-right) 扫描输入
      \item[$L:$] 构建\purple{最左 (leftmost) 推导}
      \item[$1:$] 只需向前看\purple{一个输入符号}便可确定使用哪条产生式
    \end{description}
  \end{center}
\end{frame}
%%%%%%%%%%%%%%%%%%%%

%%%%%%%%%%%%%%%%%%%%
\begin{frame}{}
  \fig{width = 0.60\textwidth}{figs/cfg-hierarchy}

  \begin{center}
    \red{What is $LL(0)$?}
  \end{center}
\end{frame}
%%%%%%%%%%%%%%%%%%%%

%%%%%%%%%%%%%%%%%%%%
\begin{frame}{}
  \begin{center}
    \red{\large {\bf 非递归}的预测分析算法}

    \vspace{0.60cm}
    \fig{width = 0.70\textwidth}{figs/ll1-iteration-model}
  \end{center}
\end{frame}
%%%%%%%%%%%%%%%%%%%%

%%%%%%%%%%%%%%%%%%%%
\begin{frame}{}
  \begin{center}
    \red{\large {\bf 非递归}的预测分析算法}

    \vspace{0.60cm}
    \fig{width = 0.95\textwidth}{figs/ll1-iteration-alg}
  \end{center}
\end{frame}
%%%%%%%%%%%%%%%%%%%%

%%%%%%%%%%%%%%%%%%%%
\begin{frame}{}
  \begin{center}
    不是 $LL(1)$ 文法怎么办?

    \vspace{0.60cm}
    \red{\bf 改造它}

    \vspace{0.80cm}

    \blue{消除左递归}

    \vspace{0.30cm}
    \blue{提取左公因子}
  \end{center}
\end{frame}
%%%%%%%%%%%%%%%%%%%%

%%%%%%%%%%%%%%%%%%%%
\begin{frame}{}
  \begin{center}
    \uncover<2->{
      $E$ 在\purple{\bf 不消耗任何词法单元}的情况下, 直接递归调用 $E$, 造成\green{\bf 死循环}
    }
    % cfg-expr-add-sub-mul-div-ETF.tex

\begin{empheq}[box=\widefbox]{align*}
  E &\to E + T \mid E - T \mid T \\[8pt]
  T &\to T \ast F \mid T / F \mid F \\[8pt]
  F &\to (E) \mid \id \mid \num
\end{empheq}

    \uncover<3->{
      \[
        \first(E + T) \cap \first(T) \neq \emptyset
      \]
      \green{\bf 不是 $LL(1)$ 文法}
    }
  \end{center}
\end{frame}
%%%%%%%%%%%%%%%%%%%%

%%%%%%%%%%%%%%%%%%%%
\begin{frame}{}
  \begin{center}
    \begin{columns}
      \column{0.28\textwidth}
        % cfg-expr-add-mul-no-left-recursion.tex

\begin{empheq}[box=\widefbox]{align*}
  E &\to T E' \\[8pt]
  E' &\to +\; T E' \mid \epsilon \\[8pt]
  T &\to F T' \\[8pt]
  T' &\to \ast\; F T' \mid \epsilon \\[8pt]
  F &\to (E) \mid \id
\end{empheq}
      \column{0.72\textwidth}
        \uncover<2->{\fig{width = 1.00\textwidth}{figs/ll1-table-expr}}
    \end{columns}

    \begin{columns}
      \column{0.50\textwidth}
        \begin{align*}
          \first(F) &= \set{(, \id} \\
          \first(T) &= \set{(, \id} \\
          \red{\first(E)} &= \set{(, \id} \\
          \first(E') &= \set{+, \blue{\epsilon}} \\
          \first(T') &= \set{\ast, \epsilon}
        \end{align*}
      \column{0.50\textwidth}
        \begin{align*}
          \follow(E) &= \blue{\follow(E')} = \set{), \$} \\
          \follow(T) &= \follow(T') = \set{+, ), \$} \\
          \follow(F) &= \set{+, \ast, ), \$}
        \end{align*}
    \end{columns}
  \end{center}
\end{frame}
%%%%%%%%%%%%%%%%%%%%

%%%%%%%%%%%%%%%%%%%%
\begin{frame}{}
  \begin{center}
    \blue{\bf 文件结束符 $\$$的必要性}

    \fig{width = 0.70\textwidth}{figs/ll1-stack-expr}
  \end{center}
\end{frame}
%%%%%%%%%%%%%%%%%%%%

%%%%%%%%%%%%%%%%%%%%
\begin{frame}{}
  \begin{center}
    ANTLR4 可以处理直接左递归文法, \red{\bf 不要}改写文法

    \vspace{0.50cm}
    \texttt{5-parser-antlr/Expr.g4}

    \vspace{0.50cm}
    \fig{width = 0.30\textwidth}{figs/stay-tuned}
  \end{center}
\end{frame}
%%%%%%%%%%%%%%%%%%%%

%%%%%%%%%%%%%%%%%%%%
\begin{frame}{}
  \begin{center}
    \fig{width = 0.60\textwidth}{figs/cfg-if-then-else-short}

    \vspace{0.50cm}
    \red{\bf 提取左公因子}
    \vspace{0.50cm}

    \fig{width = 0.40\textwidth}{figs/cfg-if-then-else-short-left-factor}
  \end{center}
\end{frame}
%%%%%%%%%%%%%%%%%%%%

%%%%%%%%%%%%%%%%%%%%
\begin{frame}{}
  \fig{width = 0.30\textwidth}{figs/cfg-if-then-else-short-left-factor}

  \fig{width = 0.90\textwidth}{figs/ll1-table-if-then-else-short}

  \pause
  \[
    \ifkw\; E_{1} \;\thenkw\; \ifkw\; E_{2} \;\thenkw\; S_{1} \;\elsekw\; S_{2}
  \]

  \begin{center}
    \red{\bf 解决二义性:} 选择 $S' \to eS$, 将 $\elsekw$ 与前面最近的 $\thenkw$ 关联起来
  \end{center}
\end{frame}
%%%%%%%%%%%%%%%%%%%%

%%%%%%%%%%%%%%%%%%%%
\begin{frame}{}
  \begin{center}
    ANTLR4 可以处理有左公因子的文法, \red{\bf 不要}改写文法 (\gray{暂时存疑})

    \vspace{0.50cm}
    \texttt{6-parser-ll/IfStat.g4}

    \vspace{0.80cm}
    \texttt{if a then if a then if a then b else c else c else c}
  \end{center}
\end{frame}
%%%%%%%%%%%%%%%%%%%%