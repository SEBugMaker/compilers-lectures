% cfg.tex

%%%%%%%%%%%%%%%%%%%%
\begin{frame}{}
  \fig{width = 0.50\textwidth}{figs/cfg}

  \begin{center}
    上下文无关文法
  \end{center}
\end{frame}
%%%%%%%%%%%%%%%%%%%%

%%%%%%%%%%%%%%%%%%%%
\begin{frame}{}
  \begin{definition}[Context-Free Grammar (CFG); 上下文无关文法]
    上下文无关文法 $G$ 是一个四元组 $G = (T, N, P, S)$:
    \vspace{0.30cm}

    \begin{itemize}
      \setlength{\itemsep}{8pt}
      \item $T$ 是\red{\bf 终结符号} (\teal{Terminal}) 集合, 对应于词法分析器产生的词法单元;
      \item $N$ 是\red{\bf 非终结符号} (\teal{Non-terminal}) 集合;
      \item $P$ 是\red{\bf 产生式} (\teal{Production}) 集合;
        \[
          \boxed{A \in N \longrightarrow \alpha \in (T \cup N)^{\ast}}
        \]
        \vspace{-0.60cm}
        \begin{description}
          \setlength{\itemsep}{5pt}
          \item[头部/左部 (Head) $A$:] \purple{\bf 单个}非终结符
          \item[体部/右部 (Body) $\alpha$:] 终结符与非终结符构成的串, 也可以是空串 $\epsilon$
        \end{description}
      \item $S$ 为\red{\bf 开始} (\teal{Start}) 符号。要求 $S \in N$且唯一。
    \end{itemize}
  \end{definition}
\end{frame}
%%%%%%%%%%%%%%%%%%%%

%%%%%%%%%%%%%%%%%%%%
\begin{frame}{}
  \[
    G = (\set{a,b}, \set{S}, P, S)
  \]

  % cfg-anbn.tex

\begin{empheq}[box=\widefbox]{align*}
  S &\to aSb \\[8pt]
  S &\to \epsilon
\end{empheq}
\end{frame}
%%%%%%%%%%%%%%%%%%%%

%%%%%%%%%%%%%%%%%%%%
\begin{frame}{}
  \[
    G = (\set{(,)}, \set{S}, P, S)
  \]

  % cfg-bp.tex

\begin{empheq}[box=\widefbox]{align*}
  S &\to SS \\[8pt]
  S &\to (S) \\[8pt]
  S &\to () \\[8pt]
  S &\to \epsilon
\end{empheq}
\end{frame}
%%%%%%%%%%%%%%%%%%%%

%%%%%%%%%%%%%%%%%%%%
% \begin{frame}{}
%   \[
%     G = (\set{a,b}, \set{S}, P, S)
%   \]
%
%   % cfg-palindrome.tex

\begin{empheq}[box=\widefbox]{align*}
  S &\to aSa \\[8pt]
  S &\to bSb \\[8pt]
  S &\to a  \\[8pt]
  S &\to b \\[8pt]
  S &\to \epsilon
\end{empheq}
% \end{frame}
%%%%%%%%%%%%%%%%%%%%

%%%%%%%%%%%%%%%%%%%%
\begin{frame}{}
  \begin{center}
    \fig{width = 0.75\textwidth}{figs/cfg-if-then-else}

    \vspace{0.60cm}
    \blue{\bf 条件语句}文法

    \vspace{0.20cm}
    \red{\bf 悬空 (Dangling)-else} 文法
  \end{center}
\end{frame}
%%%%%%%%%%%%%%%%%%%%

%%%%%%%%%%%%%%%%%%%%
\begin{frame}{}
  \fig{width = 0.90\textwidth}{figs/cfg-if-then-else-list}

  \vspace{0.30cm}
  \begin{center}
    \blue{\bf 约定:} 如果没有明确指定, 第一个产生式的头部就是开始符号
  \end{center}
\end{frame}
%%%%%%%%%%%%%%%%%%%%

%%%%%%%%%%%%%%%%%%%%
\begin{frame}{}
  \begin{center}
    关于\blue{\bf 终结符号}的约定

    \vspace{0.30cm}
    \fig{width = 1.00\textwidth}{figs/convention-terminal}
  \end{center}
\end{frame}
%%%%%%%%%%%%%%%%%%%%

%%%%%%%%%%%%%%%%%%%%
\begin{frame}{}
  \begin{center}
    关于\blue{\bf 非终结符号}的约定

    \vspace{0.30cm}
    \fig{width = 0.90\textwidth}{figs/convention-nonterminal}
  \end{center}
\end{frame}
%%%%%%%%%%%%%%%%%%%%

%%%%%%%%%%%%%%%%%%%%
% \begin{frame}{}
%   \begin{center}
%     \fig{width = 1.00\textwidth}{figs/convention-symbols}
%   \end{center}
% \end{frame}
%%%%%%%%%%%%%%%%%%%%

%%%%%%%%%%%%%%%%%%%%
\begin{frame}{}
  \fig{width = 0.60\textwidth}{figs/syntax-semantics}
  \begin{center}
    语义: 上下文无关文法 $G$ 定义了一个\red{\bf 语言} $L(G)$

    \pause
    \vspace{0.50cm}
    语言是\blue{\bf 串}的集合

    \vspace{0.30cm}
    串从何来?
  \end{center}
\end{frame}
%%%%%%%%%%%%%%%%%%%%

%%%%%%%%%%%%%%%%%%%%
\begin{frame}{}
  \begin{center}
    {\large \red{\bf 推导}} (Derivation)

    % cfg-expr-add-mul.tex

\begin{empheq}[box=\widefbox]{align*}
  E \to E + E \mid E \ast E \mid (E) \mid \id
\end{empheq}

    \vspace{0.50cm}
    推导即是将某个产生式的左边\blue{\bf 替换}成它的右边

    \vspace{1.00cm}
    每一步推导需要选择\purple{\bf 替换哪个非终结符号}, 以及\purple{\bf 使用哪个产生式}
  \end{center}
\end{frame}
%%%%%%%%%%%%%%%%%%%%

%%%%%%%%%%%%%%%%%%%%
\begin{frame}{}
  \begin{center}
    {\large \red{\bf 推导}} (Derivation)
  \end{center}

  % cfg-expr-add-mul.tex

\begin{empheq}[box=\widefbox]{align*}
  E \to E + E \mid E \ast E \mid (E) \mid \id
\end{empheq}

  \vspace{-0.50cm}
  \[
    E \implies -E \implies -(E) \implies -(E + E) \implies -(\id + E) \implies -(\id + \id)
  \]

  \pause
  \vspace{-0.30cm}
  \begin{align*}
    &E \implies -E: \text{经过一步推导得出} \\
    &E \dplus -(\id + E): \text{经过一步或多步推导得出} \\
    &E \dstar -(\id + E): \text{经过零步或多步推导得出}
  \end{align*}

  \pause
  \vspace{-0.50cm}
  \[
    E \implies -E \implies -(E) \implies -(E + E) \implies \blue{-(E + \id)} \implies -(\id + \id)
  \]
\end{frame}
%%%%%%%%%%%%%%%%%%%%

%%%%%%%%%%%%%%%%%%%%
\begin{frame}{}
  \begin{definition}[Sentential Form; 句型]
    如果 $S \dstar \alpha$, 且 $\alpha \in (T \cup N)^{\ast}$,
    则称 $\alpha$ 是文法 $G$ 的一个\red{\bf 句型}。
  \end{definition}

  \vspace{0.30cm}
  % cfg-expr-add-mul.tex

\begin{empheq}[box=\widefbox]{align*}
  E \to E + E \mid E \ast E \mid (E) \mid \id
\end{empheq}

  \vspace{-0.80cm}
  \[
    E \implies -E \implies -(E) \implies -(E + E)
      \implies \red{-(\id + E)} \implies \blue{-(\id + \id)}
  \]

  \pause
  \begin{definition}[Sentence; 句子]
    如果 $S \dstar w$, 且 $w \in T^{\ast}$,
    则称 $w$ 是文法 $G$ 的一个\blue{\bf 句子}。
  \end{definition}
\end{frame}
%%%%%%%%%%%%%%%%%%%%

%%%%%%%%%%%%%%%%%%%%
\begin{frame}{}
  \begin{definition}[文法$G$生成的语言 $L(G)$]
    文法 $G$ 的\red{\bf 语言} $L(G)$ 是它能推导出的\blue{\bf 所有句子}构成的集合。

    \[
      w \in L(G) \iff S \dstar w
    \]
  \end{definition}
\end{frame}
%%%%%%%%%%%%%%%%%%%%

%%%%%%%%%%%%%%%%%%%%
\begin{frame}{}
  \begin{center}
    关于文法 $G$ 的\red{\bf 两个基本问题:}
  \end{center}

  \vspace{0.60cm}
  \begin{columns}
    \column{0.10\textwidth}
    \column{0.80\textwidth}
      \begin{itemize}
        \setlength{\itemsep}{15pt}
        \item \blue{\bf Membership 问题:}
          给定字符串 $x \in \red{T^{\ast}}$, $x \in L(G)$?
        \item \blue{$L(G)$} 究竟是什么?
      \end{itemize}
    \column{0.10\textwidth}
  \end{columns}
\end{frame}
%%%%%%%%%%%%%%%%%%%%

%%%%%%%%%%%%%%%%%%%%
\begin{frame}{}
  \begin{center}
    \blue{给定字符串 $x \in T^{\ast}$, $x \in L(G)$?}

    \vspace{0.20cm}
    (即, 检查$x$是否符合文法$G$)

    \pause
    \vspace{1.00cm}
    这就是\red{\bf 语法分析器}的任务:

    \vspace{0.30cm}
    为输入的词法单元流寻找推导、\purple{\bf 构建语法分析树}, 或者报错
  \end{center}
\end{frame}
%%%%%%%%%%%%%%%%%%%%

%%%%%%%%%%%%%%%%%%%%
\begin{frame}{}
  \begin{center}
    根节点是文法$G$的起始符号

    \fig{width = 0.60\textwidth}{figs/tree-statement-wiki}

    叶子节点是输入的词法单元流

    \vspace{0.50cm}
    常用的语法分析器以\red{\bf 自顶向下}或\red{\bf 自底向上}的方式构建中间部分
  \end{center}
\end{frame}
%%%%%%%%%%%%%%%%%%%%

%%%%%%%%%%%%%%%%%%%%
\begin{frame}{}
  \begin{center}
    \blue{$L(G)$ 是什么?}

    \vspace{0.60cm}
    这是\red{\bf 程序设计语言设计者}需要考虑的问题
  \end{center}
\end{frame}
%%%%%%%%%%%%%%%%%%%%

%%%%%%%%%%%%%%%%%%%%
\begin{frame}{}
  \begin{columns}
    \column{0.50\textwidth}
      % cfg-bp.tex

\begin{empheq}[box=\widefbox]{align*}
  S &\to SS \\[8pt]
  S &\to (S) \\[8pt]
  S &\to () \\[8pt]
  S &\to \epsilon
\end{empheq}

      \[
        L(G) = \pause \set{\text{良匹配括号串}}
      \]
    \column{0.50\textwidth}
      \pause
      % cfg-anbn.tex

\begin{empheq}[box=\widefbox]{align*}
  S &\to aSb \\[8pt]
  S &\to \epsilon
\end{empheq}

      \[
        L(G) = \pause \set{a^{n}b^{n} \mid n \ge 0}
      \]
  \end{columns}
\end{frame}
%%%%%%%%%%%%%%%%%%%%

%%%%%%%%%%%%%%%%%%%%
\begin{frame}{}
  \begin{center}
    字母表 $\Sigma = \set{a, b}$ 上的所有\blue{\bf 回文串} (Palindrome) 构成的语言

    \pause
    \vspace{0.30cm}
    % cfg-palindrome.tex

\begin{empheq}[box=\widefbox]{align*}
  S &\to aSa \\[8pt]
  S &\to bSb \\[8pt]
  S &\to a  \\[8pt]
  S &\to b \\[8pt]
  S &\to \epsilon
\end{empheq}

    \pause
    \[
      S \to a S a \;|\; bSb \;|\; a \;|\; b \;|\; \epsilon
    \]
  \end{center}
\end{frame}
%%%%%%%%%%%%%%%%%%%%

%%%%%%%%%%%%%%%%%%%%
\begin{frame}{}
  \[
    \set{b^{n}a^{m}b^{2n} \mid n \ge 0, m \ge 0}
  \]

  \pause
  \vspace{0.50cm}
  % cfg-bn-am-b2n.tex

\begin{empheq}[box=\widefbox]{align*}
  S &\to bSbb \mid A \\[8pt]
  A &\to aA \mid \epsilon
\end{empheq}
\end{frame}
%%%%%%%%%%%%%%%%%%%%

%%%%%%%%%%%%%%%%%%%%
\begin{frame}{}
  \[
    \set{x \in \set{a, b}^{\ast} \mid x \text{ 中 }\; a, b \text{ 个数\red{相同}}}
  \]

  \pause
  \vspace{0.50cm}
  % cfg-equal-number-a-b.tex

\begin{empheq}[box=\widefbox]{align*}
  V \to aVbV \mid bVaV \mid \epsilon
\end{empheq}
\end{frame}
%%%%%%%%%%%%%%%%%%%%

%%%%%%%%%%%%%%%%%%%%
\begin{frame}{}
  \[
    \set{x \in \set{a, b}^{\ast} \mid x \text{ 中 }\; a, b \text{ 个数\red{不同}}}
  \]

  \pause
  \vspace{0.50cm}
  \begin{columns}
    \column{0.50\textwidth}
      % cfg-unequal-number-a-b.tex

\begin{empheq}[box=\widefbox]{align*}
  S &\to T \mid U \\[8pt]
  T &\to VaT \mid VaV \\[8pt]
  U &\to VbU \mid VbV \\[8pt]
  V &\to aVbV \mid bVaV \mid \epsilon
\end{empheq}
    \column{0.50\textwidth}
      \pause
      \fig{width = 0.70\textwidth}{figs/keep-calm-prove-it}
      \begin{center}
        \blue{\bf 练习 (非作业):} 证明之
      \end{center}
  \end{columns}
\end{frame}
%%%%%%%%%%%%%%%%%%%%

%%%%%%%%%%%%%%%%%%%%
\begin{frame}{}
  \fig{width = 0.60\textwidth}{figs/sth-interesting}

  \vspace{0.30cm}
  \begin{center}
    \href{https://en.wikipedia.org/wiki/L-system}{\teal{$L$-System}} \\[6pt]
    (\blue{注: 这不是上下文无关文法, 但精神上高度一致, 并且更有趣})
  \end{center}
\end{frame}
%%%%%%%%%%%%%%%%%%%%

%%%%%%%%%%%%%%%%%%%%
\begin{frame}{}
  \[
    g + g + f − g − g
  \]

  \vspace{0.60cm}
  \begin{center}
    $g:$ Move forward with the pen down \\[15pt]
    $f:$ Move forward with the pen up \\[15pt]
    $+:$ Turn to its right \\[15pt]
    $-:$ Turn to its left
  \end{center}
\end{frame}
%%%%%%%%%%%%%%%%%%%%

%%%%%%%%%%%%%%%%%%%%
\begin{frame}{}
  \fig{width = 0.60\textwidth}{figs/jflap-failed-tree}
\end{frame}
%%%%%%%%%%%%%%%%%%%%

%%%%%%%%%%%%%%%%%%%%
\begin{frame}{}
  \begin{columns}
    \column{0.50\textwidth}
      \fig{width = 0.95\textwidth}{figs/jflap-fixed-tree-rules}
      \begin{center}
        Pushing and Popping
      \end{center}
    \column{0.50\textwidth}
      \pause
      \fig{width = 0.95\textwidth}{figs/jflap-fixed-tree}
  \end{columns}
\end{frame}
%%%%%%%%%%%%%%%%%%%%

%%%%%%%%%%%%%%%%%%%%
\begin{frame}{}
  \begin{columns}
    \column{0.50\textwidth}
      \fig{width = 0.90\textwidth}{figs/jflap-tree-growing}
    \column{0.50\textwidth}
      \pause
      \fig{width = 0.90\textwidth}{figs/jflap-tree-growing-rules}
  \end{columns}
\end{frame}
%%%%%%%%%%%%%%%%%%%%

%%%%%%%%%%%%%%%%%%%%
\begin{frame}{}
  \fig{width = 0.70\textwidth}{figs/jflap-dragon-rules}
\end{frame}
%%%%%%%%%%%%%%%%%%%%

%%%%%%%%%%%%%%%%%%%%
\begin{frame}{}
  \fig{width = 0.60\textwidth}{figs/jflap-dragon}
  \begin{center}
    \href{https://en.wikipedia.org/wiki/L-system\#Example_6:_Dragon_curve}{The Dragon Curve}
  \end{center}
\end{frame}
%%%%%%%%%%%%%%%%%%%%

%%%%%%%%%%%%%%%%%%%%
\begin{frame}{}
  \fig{width = 0.50\textwidth}{figs/S-curve-grammar}

  \vspace{0.20cm}
  \begin{center}
    $A, B:$ \blue{向右}移动并画线 \\[12pt]
    $+:$ 左转 \\[12pt]
    $-:$ 右转

    \vspace{0.60cm}
    每一步都\red{\bf 并行地}应用\red{\bf 所有}规则
  \end{center}
\end{frame}
%%%%%%%%%%%%%%%%%%%%

%%%%%%%%%%%%%%%%%%%%
\begin{frame}{}
  \begin{center}
    \[
      A
    \]

    \[
      \blue{B} - \purple{A} - \teal{B}
    \]

    \pause
    \[
      \blue{A + B + A} - \purple{B - A - B} - \teal{A + B + A}
    \]

    \pause
    \fig{width = 0.40\textwidth}{figs/S-curve-2}
  \end{center}
\end{frame}
%%%%%%%%%%%%%%%%%%%%

%%%%%%%%%%%%%%%%%%%%
\begin{frame}{}
  \fig{width = 0.60\textwidth}{figs/S-curve}
  \begin{center}
    \teal{Sierpinski arrowhead curve ($n = 2, 4, 6, 8$)}
  \end{center}
\end{frame}
%%%%%%%%%%%%%%%%%%%%

%%%%%%%%%%%%%%%%%%%%
\begin{frame}{}
  \fig{width = 0.40\textwidth}{figs/dragon-curve-grammar}

  \vspace{0.20cm}
  \begin{center}
    $F, G:$ \blue{向前}移动并画线 \\[12pt]
    $+:$ 左转 \\[12pt]
    $-:$ 右转 \\[12pt]

    \vspace{0.20cm}
    每一步都\red{\bf 并行地}应用\red{\bf 所有}规则
  \end{center}
\end{frame}
%%%%%%%%%%%%%%%%%%%%

%%%%%%%%%%%%%%%%%%%%
\begin{frame}{}
  \fig{width = 0.70\textwidth}{figs/dragon-curve}
  \begin{center}
    \href{https://en.wikipedia.org/wiki/Dragon\_curve}{\teal{Dragon Curve} ($n = 10$)}
  \end{center}
\end{frame}
%%%%%%%%%%%%%%%%%%%%

%%%%%%%%%%%%%%%%%%%%
\begin{frame}{}
  \begin{center}
    为什么不使用优雅、强大的\blue{\bf 正则表达式}描述程序设计语言的语法?

    \vspace{0.50cm}
    \fig{width = 0.40\textwidth}{figs/grammar-hierarchy}
    正则表达式的表达能力\red{\bf 严格弱于}上下文无关文法
  \end{center}
\end{frame}
%%%%%%%%%%%%%%%%%%%%

%%%%%%%%%%%%%%%%%%%%
\begin{frame}{}
  \begin{center}
    每个\blue{正则表达式} $r$ 对应的语言 $L(r)$ 都可以使用\blue{上下文无关文法}来描述

    \[
      r = (a | b)^{\ast} abb
    \]

    \pause
    \fig{width = 0.50\textwidth}{figs/nfa-abb}

    \pause
    \fig{width = 0.40\textwidth}{figs/nfa-abb-cfg}
    此外, 若 $\delta(A_i, \epsilon) = A_{j}$, 则添加 $A_{i} \to A_{j}$
  \end{center}
\end{frame}
%%%%%%%%%%%%%%%%%%%%

%%%%%%%%%%%%%%%%%%%%
\begin{frame}{}
  \begin{center}
    % cfg-anbn.tex

\begin{empheq}[box=\widefbox]{align*}
  S &\to aSb \\[8pt]
  S &\to \epsilon
\end{empheq}

    \[
      L = \set{a^{n} b^{n} \mid n \ge 0}
    \]
    该语言\red{\bf 无法}使用正则表达式来描述
  \end{center}
\end{frame}
%%%%%%%%%%%%%%%%%%%%

%%%%%%%%%%%%%%%%%%%%
\begin{frame}{}
  \begin{theorem}
    $L = \set{a^{n} b^{n} \mid n \ge 0}$ 无法使用正则表达式描述。
  \end{theorem}

  \pause
  \begin{center}
    \red{\bf 反证法}

    \pause
    \vspace{0.30cm}
    假设存在正则表达式 $r$: $L(r) = L$

    \pause
    \vspace{0.30cm}
    则存在\blue{\bf 有限}状态自动机 $D(r)$: $L(D(r)) = L$; 设其状态数为 $k$

    \pause
    \vspace{0.30cm}
    \fbox{\red{考虑输入 $a^{m} (m > k)$}}
    \fig{width = 0.90\textwidth}{figs/pumping-lemma-anbn}

    \pause
    \vspace{0.30cm}
    $D(r)$ 也能接受 $a^{i+j} b^{i}$; \red{\bf 矛盾!}
  \end{center}
\end{frame}
%%%%%%%%%%%%%%%%%%%%

%%%%%%%%%%%%%%%%%%%%
\begin{frame}{}
  \begin{center}
    \[
      L = \set{a^{n}b^{n} \mid n \ge 0}
    \]
    \href{https://en.wikipedia.org/wiki/Pumping\_lemma\_for\_regular\_languages}{
      \teal{Pumping Lemma for \red{Regular Languages}}}

    \pause
    \vspace{0.60cm}
    \[
      L = \set{a^{n}b^{n}c^{n} \mid n \ge 0}
    \]
    \href{https://en.wikipedia.org/wiki/Pumping\_lemma\_for\_context-free\_languages}{
      \teal{Pumping Lemma for \red{Context-free Languages}}}
  \end{center}
\end{frame}
%%%%%%%%%%%%%%%%%%%%