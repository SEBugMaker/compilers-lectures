% min-dfa.tex

%%%%%%%%%%%%%%%%%%%%
\begin{frame}{}
  \fig{width = 0.70\textwidth}{figs/re-dfa-lexer}

  \vspace{0.30cm}
  \begin{center}
    DFA 最小化
  \end{center}
\end{frame}
%%%%%%%%%%%%%%%%%%%%

%%%%%%%%%%%%%%%%%%%%
\begin{frame}{}
  \fig{width = 0.50\textwidth}{figs/dfa-abb}

  \[
    \blue{L(\mathcal{A}) = L((a|b)^{\ast}abb)}
  \]

  \fig{width = 0.50\textwidth}{figs/dfa-abb-subset}

  \begin{center}
    子集构造法得到的 DFA
  \end{center}
\end{frame}
%%%%%%%%%%%%%%%%%%%%

%%%%%%%%%%%%%%%%%%%%
\begin{frame}{}
  \begin{center}
    \red{\bf DFA 最小化算法}基本思想: \blue{\bf 等价}的状态可以合并

    \vspace{0.30cm}
    \fig{width = 0.30\textwidth}{figs/Hopcroft}
    \teal{John Hopcroft (1939 $\sim$)}

    \vspace{0.30cm}
    \begin{quote}
      \centering
      ``With \blue{Robert E. Tarjan}, for fundamental achievements in the design and analysis
      of \red{\emph{\textbf{algorithms and data structures}}}'' \\
      \hfill --- Turing Award, 1986
    \end{quote}
  \end{center}
\end{frame}
%%%%%%%%%%%%%%%%%%%%

%%%%%%%%%%%%%%%%%%%%
\begin{frame}{}
  \begin{center}
    如何定义\red{\bf 等价}状态?

    \fig{width = 0.50\textwidth}{figs/dfa-abb-subset}

    \[
      s \sim t \iff \forall a \in \Sigma.\;
        \left((s \rel{a} s') \land (t \rel{a} t')\right) \implies (\red{s' = t'}).
    \]

    \pause
    但是, 这个定义是错误的
  \end{center}
\end{frame}
%%%%%%%%%%%%%%%%%%%%

%%%%%%%%%%%%%%%%%%%%
\begin{frame}{}
  \[
    s \sim t \iff \forall a \in \Sigma.\;
      \left((s \rel{a} s') \land (t \rel{a} t')\right) \implies (\red{s' = t'}).
  \]

  \fig{width = 0.50\textwidth}{figs/dfa-abb-subset}

  \vspace{-0.30cm}
  \[
    A \sim C \sim E
  \]

  \pause
  \vspace{0.20cm}
  \begin{center}
    但是, 接受状态与非接受状态必定不等价

    \vspace{0.30cm}
    空串 $\epsilon$ \red{\bf 区分}了这两类状态
  \end{center}
\end{frame}
%%%%%%%%%%%%%%%%%%%%

%%%%%%%%%%%%%%%%%%%%
\begin{frame}{}
  \[
    s \sim t \iff \forall a \in \Sigma.\;
      \left((s \rel{a} s') \land (t \rel{a} t')\right) \implies (\red{s' = t'}).
  \]
  \fig{width = 0.60\textwidth}{figs/min-dfa-wiki-before}
  \[
    c \sim d \sim e \qquad \red{a \nsim b}
  \]
  \fig{width = 0.60\textwidth}{figs/min-dfa-wiki-after}
\end{frame}
%%%%%%%%%%%%%%%%%%%%

%%%%%%%%%%%%%%%%%%%%
\begin{frame}{}
  \begin{center}
    如何定义\red{\bf 等价}状态?
  \end{center}

  \fig{width = 0.50\textwidth}{figs/dfa-abb-subset}

  \[
    s \sim t \iff \forall a \in \Sigma.\;
      \left((s \rel{a} s') \land (t \rel{a} t')\right) \implies (\red{s' \sim t'}).
  \]

  \pause
  \[
    s \nsim t \iff \exists a \in \Sigma.\;
      (s \rel{a} s') \land (t \rel{a} t') \land (\red{s' \nsim t'})
  \]
\end{frame}
%%%%%%%%%%%%%%%%%%%%

%%%%%%%%%%%%%%%%%%%%
\begin{frame}{}
  \begin{center}
    \[
      s \sim t \iff \forall a \in \Sigma.\;
        \left((s \rel{a} s') \land (t \rel{a} t')\right) \implies (s' \sim t').
    \]

    基于该定义, 不断\red{\bf 合并}等价的状态, 直到无法合并为止

    \pause
    \vspace{0.80cm}
    \red{\bf 但是, 这是一个递归定义, 从哪里开始呢?}

    \pause
    \vspace{0.80cm}
    \blue{$Q:$ 所有接受状态都是等价的吗?}

    \pause
    \vspace{0.50cm}
    \fig{width = 0.60\textwidth}{figs/a-ab}
  \end{center}
\end{frame}
%%%%%%%%%%%%%%%%%%%%

%%%%%%%%%%%%%%%%%%%%
\begin{frame}{}
  \begin{center}
    \[
      s \sim t \iff \forall a \in \Sigma.\;
        \left((s \rel{a} s') \land (t \rel{a} t')\right) \implies (s' \sim t').
    \]

    缺少基础情况, 不知从何下手

    \pause
    \[
      s \nsim t \iff \exists a \in \Sigma.\;
        (s \rel{a} s') \land (t \rel{a} t') \land (s' \nsim t')
    \]

    \pause
    \red{\bf 划分, 而非合并}
  \end{center}

\end{frame}
%%%%%%%%%%%%%%%%%%%%

%%%%%%%%%%%%%%%%%%%%
\begin{frame}{}
  \begin{center}
    \[
      s \nsim t \iff \exists a \in \Sigma.\;
        (s \rel{a} s') \land (t \rel{a} t') \land (s' \nsim t')
    \]

    \pause
    \vspace{0.30cm}
    \red{\bf 从哪里开始呢?}
  \end{center}

  \pause
  \begin{columns}
    \column{0.50\textwidth}
      \fig{width = 1.00\textwidth}{figs/dfa-abb-subset}
    \column{0.50\textwidth}
      \[
        \Pi = \set{F, S \setminus F}
      \]

      \vspace{0.30cm}
      \begin{center}
        接受状态与非接受状态必定不等价

        \pause
        \vspace{0.50cm}
        空串 $\epsilon$ {\bf 区分}了这两类状态
      \end{center}
  \end{columns}
\end{frame}
%%%%%%%%%%%%%%%%%%%%

%%%%%%%%%%%%%%%%%%%%
\begin{frame}{}
  \begin{center}
    \fig{width = 0.40\textwidth}{figs/blackboard}

    \vspace{0.30cm}
    板书演示算法过程
  \end{center}
\end{frame}
%%%%%%%%%%%%%%%%%%%%

%%%%%%%%%%%%%%%%%%%%
\begin{frame}{}
  \begin{align*}
    \Pi_{0} &= \set{\set{A, B, C, D}, \set{E}} \\[10pt]
    \Pi_{1} &= \set{\set{A, B, C}, \set{D}, \set{E}} \\[10pt]
    \Pi_{2} &= \set{\set{A, C}, \set{B}, \set{D}, \set{E}} \\[10pt]
    \Pi_{3} &= \Pi_{2}
  \end{align*}
\end{frame}
%%%%%%%%%%%%%%%%%%%%

%%%%%%%%%%%%%%%%%%%%
\begin{frame}{}
  \fig{width = 0.50\textwidth}{figs/dfa-abb-subset}

  \begin{center}
    \blue{\bf 合并}等价状态 $A \sim C$
  \end{center}

  \fig{width = 0.50\textwidth}{figs/dfa-abb}

  \pause
  \begin{center}
    \red{$Q:$ 合并后是否一定还是 DFA?}
  \end{center}
\end{frame}
%%%%%%%%%%%%%%%%%%%%

%%%%%%%%%%%%%%%%%%%%
\begin{frame}{}
  \begin{center}
    DFA 最小化\red{\bf 等价状态划分}方法
  \end{center}

  \[
    \Pi = \set{F, S \setminus F}
  \]

  \fig{width = 0.85\textwidth}{figs/partition}

  \begin{center}
    直到再也无法\blue{\bf 划分}为止 \quad (\purple{不动点!})

    \vspace{0.40cm}
    然后, 将同一等价类里的状态\blue{\bf 合并}
  \end{center}
\end{frame}
%%%%%%%%%%%%%%%%%%%%

%%%%%%%%%%%%%%%%%%%%
\begin{frame}{}
  \begin{center}
    如何分析DFA最小化算法的\red{\bf 复杂度}?

    \pause
    \vspace{0.60cm}
    为什么 DFA 最小化算法是\red{\bf 正确}的?

    \pause
    \vspace{0.60cm}
    最小化DFA是\red{\bf 唯一}的吗?

    \pause
    \vspace{0.30cm}
    \fig{width = 0.50\textwidth}{figs/report}
    \teal{可选报告 (1)}
  \end{center}
\end{frame}
%%%%%%%%%%%%%%%%%%%%

%%%%%%%%%%%%%%%%%%%%
% \begin{frame}{}
%   \begin{center}
%     如何定义\red{\bf 等价}状态?
%
%   \end{center}
%
%   \pause
%   \begin{definition}[等价状态]
%     两个状态是等价的当且仅当它们对任何相同输入串的行为是相同的。
%   \end{definition}
% \end{frame}
%%%%%%%%%%%%%%%%%%%%
