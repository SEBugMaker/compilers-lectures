% overview.tex

%%%%%%%%%%%%%%%%%%%%
\begin{frame}{}
  \fig{width = 0.60\textwidth}{figs/dfa-lexer}
  \begin{center}
    自动化词法分析器
  \end{center}
\end{frame}
%%%%%%%%%%%%%%%%%%%%

%%%%%%%%%%%%%%%%%%%%
\begin{frame}{}
  \begin{center}
    \blue{\large 自动机} \\[5pt]
    (Automaton; Automata)

    \fig{width = 0.50\textwidth}{figs/off-on}
    \vspace{-0.30cm}
    ``开关''自动机

    \vspace{0.80cm}
    两大要素: {\bf 状态集} $S$ 以及{\bf 状态转移函数} $\delta$
  \end{center}
\end{frame}
%%%%%%%%%%%%%%%%%%%%

%%%%%%%%%%%%%%%%%%%%
\begin{frame}{}
  \begin{center}
    \fig{width = 0.60\textwidth}{figs/automata-theory-wiki}

    \vspace{0.50cm}
    根据\red{\bf 表达/计算能力}的强弱, 自动机可以分为不同层次。
  \end{center}
\end{frame}
%%%%%%%%%%%%%%%%%%%%

%%%%%%%%%%%%%%%%%%%%
\begin{frame}{}
  \begin{center}
    元胞自动机 (Cellular Automaton)
  \end{center}

  \vspace{0.30cm}
  \begin{columns}
    \column{0.60\textwidth}
      \fig{width = 1.00\textwidth}{figs/breeder}
      \begin{center}
        ``播种机''、``滑翔机枪''与``滑翔机'' \\[5pt]
      \end{center}
    \column{0.40\textwidth}
      \fig{width = 0.90\textwidth}{figs/John-Conway}
      \begin{center}
        \teal{John Horton Conway} \\[3pt]
        \teal{(1937 $\sim$ 2020)}
      \end{center}
  \end{columns}

  \begin{center}
    \teal{\url{https://en.wikipedia.org/wiki/File:Conways_game_of_life_breeder_animation.gif}}
  \end{center}
\end{frame}
%%%%%%%%%%%%%%%%%%%%

%%%%%%%%%%%%%%%%%%%%
\begin{frame}{}
  \begin{center}
    \fig{width = 0.40\textwidth}{figs/video}

    \vspace{0.80cm}
    ``生命游戏''(Game of Life)史诗级巨作 \\[8pt]
    \teal{\url{https://www.youtube.com/watch?v=C2vgICfQawE&t=270s}}
  \end{center}
\end{frame}
%%%%%%%%%%%%%%%%%%%%

%%%%%%%%%%%%%%%%%%%%
\begin{frame}{}
  \begin{center}
    \red{\bf 目标: 正则表达式 RE $\implies$ 词法分析器}

    \vspace{0.30cm}
    \fig{width = 0.70\textwidth}{figs/re-dfa-lexer}

    \vspace{0.50cm}
    \purple{终点固然令人向往, 这一路上的风景更是美不胜收}
  \end{center}
\end{frame}
%%%%%%%%%%%%%%%%%%%%