% error.tex

%%%%%%%%%%%%%%%%%%%%
\begin{frame}{}
  \begin{center}
    语法分析阶段的主题之三: \red{\bf 错误恢复}

    \fig{width = 0.45\textwidth}{figs/keep-calm-recovery}

    报错、\blue{\bf 恢复}、继续分析
  \end{center}
\end{frame}
%%%%%%%%%%%%%%%%%%%%

%%%%%%%%%%%%%%%%%%%%
\begin{frame}{}
  \fig{width = 0.50\textwidth}{figs/panic}

  \vspace{0.30cm}
  \begin{center}
    \blue{\bf 恐慌(Panic)模式:} 丢弃输入、调整状态、假装成功
  \end{center}
\end{frame}
%%%%%%%%%%%%%%%%%%%%

%%%%%%%%%%%%%%%%%%%%
\begin{frame}{}
  \begin{center}
    \blue{\bf 分号}作为\blue{\bf 语句}分隔符, 可用作\red{\bf 同步单词} (Synchronizing Word)

    \vspace{0.80cm}
    \begin{description}
      \setlength{\itemsep}{15pt}
      \item[丢弃输入:] 不断调用词法分析器, 直到找到下一个分号
      \item[调整状态:] 不断出栈, 直到找到一个状态$s$满足
        \[
          \goto[s, Stmt] \neq \teal{\textsc{error}}
        \]
      \item[假装成功:] 将状态 $\goto[s, Stmt]$ 压栈, 恢复语法分析过程
    \end{description}

    \pause
    \vspace{0.80cm}
    终结符 \blue{\bf $a$} 作为非终结符 \blue{\bf $A$} 的\red{\bf 同步单词} (如, $a \in \follow(A)$)

    \pause
    \vspace{0.50cm}
    可为\purple{\bf 多个}非终结符 $A$ 设置相应的同步单词 $a$
  \end{center}
\end{frame}
%%%%%%%%%%%%%%%%%%%%