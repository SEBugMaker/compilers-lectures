% type-system.tex

%%%%%%%%%%%%%%%%%%%%
\begin{frame}{}
  \begin{center}
    \begin{definition}[符号表 (Symbol Table)]
      \red{\bf 符号表}是用于保存各种信息的\blue{\bf 数据结构}。
    \end{definition}

    \vspace{0.50cm}
    \purple{\bf 标识符:} 词素、类型、大小、存储位置等

    \pause
    \fig{width = 0.90\textwidth}{figs/st-api}
    \teal{通常使用 \texttt{HashTable}}
  \end{center}
\end{frame}
%%%%%%%%%%%%%%%%%%%%

%%%%%%%%%%%%%%%%%%%%
\begin{frame}{}
  \begin{center}
    为每个\red{\bf 作用域}建立单独的符号表

    \fig{width = 0.75\textwidth}{figs/nested-symbol-tables}

    可以使用符号表\blue{\bf 栈}实现\purple{\bf 树形的}嵌套关系
  \end{center}
\end{frame}
%%%%%%%%%%%%%%%%%%%%

%%%%%%%%%%%%%%%%%%%%
\begin{frame}{}
  \begin{center}
    \fig{width = 0.80\textwidth}{figs/Env}
  \end{center}
\end{frame}
%%%%%%%%%%%%%%%%%%%%

%%%%%%%%%%%%%%%%%%%%
\begin{frame}{}
  \begin{center}
    \fig{width = 1.00\textwidth}{figs/xy-code}

    \pause
    \vspace{0.80cm}
    \red{\bf 翻译任务:} 忽略\blue{\bf 声明}部分, 为每个标识符标明\blue{\bf 类型}

    \vspace{0.80cm}
    \fig{width = 1.00\textwidth}{figs/xy-type}
  \end{center}
\end{frame}
%%%%%%%%%%%%%%%%%%%%

%%%%%%%%%%%%%%%%%%%%
\begin{frame}{}
  \begin{center}
    \begin{definition}[类型表达式 (Type Expressions)]
      \begin{itemize}
        \setlength{\itemsep}{12pt}
        \item \blue{\bf 基本类型}是类型表达式;
          \begin{itemize}
            \item char, bool, int, float, double, void, $\dots$
          \end{itemize}
        \item \blue{\bf 类名}是类型表达式;
        \pause
        \item 如果 $t$ 是类型表达式, 则 \red{\bf array(num, t)} 是类型表达式;
        \item \red{\bf record($\langle \text{id : t}, \dots \rangle$)} 是类型表达式;
        \pause
        \item 如果 $s$ 和 $t$ 是类型表达式, 则 \purple{$s \times t$} 是类型表达式;
        \item 如果 $s$ 和 $t$ 是类型表达式, 则 \purple{$s \to t$} 是类型表达式;
        \pause
        \item \gray{\bf 类型表达式可以包含取值为类型表达式的变量。}
      \end{itemize}
    \end{definition}
  \end{center}
\end{frame}
%%%%%%%%%%%%%%%%%%%%

%%%%%%%%%%%%%%%%%%%%
\begin{frame}{}
  \begin{center}
    \fig{width = 0.60\textwidth}{figs/cfg-block}
  \end{center}
\end{frame}
%%%%%%%%%%%%%%%%%%%%

%%%%%%%%%%%%%%%%%%%%
\begin{frame}{}
  \begin{center}
    \texttt{\{ char y; \{ bool y; y; \} y; \}}

    \vspace{0.80cm}
    \texttt{\{ \{ y : bool; \} y:char; \}}
  \end{center}
\end{frame}
%%%%%%%%%%%%%%%%%%%%

%%%%%%%%%%%%%%%%%%%%
\begin{frame}{}
  \begin{center}
    \fig{width = 0.60\textwidth}{figs/SDT-block}
  \end{center}
\end{frame}
%%%%%%%%%%%%%%%%%%%%

%%%%%%%%%%%%%%%%%%%%
\begin{frame}{}
  \begin{center}
    \red{\bf 类型声明}

    \fig{width = 0.80\textwidth}{figs/decl-code}

    \vspace{0.60cm}
    \blue{\bf \purple{\bf 符号表}中记录标识符的类型、宽度 (width)、偏移地址 (offset)}
  \end{center}
\end{frame}
%%%%%%%%%%%%%%%%%%%%

%%%%%%%%%%%%%%%%%%%%
\begin{frame}{}
  \begin{center}
    \fig{width = 0.80\textwidth}{figs/cfg-decl}

    \vspace{0.80cm}
    需要为每个 \red{\bf record} 生成单独的符号表
  \end{center}
\end{frame}
%%%%%%%%%%%%%%%%%%%%

%%%%%%%%%%%%%%%%%%%%
\begin{frame}{}
  \begin{center}
    \red{\bf 全局变量 $t$ 与 $w$} 将 $B$ 的\blue{\bf 类型与宽度}信息递给产生式 \purple{\bf $C \to \epsilon$}

    \vspace{0.30cm}
    \fig{width = 0.90\textwidth}{figs/SDT-type}

    \vspace{-0.80cm}
    \[
      \teal{\texttt{float x}}
    \]
  \end{center}
\end{frame}
%%%%%%%%%%%%%%%%%%%%

%%%%%%%%%%%%%%%%%%%%
\begin{frame}{}
  \begin{center}
    \fig{width = 0.95\textwidth}{figs/anno-type-width}
    \vspace{-0.50cm}
    \[
      \teal{\texttt{int[2][3]}}
    \]
  \end{center}
\end{frame}
%%%%%%%%%%%%%%%%%%%%

%%%%%%%%%%%%%%%%%%%%
\begin{frame}{}
  \begin{center}
    \red{\bf 全局变量 offset 表示变量的相对地址}

    \vspace{0.20cm}
    \blue{\bf 全局变量 top 表示当前的符号表}

    \fig{width = 0.95\textwidth}{figs/SDT-offset}

    \vspace{-0.80cm}
    \[
      \teal{\texttt{float x; float y;}}
    \]
  \end{center}
\end{frame}
%%%%%%%%%%%%%%%%%%%%

%%%%%%%%%%%%%%%%%%%%
\begin{frame}{}
  \begin{center}
    \fig{width = 0.95\textwidth}{figs/SDT-record}

    \pause
    \vspace{0.60cm}
    \red{\bf record 类型表达式: \texttt{record(top)}}

    \pause
    \vspace{0.30cm}
    \blue{\bf 全局变量 top 表示当前的符号表}

    \pause
    \vspace{0.30cm}
    \green{\bf 全局变量 Env 表示符号表栈}

    \pause
    \vspace{-0.60cm}
    \[
      \teal{\texttt{record \{ float x; float y; \} p;}}
    \]
  \end{center}
\end{frame}
%%%%%%%%%%%%%%%%%%%%

%%%%%%%%%%%%%%%%%%%%
\begin{frame}{}
  \begin{center}
    \fig{width = 0.90\textwidth}{figs/decl-code}
  \end{center}
\end{frame}
%%%%%%%%%%%%%%%%%%%%

%%%%%%%%%%%%%%%%%%%%
\begin{frame}{}
  \begin{center}
    \fig{width = 0.80\textwidth}{figs/dynamic-static-weak-strong}

    \vspace{0.30cm}
    \teal{\url{https://youtu.be/C5fr0LZLMAs}}
  \end{center}
\end{frame}
%%%%%%%%%%%%%%%%%%%%

%%%%%%%%%%%%%%%%%%%%
\begin{frame}{}
  \begin{center}
    \red{\bf 类型检查}的常见形式

    \fig{width = 0.80\textwidth}{figs/type-checking-rule}
  \end{center}
\end{frame}
%%%%%%%%%%%%%%%%%%%%

%%%%%%%%%%%%%%%%%%%%
\begin{frame}{}
  \begin{center}
    \fig{width = 0.50\textwidth}{figs/struct-stack-set}
  \end{center}
\end{frame}
%%%%%%%%%%%%%%%%%%%%

%%%%%%%%%%%%%%%%%%%%
\begin{frame}{}
  \begin{center}
    \begin{definition}[结构等价 (Structurally Equivalent)]
      两种类型\red{\bf 结构等价}当且仅当以下任一条件为真:
      \begin{itemize}
        \item 它们是相同的基本类型;
        \item 它们是将相同的类型构造算子应用于结构等价的类型而构造得到;
        \item \purple{\bf 一个类型是另一个类型表达式的名字}。
      \end{itemize}
    \end{definition}

    \pause
    \vspace{0.80cm}
    \begin{definition}[名等价 (Name Equivalent)]
      两种类型\blue{\bf 名等价}当且仅当以下任一条件为真:
      \begin{itemize}
        \item 它们是相同的基本类型;
        \item 它们是将相同的类型构造算子应用于结构等价的类型而构造得到。
      \end{itemize}
    \end{definition}
  \end{center}
\end{frame}
%%%%%%%%%%%%%%%%%%%%

%%%%%%%%%%%%%%%%%%%%
\begin{frame}{}
  \begin{center}
    \red{\bf 结构等价}中的``结构''又是什么意思?

    \vspace{0.60cm}
    $array(\blue{n}, t)  \qquad array(\blue{m}, t)$

    \pause
    \vspace{0.80cm}
    % record($\langle \text{\red{a} : int} \rangle$) \qquad
    % record($\langle \text{\red{b} : int} \rangle$)
    \begin{columns}
      \column{0.50\textwidth}
        \fig{width = 0.80\textwidth}{figs/record-ab}
      \column{0.50\textwidth}
        \fig{width = 0.80\textwidth}{figs/record-ba}
    \end{columns}

    \pause
    \vspace{1.00cm}
    \blue{\bf 不同的语言有不同的设计方案}
  \end{center}
\end{frame}
%%%%%%%%%%%%%%%%%%%%

%%%%%%%%%%%%%%%%%%%%
\begin{frame}{}
  \begin{center}
    \blue{\bf 类型综合:} 根据子表达式的类型确定表达式的类型

    \vspace{0.60cm}
    \fig{width = 0.80\textwidth}{figs/type-synthesis}

    \[
        \teal{E_{1} + E_{2}}
    \]
  \end{center}
\end{frame}
%%%%%%%%%%%%%%%%%%%%

%%%%%%%%%%%%%%%%%%%%
\begin{frame}{}
  \begin{center}
    \fig{width = 0.60\textwidth}{figs/error-functions}

    \vspace{0.60cm}
    \blue{\bf 重载函数的类型综合规则}

    \vspace{0.60cm}
    \fig{width = 0.90\textwidth}{figs/function-type-synthesis}
  \end{center}
\end{frame}
%%%%%%%%%%%%%%%%%%%%

%%%%%%%%%%%%%%%%%%%%
\begin{frame}{}
  \begin{center}
    \blue{\bf 类型推导:} 根据某语言结构的使用方式确定表达式的类型

    \vspace{0.50cm}
    \fig{width = 1.00\textwidth}{figs/type-inference}

    \[
        \teal{null(x): \text{$x$ 是一个列表, 它的元素类型未知}}
    \]
  \end{center}
\end{frame}
%%%%%%%%%%%%%%%%%%%%

%%%%%%%%%%%%%%%%%%%%
\begin{frame}{}
  \begin{center}
    \red{\bf 类型转换}

    \fig{width = 0.40\textwidth}{figs/int-to-float}

    \pause
    \vspace{0.50cm}
    \fig{width = 1.00\textwidth}{figs/type-int-float}
    \pause
    \vspace{-0.50cm}
    \red{\bf 不要写这样的代码!!!}
  \end{center}
\end{frame}
%%%%%%%%%%%%%%%%%%%%

%%%%%%%%%%%%%%%%%%%%
\begin{frame}{}
  \begin{center}
    \fig{width = 0.80\textwidth}{figs/widen-narrow}
  \end{center}
\end{frame}
%%%%%%%%%%%%%%%%%%%%

%%%%%%%%%%%%%%%%%%%%
\begin{frame}{}
  \begin{center}
    \fig{width = 0.80\textwidth}{figs/expr-types}
    \fig{width = 0.70\textwidth}{figs/widen}
  \end{center}
\end{frame}
%%%%%%%%%%%%%%%%%%%%

%%%%%%%%%%%%%%%%%%%%
\begin{frame}{}
  \begin{center}
    \fig{width = 0.90\textwidth}{figs/type-systems-good-bad-ugly}

    \vspace{0.30cm}
    \teal{\url{https://youtu.be/SWTWkYbcWU0}}
  \end{center}
\end{frame}
%%%%%%%%%%%%%%%%%%%%