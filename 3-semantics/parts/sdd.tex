% sdd.tex

%%%%%%%%%%%%%%%%%%%%
\begin{frame}{}
  \begin{center}
    \begin{definition}[语法制导定义 (Syntax-Directed Definition; SDD)]
      SDD 是一个上下文无关文法和\red{\bf 属性}及\blue{\bf 规则}的结合。
    \end{definition}

    \vspace{0.50cm}
    \red{每个文法符号都可以关联多个属性}

    \fig{width = 0.60\textwidth}{figs/SDD-expr-left-recursion}

    \blue{每个产生式都可以关联一组规则}
  \end{center}
\end{frame}
%%%%%%%%%%%%%%%%%%%%

%%%%%%%%%%%%%%%%%%%%
\begin{frame}{}
  \begin{center}
    \begin{definition}[语法制导定义 (Syntax-Directed Definition; SDD)]
      SDD 是一个上下文无关文法和\red{\bf 属性}及\blue{\bf 规则}的结合。
    \end{definition}

    \vspace{0.50cm}
    SDD \red{\bf 唯一确定}了语法分析树上每个非终结符节点的属性值

    \fig{width = 0.60\textwidth}{figs/SDD-expr-left-recursion}

    SDD \red{\bf 没有}规定以什么方式、什么顺序计算这些属性值
  \end{center}
\end{frame}
%%%%%%%%%%%%%%%%%%%%

%%%%%%%%%%%%%%%%%%%%
\begin{frame}{}
  \begin{center}
    \red{\bf 注释(annotated)语法分析树:} 显示了各个属性值的语法分析树

    \fig{width = 0.60\textwidth}{figs/anno-expr-left-recursion}
    \[
        \teal{3 \ast 5 + 4}
    \]

    % \begin{columns}
    %   \column{0.45\textwidth}
    %     \fig{width = 0.90\textwidth}{figs/SDD-expr-left-recursion}
    %   \column{0.55\textwidth}
    %     \fig{width = 1.00\textwidth}{figs/anno-expr-left-recursion}
    %     \[
    %       \teal{3 \ast 5 + 4}
    %     \]
    % \end{columns}
  \end{center}
\end{frame}
%%%%%%%%%%%%%%%%%%%%

%%%%%%%%%%%%%%%%%%%%
\begin{frame}{}
  \begin{center}
    \begin{definition}[综合属性 (Synthesized Attribute)]
      节点$N$上的\red{\bf 综合属性}只能通过
      \blue{$N$的子节点或$N$本身}的属性来定义。
    \end{definition}

    \vspace{0.50cm}
    \fig{width = 0.60\textwidth}{figs/SDD-expr-left-recursion}

    \pause
    \vspace{0.50cm}
    \begin{definition}[$S$ 属性定义 ($S$-Attributed Definition)]
      如果一个SDD的每个属性都是综合属性, 则它是 $S$ 属性定义。
    \end{definition}
  \end{center}
\end{frame}
%%%%%%%%%%%%%%%%%%%%

%%%%%%%%%%%%%%%%%%%%
\begin{frame}{}
  \begin{center}
    \red{\bf 依赖图}用于确定一棵\blue{给定的语法分析树}中\purple{各个属性实例}之间的依赖关系

    \begin{columns}
      \column{0.45\textwidth}
        \fig{width = 0.90\textwidth}{figs/SDD-expr-left-recursion}

        \fig{width = 0.90\textwidth}{figs/dep-expr-left-recursion}
      \column{0.55\textwidth}
        \fig{width = 1.00\textwidth}{figs/anno-expr-left-recursion}
    \end{columns}

    \pause
    \vspace{0.80cm}
    \teal{\bf $S$ 属性定义}的依赖图描述了属性实例之间\teal{\bf 自底向上}的信息流
  \end{center}
\end{frame}
%%%%%%%%%%%%%%%%%%%%

%%%%%%%%%%%%%%%%%%%%
\begin{frame}{}
  \begin{center}
    \teal{\bf $S$ 属性定义}的依赖图描述了属性实例之间\teal{\bf 自底向上}的信息流

    \vspace{0.80cm}
    因此, 此类属性值的计算可以自然地在\purple{自底向上}的语法分析\red{\bf 过程中}实现

    \pause
    \vspace{0.80cm}
    当$LR$语法分析器进行\red{\bf 归约}时, 计算相应节点的综合属性值
  \end{center}
\end{frame}
%%%%%%%%%%%%%%%%%%%%

%%%%%%%%%%%%%%%%%%%%
\begin{frame}{}
  \begin{center}
    \teal{\bf $S$ 属性定义}的依赖图描述了属性实例之间\teal{\bf 自底向上}的信息流

    \vspace{0.80cm}
    此类属性值的计算也可以在\purple{自顶向下}的语法分析\red{\bf 过程中}实现

    \pause
    \vspace{0.80cm}
    在$LL$语法分析器中, 递归下降函数$A$\red{\bf 返回}时, \\
    计算相应节点$A$的综合属性值
  \end{center}
\end{frame}
%%%%%%%%%%%%%%%%%%%%

%%%%%%%%%%%%%%%%%%%%
\begin{frame}{}
  \begin{center}
    $T'$有一个综合属性$syn$与一个继承属性$inh$

    \fig{width = 0.60\textwidth}{figs/SDD-expr-no-left-recursion}

    \begin{definition}[继承属性 (Inherited Attribute)]
      节点$N$上的\red{\bf 继承属性}只能通过\blue{$N$的父节点、$N$本身和$N$的兄弟节点}上的属性来定义。
    \end{definition}
  \end{center}
\end{frame}
%%%%%%%%%%%%%%%%%%%%

%%%%%%%%%%%%%%%%%%%%
\begin{frame}{}
  \begin{center}
    继承属性 \red{$T'.inh$} 用于在表达式中从左向右传递中间计算结果

    \vspace{0.50cm}
    \fig{width = 0.80\textwidth}{figs/anno-expr-no-left-recursion}
    \[
        \teal{3 \ast 5}
    \]

    % \begin{columns}
    %   \column{0.45\textwidth}
    %     \fig{width = 1.00\textwidth}{figs/SDD-expr-no-left-recursion}
    %   \column{0.55\textwidth}
    %     \fig{width = 1.00\textwidth}{figs/anno-expr-no-left-recursion}
    %     \[
    %       \teal{3 \ast 5}
    %     \]
    % \end{columns}

    \pause
    \blue{在右递归文法下实现了左结合}
  \end{center}
\end{frame}
%%%%%%%%%%%%%%%%%%%%

%%%%%%%%%%%%%%%%%%%%
\begin{frame}{}
  \begin{center}
    \red{\bf 依赖图}用于确定一棵\blue{给定的语法分析树}中\purple{各个属性实例}之间的依赖关系

    \vspace{0.50cm}
    \fig{width = 0.80\textwidth}{figs/dep-expr-no-left-recursion}

    % \begin{columns}
    %   \column{0.45\textwidth}
    %     \fig{width = 1.00\textwidth}{figs/anno-expr-no-left-recursion}
    %   \column{0.55\textwidth}
    %     \fig{width = 1.00\textwidth}{figs/dep-expr-no-left-recursion}
    % \end{columns}

    \vspace{0.50cm}
    \red{\bf 信息流向:} 先从左向右、从上到下传递信息, 再从下到上传递信息
  \end{center}
\end{frame}
%%%%%%%%%%%%%%%%%%%%

%%%%%%%%%%%%%%%%%%%%
\begin{frame}{}
  \begin{center}
    \begin{definition}[$L$ 属性定义 ($L$-Attributed Definition)]
      如果一个SDD的每个属性
      \begin{enumerate}[(1)]
        \setlength{\itemsep}{8pt}
        \item 要么是综合属性,
        \item 要么是继承属性, 但是它的规则满足如下限制: \\
          对于产生式 $A \to X_{1}X_{2} \dots X_{n}$ 及其对应规则定义的继承属性 $X_{i}.a$,
          则这个规则只能使用
          \begin{enumerate}[(a)]
            \setlength{\itemsep}{6pt}
            \item 和\red{\bf 产生式头 $A$}关联的\blue{\bf 继承}属性;
            \item 位于\red{\bf $X_{i}$左边}的文法符号实例 $X_{1}$、$X_{2}$、$\dots$、$X_{i-1}$
              相关的\blue{\bf 继承}属性或\blue{\bf 综合}属性;
            \item 和\red{\bf 这个$X_{i}$的实例本身}相关的继承属性或综合属性,
              但是在由这个$X_{i}$的全部属性组成的依赖图中\blue{\bf 不存在环}。
          \end{enumerate}
      \end{enumerate}
      则它是$L$属性定义。
    \end{definition}
  \end{center}
\end{frame}
%%%%%%%%%%%%%%%%%%%%

%%%%%%%%%%%%%%%%%%%%
\begin{frame}{}
  \begin{center}
    非$L$属性定义
    \fig{width = 0.60\textwidth}{figs/not-L-attribute}

    \vspace{0.80cm}
    作为继承属性, $B.i$ 依赖了右边的 $C.c$ 属性与头部 $A.s$ 综合属性
  \end{center}
\end{frame}
%%%%%%%%%%%%%%%%%%%%

%%%%%%%%%%%%%%%%%%%%
\begin{frame}{}
  \begin{center}
    \blue{$L$属性文法: 依赖图是无环的}

    \fig{width = 0.50\textwidth}{figs/SDD-expr-no-left-recursion}

    \pause
    \begin{columns}
      \column{0.20\textwidth}
      \column{0.60\textwidth}
        \fig{width = 1.00\textwidth}{figs/dep-expr-no-left-recursion}
      \column{0.20\textwidth}
    \end{columns}
  \end{center}
\end{frame}
%%%%%%%%%%%%%%%%%%%%

%%%%%%%%%%%%%%%%%%%%
\begin{frame}{}
  \begin{center}
    (左递归) $S$属性文法 $\implies$ (右递归) $L$属性文法

    \begin{columns}
      \column{0.50\textwidth}
        \fig{width = 1.00\textwidth}{figs/SDD-expr-left-recursion}
      \column{0.50\textwidth}
        \fig{width = 1.00\textwidth}{figs/SDD-expr-no-left-recursion}
    \end{columns}

    \vspace{0.60cm}
    仍保持了操作的左结合性
  \end{center}
\end{frame}
%%%%%%%%%%%%%%%%%%%%

%%%%%%%%%%%%%%%%%%%%
\begin{frame}{}
  \begin{center}
    (左递归) $S$属性文法
    \begin{alignat*}{3}
      A &\to A_{1} Y \qquad && A.a = g(A_{1}.a, Y.y) \\[8pt]
      A &\to X && A.a = f(X.x)
    \end{alignat*}

    \pause
    \vspace{0.80cm}
    (右递归) $L$属性文法
    \begin{alignat*}{3}
      A &\to XR \qquad && \red{R.i} = f(X.x);\quad A.a = R.s \\[8pt]
      R &\to YR_{1} && \red{R_{1}.i} = g(R.i, Y.y);\quad R.s = R_{1}.s \\[8pt]
      R &\to \epsilon && \blue{R.s = R.i}
    \end{alignat*}
  \end{center}
\end{frame}
%%%%%%%%%%%%%%%%%%%%

%%%%%%%%%%%%%%%%%%%%
\begin{frame}{}
  \begin{center}
    \red{\bf 类型声明}文法举例

    \fig{width = 0.70\textwidth}{figs/SDD-type-decl}
    \[
      \teal{\floatkw\; \id_{1}, \id_{2}, \id_{3}}
    \]
  \end{center}
\end{frame}
%%%%%%%%%%%%%%%%%%%%

%%%%%%%%%%%%%%%%%%%%
\begin{frame}{}
  \begin{center}
    \red{$L.inh$} 将声明的类型沿着标识符列表向下传递, 并被加入到符号表中

    \fig{width = 0.70\textwidth}{figs/anno-type-decl}
    \[
        \teal{\floatkw\; \id_{1}, \id_{2}, \id_{3}}
    \]

    % \begin{columns}
    %   \column{0.45\textwidth}
    %     \fig{width = 1.00\textwidth}{figs/SDD-type-decl}
    %   \column{0.55\textwidth}
    %     \fig{width = 1.05\textwidth}{figs/anno-type-decl}
    %     \[
    %       \teal{\floatkw\; \id_{1}, \id_{2}, \id_{3}}
    %     \]
    % \end{columns}

    \pause
    \vspace{0.30cm}
    \blue{$addType()$}是一种受控的副作用, 使用\red{全局}符号表, 更易实现
  \end{center}
\end{frame}
%%%%%%%%%%%%%%%%%%%%

%%%%%%%%%%%%%%%%%%%%
\begin{frame}{}
  \begin{center}
    \red{\bf 数组类型}文法举例
    \fig{width = 0.70\textwidth}{figs/SDD-type-array}
    \[
      \teal{\intkw[2][3]}
    \]

    \pause
    \vspace{0.30cm}
    语义规则用以生成\red{\bf 类型表达式} \blue{$array(2, array(3, integer))$}

    \pause
    \vspace{0.30cm}
    \teal{$[\;]$}是右结合的
  \end{center}
\end{frame}
%%%%%%%%%%%%%%%%%%%%

%%%%%%%%%%%%%%%%%%%%
\begin{frame}{}
  \begin{center}
    继承属性 \red{$C.b$} 将一个基本类型沿着树向下传播

    \fig{width = 0.80\textwidth}{figs/anno-type-array}
    \[
        \teal{\intkw[2][3]}
    \]

    % \begin{columns}
    %   \column{0.40\textwidth}
    %     \fig{width = 1.00\textwidth}{figs/SDD-type-array}
    %   \column{0.60\textwidth}
    % \end{columns}

    \vspace{0.30cm}
    综合属性 \blue{$C.t$} 收集最终得到的类型表达式
  \end{center}
\end{frame}
%%%%%%%%%%%%%%%%%%%%

%%%%%%%%%%%%%%%%%%%%
\begin{frame}{}
  \begin{center}
    表达式的\red{\bf 抽象语法树} $S$属性定义

    \fig{width = 0.90\textwidth}{figs/SDD-expr-ast-S}
  \end{center}
\end{frame}
%%%%%%%%%%%%%%%%%%%%

%%%%%%%%%%%%%%%%%%%%
\begin{frame}{}
  \begin{center}
    \red{\bf 抽象语法树:} 丢弃非本质的东西, 仅保留重要结构信息

    \fig{width = 0.70\textwidth}{figs/ast-S}
    \vspace{-0.50cm}
    \[
      \teal{a - 4 + c}
    \]
  \end{center}
\end{frame}
%%%%%%%%%%%%%%%%%%%%

%%%%%%%%%%%%%%%%%%%%
\begin{frame}{}
  \begin{center}
    表达式的\red{\bf 抽象语法树} $L$属性定义

    \fig{width = 0.80\textwidth}{figs/SDD-expr-ast-L}
  \end{center}
\end{frame}
%%%%%%%%%%%%%%%%%%%%

%%%%%%%%%%%%%%%%%%%%
\begin{frame}{}
  \begin{center}
    \fig{width = 0.90\textwidth}{figs/ast-L}
    \[
      \teal{a - 4 + c}
    \]
    % \begin{columns}
    %   \column{0.50\textwidth}
    %     \fig{width = 1.00\textwidth}{figs/SDD-expr-ast-L}
    %   \column{0.55\textwidth}
    % \end{columns}
  \end{center}
\end{frame}
%%%%%%%%%%%%%%%%%%%%

%%%%%%%%%%%%%%%%%%%%
\begin{frame}{}
  \begin{definition}[后缀表示 (Postfix Notation)]
    \begin{enumerate}[(1)]
      \setlength{\itemsep}{8pt}
      \item 如果$E$是一个变量或常量, 则$E$的后缀表示是$E$本身;
      \item 如果$E$是形如 $E_{1} \;\text{\bf op}\; E_{2}$的表达式,
        则$E$的后缀表示是$E'_{1} E'_{2} \text{\bf op}$,
        这里 $E'_{1}$ 和 $E'_{2}$ 分别是 $E_{1}$ 与 $E_{2}$ 的后缀表达式;
      \item 如果$E$是形如$(E_{1})$的表达式, 则$E$的后缀表示是$E_{1}$的后缀表示。
    \end{enumerate}
  \end{definition}

  \pause
  \begin{center}
    \[
      (9-5)+2 \implies 95-2+
    \]
    \[
      9-(5+2) \implies 952+-
    \]
    \pause
    \[
      952+-3\ast \implies\pause (9-(5+2)) \ast 3
    \]
  \end{center}
\end{frame}
%%%%%%%%%%%%%%%%%%%%

%%%%%%%%%%%%%%%%%%%%
\begin{frame}{}
  \begin{center}
    \red{\bf 后缀表达式} $S$ 属性定义

    \fig{width = 0.80\textwidth}{figs/SDD-expr-postfix}

    ``||''表示字符串的连接
  \end{center}
\end{frame}
%%%%%%%%%%%%%%%%%%%%

%%%%%%%%%%%%%%%%%%%%
\begin{frame}{}
  \fig{width = 0.80\textwidth}{figs/anno-expr-postfix}
  \[
    \teal{9 - 5 + 2}
  \]
\end{frame}
%%%%%%%%%%%%%%%%%%%%

%%%%%%%%%%%%%%%%%%%%
\begin{frame}{}
  \begin{center}
    \red{\bf 有符号二进制数}文法

    \fig{width = 0.90\textwidth}{figs/cfg-signed-binary}
    \[
      \teal{-101}
    \]
  \end{center}
\end{frame}
%%%%%%%%%%%%%%%%%%%%

%%%%%%%%%%%%%%%%%%%%
\begin{frame}{}
  \begin{center}
    \red{\bf 有符号二进制数} $L$ 属性定义

    \fig{width = 0.80\textwidth}{figs/SDD-signed-binary}
  \end{center}
\end{frame}
%%%%%%%%%%%%%%%%%%%%

%%%%%%%%%%%%%%%%%%%%
\begin{frame}{}
  \begin{center}
    \fig{width = 0.70\textwidth}{figs/anno-signed-binary}
    \vspace{-0.50cm}
    \[
      \teal{-101}
    \]
  \end{center}
\end{frame}
%%%%%%%%%%%%%%%%%%%%