% antlr.tex

%%%%%%%%%%%%%%%%%%%%
\begin{frame}{}
  \fig{width = 0.50\textwidth}{figs/talk-cheap}
\end{frame}
%%%%%%%%%%%%%%%%%%%%

%%%%%%%%%%%%%%%%%%%%
\begin{frame}{}
  \begin{center}
    {\Large \href{https://github.com/courses-at-nju-by-hfwei/2023-compilers-coding/blob/main/src/main/antlr/cymbol/Cymbol.g4}{Cymbol.g4}}

    \fig{width = 0.60\textwidth}{figs/factorial}
  \end{center}
\end{frame}
%%%%%%%%%%%%%%%%%%%%

%%%%%%%%%%%%%%%%%%%%
\begin{frame}{}
  \begin{center}
    \fig{width = 0.40\textwidth}{figs/call-graph}
    \vspace{-0.60cm}
    函数调用图 (Function Call Graph)
  \end{center}
\end{frame}
%%%%%%%%%%%%%%%%%%%%

%%%%%%%%%%%%%%%%%%%%
\begin{frame}{}
  \begin{center}
    {\Large \href{https://github.com/courses-at-nju-by-hfwei/2023-compilers-coding/blob/main/src/main/antlr/ifstat/IfStat.g4}{IfStat.g4}}
  \end{center}

  \begin{columns}
	\column{0.40\textwidth}
	  \fig{width = 0.70\textwidth}{figs/girl-woman}
    \begin{center}
      二义性文法
    \end{center}
	\column{0.60\textwidth}
    \pause
	  \fig{width = 1.00\textwidth}{figs/ifstat-g4}
  \end{columns}
\end{frame}
%%%%%%%%%%%%%%%%%%%%

%%%%%%%%%%%%%%%%%%%%
\begin{frame}{}
  \begin{center}
    {\Large \href{https://github.com/courses-at-nju-by-hfwei/2023-compilers-coding/blob/main/src/main/antlr/ifstat/IfStat.g4}{IfStat.g4}}
  \end{center}

  \fig{width = 0.50\textwidth}{figs/ifstat-g4}

  \pause
  \vspace{0.30cm}
  \fig{width = 0.80\textwidth}{figs/ifstat-open-matched-g4}

  \begin{center}
    {\Large \href{https://github.com/courses-at-nju-by-hfwei/2023-compilers-coding/blob/main/src/main/antlr/ifstat/IfStatOpenMatched.g4}{IfStatOpenMatched.g4}}
  \end{center}
\end{frame}
%%%%%%%%%%%%%%%%%%%%

%%%%%%%%%%%%%%%%%%%%
\begin{frame}{}
  \begin{center}
    {\Large \href{https://github.com/courses-at-nju-by-hfwei/2023-compilers-coding/blob/main/src/main/antlr/expr/Expr.g4}{Expr.g4}}
  \end{center}

  \fig{width = 0.30\textwidth}{figs/expr-g4}
  \begin{center}
    运算符的\blue{\bf 结合性}带来的\red{\bf 二义性}
  \end{center}
\end{frame}
%%%%%%%%%%%%%%%%%%%%

%%%%%%%%%%%%%%%%%%%%
\begin{frame}{}
  \begin{center}
    {\Large \href{https://github.com/courses-at-nju-by-hfwei/2023-compilers-coding/blob/main/src/main/antlr/expr/ExprAssoc.g4}{ExprAssoc.g4}}

    \vspace{0.30cm}
    \fig{width = 0.60\textwidth}{figs/expr-assoc}

    \vspace{0.30cm}
    右结合运算符、前缀运算符与后缀运算符的结合性
  \end{center}
\end{frame}
%%%%%%%%%%%%%%%%%%%%

%%%%%%%%%%%%%%%%%%%%
\begin{frame}{}
  \begin{center}
    {\Large \href{https://github.com/courses-at-nju-by-hfwei/compilers-antlr/blob/main/src/main/antlr/expr/Expr.g4}{Expr.g4}}
  \end{center}

  \fig{width = 0.30\textwidth}{figs/expr-g4}

  \begin{center}
    运算符的\blue{\bf 优先级}带来的\red{\bf 二义性}
  \end{center}
\end{frame}
%%%%%%%%%%%%%%%%%%%%

%%%%%%%%%%%%%%%%%%%%
\begin{frame}{}
  \begin{columns}
    \column{0.50\textwidth}
    \uncover<2->{
      \begin{center}
        {\Large \href{https://github.com/courses-at-nju-by-hfwei/2023-compilers-coding/blob/main/src/main/antlr/expr/ExprLR.g4}{ExprLR.g4}}
        \fig{width = 0.60\textwidth}{figs/expr-lr-g4}
        左递归 (左结合)
      \end{center}
    }
    \column{0.50\textwidth}
      \fig{width = 0.60\textwidth}{figs/expr-g4}
      \vspace{0.20cm}
      \begin{center}
        ANTLR 4 可以处理该文法
      \end{center}
  \end{columns}
\end{frame}
%%%%%%%%%%%%%%%%%%%%

%%%%%%%%%%%%%%%%%%%%
\begin{frame}{}
  \begin{columns}
    \column{0.50\textwidth}
      \fig{width = 0.60\textwidth}{figs/expr-g4}
      \vspace{0.20cm}
      \begin{center}
        ANTLR 4 可以处理该文法
      \end{center}
    \column{0.50\textwidth}
      \pause
      \begin{center}
        {\Large \href{https://github.com/courses-at-nju-by-hfwei/2023-compilers-coding/blob/main/src/main/antlr/expr/ExprRR.g4}{ExprRR.g4}}
        \fig{width = 1.00\textwidth}{figs/expr-rr-g4}
        右递归 (右结合)
      \end{center}
  \end{columns}
\end{frame}
%%%%%%%%%%%%%%%%%%%%

%%%%%%%%%%%%%%%%%%%%
\begin{frame}{}
  \begin{center}
    {\Large \href{https://github.com/courses-at-nju-by-hfwei/2023-compilers-coding/tree/main/src/main/java/callgraph}{Call Graphs}}
  \end{center}

  \vspace{-0.80cm}
	\fig{width = 0.40\textwidth}{figs/call-graph}
\end{frame}
%%%%%%%%%%%%%%%%%%%%

%%%%%%%%%%%%%%%%%%%%
\begin{frame}{}
  \begin{center}
    \teal{\texttt{ParseTreeWalker}} 与 \teal{\texttt{Listener}}
  \end{center}
  \fig{width = 0.90\textwidth}{figs/ParseTreeWalker}
\end{frame}
%%%%%%%%%%%%%%%%%%%%

%%%%%%%%%%%%%%%%%%%%
\begin{frame}{}
  \fig{width = 0.65\textwidth}{figs/Visitor}
\end{frame}
%%%%%%%%%%%%%%%%%%%%