% overview.tex

%%%%%%%%%%%%%%%%%%%%
\begin{frame}{}
  \begin{center}
    \red{\bf 输入:} 词法单元流 \& \blue{\bf 语言的语法规则}

    \vspace{0.50cm}
    \fig{width = 0.90\textwidth}{figs/lexer-parser}

    \vspace{0.30cm}
    \red{\bf 输出:} 语法分析树 (Parse Tree)
  \end{center}
\end{frame}
%%%%%%%%%%%%%%%%%%%%

%%%%%%%%%%%%%%%%%%%%
\begin{frame}{}
  \begin{center}
    语法分析\blue{\bf 举例}

    \fig{width = 0.95\textwidth}{figs/statement-wiki}
  \end{center}
\end{frame}
%%%%%%%%%%%%%%%%%%%%

%%%%%%%%%%%%%%%%%%%%
\begin{frame}{}
  \begin{center}
    语法分析阶段的主题之一: \red{\bf 上下文无关文法}

    \fig{width = 0.55\textwidth}{figs/cfg-statement-wiki}
  \end{center}
\end{frame}
%%%%%%%%%%%%%%%%%%%%

%%%%%%%%%%%%%%%%%%%%
\begin{frame}{}
  \begin{center}
    语法分析阶段的主题之二: \red{\bf 构建语法分析树}

    \fig{width = 0.85\textwidth}{figs/tree-statement-wiki}
  \end{center}
\end{frame}
%%%%%%%%%%%%%%%%%%%%

%%%%%%%%%%%%%%%%%%%%
\begin{frame}{}
  \begin{center}
    语法分析阶段的主题之三: \red{\bf 错误恢复}

    \fig{width = 0.45\textwidth}{figs/keep-calm-recovery}

    报错、\blue{\bf 恢复}、继续分析
  \end{center}
\end{frame}
%%%%%%%%%%%%%%%%%%%%