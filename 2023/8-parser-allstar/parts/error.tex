% error.tex

%%%%%%%%%%%%%%%%%%%%
\begin{frame}{}
  \begin{center}
    \Large{ANTLR 4 是如何进行\purple{\bf 错误报告与恢复}的?}
  \end{center}
\end{frame}
%%%%%%%%%%%%%%%%%%%%

%%%%%%%%%%%%%%%%%%%%
\begin{frame}{}
  \begin{center}
    \fig{width = 0.45\textwidth}{figs/keep-calm-recovery}

    \vspace{0.20cm}
    报错、\blue{\bf 恢复}、继续分析
  \end{center}
\end{frame}
%%%%%%%%%%%%%%%%%%%%

%%%%%%%%%%%%%%%%%%%%
\begin{frame}{}
  \fig{width = 0.50\textwidth}{figs/panic}

  \vspace{0.30cm}
  \begin{center}
    \blue{\bf 恐慌/应急(Panic)模式:} 假装成功、调整状态、继续进行
  \end{center}
\end{frame}
%%%%%%%%%%%%%%%%%%%%

%%%%%%%%%%%%%%%%%%%%
\begin{frame}{}
\end{frame}
%%%%%%%%%%%%%%%%%%%%

%%%%%%%%%%%%%%%%%%%%
\begin{frame}{}
  \begin{center}
    四类词法、语法错误 \\[8pt]

    \fig{width = 0.95\textwidth}{figs/RecognitionException}

    \texttt{NoViableAltException} \qquad \texttt{InputMismatchException}
  \end{center}
\end{frame}
%%%%%%%%%%%%%%%%%%%%

%%%%%%%%%%%%%%%%%%%%
\begin{frame}{}
  \begin{center}
    \texttt{\bf \red{LexerNoViableAltException}} \\[20pt]
  \end{center}
\end{frame}
%%%%%%%%%%%%%%%%%%%%

%%%%%%%%%%%%%%%%%%%%
\begin{frame}{}
  \texttt{\bf \red{NoViableAltException}} \\[20pt]
\end{frame}
%%%%%%%%%%%%%%%%%%%%

%%%%%%%%%%%%%%%%%%%%
\begin{frame}{}
  \texttt{\bf \red{InputMismatchException}} \\[20pt]
\end{frame}
%%%%%%%%%%%%%%%%%%%%

%%%%%%%%%%%%%%%%%%%%
\begin{frame}{}
\end{frame}
%%%%%%%%%%%%%%%%%%%%

%%%%%%%%%%%%%%%%%%%%
\begin{frame}{}
  \begin{center}
    如果\red{\bf 下一个词法单元}符合预期, \\[8pt]
    则采用``\blue{单词法符号移除} (single-token deletion)'' \\[8pt]
    或``\blue{单词法符号补全} (single-token insertion)''策略

    \vspace{1.00cm}
    \texttt{Class.g4}
  \end{center}
\end{frame}
%%%%%%%%%%%%%%%%%%%%

%%%%%%%%%%%%%%%%%%%%
\begin{frame}{}
  \begin{center}
    \texttt{Class-DeleteToken.txt}

    \vspace{0.30cm}
    \fig{width = 0.50\textwidth}{figs/class-parse-tree-deletetoken}
  \end{center}
\end{frame}
%%%%%%%%%%%%%%%%%%%%

%%%%%%%%%%%%%%%%%%%%
\begin{frame}{}
  \begin{center}
    \texttt{Class-AddToken.txt}

    \vspace{0.30cm}
    \fig{width = 0.50\textwidth}{figs/class-parse-tree-addtoken}
  \end{center}
\end{frame}
%%%%%%%%%%%%%%%%%%%%

%%%%%%%%%%%%%%%%%%%%
\begin{frame}{}
  \begin{center}
    采用``\red{\bf 同步-返回 (sync-and-return)}'' 策略, \\[5pt]
    使用``\blue{重新同步集合 (resynchronization set)}''从\purple{当前规则}中恢复

    \vspace{1.00cm}
    \texttt{Group.g4} \\[15pt]
  \end{center}
\end{frame}
%%%%%%%%%%%%%%%%%%%%

%%%%%%%%%%%%%%%%%%%%
\begin{frame}{}
  \[
    \following(\set{\texttt{expr}, \texttt{atom}}) = \set{\hat{\quad}, \texttt{]}} \qquad
    \following(\set{\texttt{expr}}) = \set{\texttt{]}}
  \]
  \begin{columns}
    \column{0.50\textwidth}
      \fig{width = 0.50\textwidth}{figs/group-parse-tree-i^2}
    \column{0.50\textwidth}
      \fig{width = 0.40\textwidth}{figs/group-parse-tree-[]}
  \end{columns}

  \pause
  \vspace{0.60cm}
  \begin{center}
    \purple{注意 \textsc{Follow} (静态)集合与 \textsc{Following} (动态)集合的区别}
  \end{center}
\end{frame}
%%%%%%%%%%%%%%%%%%%%

%%%%%%%%%%%%%%%%%%%%
\begin{frame}{}
  \begin{center}
    \red{\bf 如何从子规则中优雅地恢复出来?} \\[20pt]

    \texttt{Class.g4} (\texttt{\blue{member+}}) \\[20pt]
    \texttt{Class-Subrule-Start.txt} (\blue{``单词法符号移除''}) \\[10pt]
    \texttt{Class-Subrule-Loop.txt} (\blue{``另一次\texttt{member}迭代''}) \\[10pt]
    \texttt{Class-Subrule-End.txt} (\blue{``退出当前\texttt{classDef}规则''})
  \end{center}
\end{frame}
%%%%%%%%%%%%%%%%%%%%

%%%%%%%%%%%%%%%%%%%%
\begin{frame}{}
  \fig{width = 0.40\textwidth}{figs/antlr4-book}

  \begin{center}
    第 9 章: 错误报告与恢复
  \end{center}
\end{frame}
%%%%%%%%%%%%%%%%%%%%