% overview.tex

%%%%%%%%%%%%%%%%%%%%
\begin{frame}{}
  \begin{center}
    \fig{width = 0.35\textwidth}{figs/antlr4-book}
  \end{center}

  \begin{enumerate}[(1)]
    \setlength{\itemsep}{8pt}
    \pause
    \item ANTLR 4 自动将类似 \texttt{expr} 的\red{左递归}规则重写成非左递归形式
    \pause
    \item ANTLR 4 提供优秀的\red{错误报告}功能和复杂的\red{错误恢复}机制
    \pause
    \item ANTLR 4 使用了一种名为 \red{Adaptive $LL(\ast)$} 的新技术
    \pause
    \item ANTLR 4 几乎能处理\red{任何文法} (二义性文法\cmark \;\; 间接左递归\xmark)
  \end{enumerate}
\end{frame}
%%%%%%%%%%%%%%%%%%%%

%%%%%%%%%%%%%%%%%%%%
% \begin{frame}{}
%   \begin{columns}
%     \column{0.80\textwidth}
%     \begin{enumerate}[(1)]
%       \setlength{\itemsep}{15pt}
%       \item ANTLR 4 是如何处理\blue{\bf 直接左递归}的?
%       \item ANTLR 4 是如何进行\purple{\bf 错误报告与恢复}的?
%       \item ANTLR 4 使用的 \red{\bf $ALL(\ast)$} 算法是如何工作的?
%       \item ANTLR 4 是如何判断并消解\cyan{\bf 二义性输入/文法}的?
%     \end{enumerate}
%   \end{columns}
% \end{frame}
%%%%%%%%%%%%%%%%%%%%